\documentclass{article}
\usepackage{mathtext} % использование кириллицы в формулах
\usepackage{cmap} % грамотное копирование кириллицы из pdf
\usepackage[T2A]{fontenc} % внутрення кодировка
\usepackage[utf8]{inputenc} % кодировка документа
\usepackage[russian]{babel} % язык документа
\usepackage{amssymb} % дополнительные символы
\usepackage{amsfonts} % математические шрифты
\usepackage{amsmath} % дополнительная математика
\usepackage{ifthen}
\makeatletter
\newcounter{ticket}[subsection]
\newenvironment{ticket}[1][]{\item[Билет \ifthenelse{\equal{#1}{}}{}{\setcounter{ticket}{#1}}\theticket\refstepcounter{ticket}:]\phantom{}\begin{enumerate}}{\end{enumerate}}
\newcounter{Ticket}[subsection]
\newcommand{\Ticket}[1][]{\item[Билет \ifthenelse{\equal{#1}{}}{}{\setcounter{ticket}{#1}}\theticket\refstepcounter{ticket}:]}
\makeatother
\title{Вопросы по математическому анализу \\ 2 семестр}
\date{Последнее обновление: \today}
\author{Национальный исследовательский \\
Нижегородский Государственный Университет \\
имени Н.И. Лобачевского \vspace{0.5em} \\
Высшая Школа Общей и Прикладной Физики \vspace{0.5em}}
\begin{document}
\maketitle
\begin{enumerate}
    \item Несобственные интегралы 1-го и 2-го рода (определения,
          примеры).
    \item Основные свойства несобственных интегралов.
    \item Критерий Коши сходимости несобственных интегралов.
    \item Критерий сходимости несобственного интеграла от
          положительной функции.
    \item Признаки сравнения для несобственных интегралов от
          положительных функций.
    \item Признак Абеля сходимости несобственных интегралов.
    \item Признак Дирихле сходимости несобственных интегралов.
    \item Пространство $\mathbb{R}^n$: основные определения и свойства операций.
    \item Неравенство Коши-Буняковского-Шварца.
    \item Метрические пространства (определения и примеры).
    \item Последовательности и пределы в метрических пространствах.
    \item Множества в метрическом пространстве (открытые, замкнутые).
    \item Точки (внутренние, внешние, предельные, граничные) множеств
          в метрическом пространстве.
    \item Компактные множества в метрических пространствах.
          Необходимое условие компактности.
    \item Критерий компактности в $\mathbb{R}^n$.
    \item Функции на метрических пространствах, непрерывность,
          различные эквивалентные определения.
    \item Образ компактного множества при непрерывном отображении.
          Обобщение 1-й и 2-й теоремы Вейершрасса.
    \item Равномерная непрерывность. Обобщение теоремы Кантора для
          непрерывной функции на компакте.
    \item Связность, линейная связность. Образ связного множества при
          непрерывном отображении.20. Покоординатная сходимость в R n.
    \item Дифференцируемость функций нескольких переменных,
          определение оператора дифференцирования.
    \item Частные производные. Матрица Якоби. Производная по
          направлению и градиент.
    \item Необходимые условие
          дифференцируемости (непрерывность;
          существование частных производных).
    \item Линии уровня функции, связь с градиентом.
    \item Свойства дифференцируемых функций. Действия с ними.
    \item Нормированные пространства (определение и примеры). Норма
          линейного оператора.
    \item Теорема о дифференцируемости композиции дифференцируемых
          функций.
    \item Следствия из теоремы о дифференцируемости композиции.
          Инвариантность формы 1-го дифференциала.
    \item Достаточное условие дифференцируемости в терминах частных
          производных.
    \item Касательная плоскость и её уравнение.
    \item Производные высших порядков. Совпадение смешанных
          частных производных.
    \item Дифференциалы высших порядков и их вычисление.
          Неинвариантность формы второго дифференциала.
    \item Формула Тейлора для функции нескольких переменных.
    \item Экстремумы функции нескольких переменных, необходимое
          условие экстремума.
    \item Достаточные условия экстремума.
    \item Теорема о неявной функции в случае, определяющем функцию
          одной переменной.
    \item Теорема о неявной функции в случае, определяющем функцию
          нескольких переменных.
    \item Теорема о неявной функции в общем (многомерном) случае
          отображения.
    \item Теорема об обратной функции.
    \item Замена переменных; полярные и сферические координаты.
    \item Условные экстремумы: необходимое условие, примеры,
    \item Метод множителей Лагранжа для условного экстремума.
    \item Ряды с постоянными членами: частичные суммы и остаток ряда,
          критерий сходимости Коши, необходимое условие сходимости.
    \item Арифметические свойства сходящихся рядов. Геометрическая
          прогрессия и её сумма.
    \item Критерий сходимости положительного ряда.
    \item Первый признак сравнения (через неравенства) для
          положительного ряда.
    \item Второй признак сравнения (через пределы) для положительного
          ряда
    \item Признак сходимости Коши для положительного ряда.
    \item Признак сходимости Даламбера для положительного ряда.
    \item Интегральный признак сходимости.
    \item Гармонический и обобщенный гармонический ряды.
    \item Признак Раабе и признак Гаусса.
    \item Абсолютная сходимость знакопеременного ряда.
    \item Признак Лейбница.
    \item Преобразование Абеля. Признак Абеля.
    \item Признак Дирихле. Пример ряда $\sum\limits_{n=1}^{+\infty}\dfrac{sin\ x}{n^\alpha}$.
    \item Ассоциативность и перестановки членов сходящегося ряда.
          Теорема Римана.
    \item Равномерная сходимость функциональных последовательностей и
          рядов. Критерий Коши.
    \item Признак Вейерштрасса равномерной сходимости. Примеры.
    \item Признаки Абеля и Дирихле равномерной сходимости.
    \item Равномерная сходимость и непрерывность.
    \item Теорема (признак) Дини (достаточное условие равномерной
          сходимости).
    \item Равномерная сходимость и интегрирование .
    \item Равномерная сходимость и дифференцирование.
    \item Степенные ряды. Область сходимости степенного ряда (теорема
          об интервале сходимости).
    \item Формула Коши Адамара.
    \item Функциональные свойства суммы степенного ряда.
    \item Теорема Абеля о поведении степенного ряда на границе интервала
          сходимости.
    \item Разложение функций в степенной ряд. Необходимое условие
          разложения.
    \item Достаточное условие разложения в ряд Тейлора .
    \item Разложение в ряд Тейлора функций: $y=e^x$, $sin(x)$, $cos(x)$.
    \item Разложение в ряд Тейлора функции $y=(1+x)^a$.
    \item Разложение в ряд Тейлора функций: $y=ln(1+x)$, $arctg(x)$.
    \item Равномерная сходимость семейства функций: определение,
          критерий Коши
    \item Непрерывные свойства предела равномерно сходящегося
          семейства функций
    \item Теорема Дини.
    \item Переход к пределу под знаком интеграла для семейства функций
    \item Перестановка интегралов для семейства функций
    \item Дифференцирование по параметру под знаком интеграла.
    \item Дифференцирование по параметру под знаком интеграла с
          переменными пределами интегрирования.
    \item Равномерная сходимость несобственных интегралов, зависящих от
          параметра. Аналог теоремы Вейерштрасса84. Аналог признаков Абеля-Дирихле равномерной сходимости
          несобственных интегралов
    \item Переход к пределу для семейства несобственных интегралов
    \item Дифференцирование несобственных интегралов по параметру
    \item Перестановка несобственных интегралов
    \item В-функция и ее свойства.
    \item Г-функция и ее свойства: график Г-функции, связь В- и Г-
          функций, формула дополнения
    \item Ряды Фурье. Коэффициенты тригонометрического ряда Фурье
    \item Ряды Фурье в евклидовом пространстве. Оптимальность ряда
          Фурье.
    \item Неравенство Бесселя и равенство Парсеваля.
    \item Различные типы сходимости тригонометрического ряда Фурье
          (сходимость в среднем квадратическом, поточечная и равномерная
          сходимость)
    \item Теоремы Вейерштрасса об аппроксимации непрерывной функции
          полиномами. Полнота тригонометрической системы функций.
    \item Некоторые достаточные условия сходимости тригонометрического
          ряда Фурье
    \item Интеграл Фурье и преобразование Фурье. Примеры.
\end{enumerate}
\end{document}