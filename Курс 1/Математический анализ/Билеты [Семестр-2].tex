\documentclass{article}
\usepackage{mathtext} % использование кириллицы в формулах
\usepackage{cmap} % грамотное копирование кириллицы из pdf
\usepackage[T2A]{fontenc} % внутрення кодировка
\usepackage[utf8]{inputenc} % кодировка документа
\usepackage[russian]{babel} % язык документа
\usepackage{amssymb} % дополнительные символы
\usepackage{amsfonts} % математические шрифты
\usepackage{amsmath} % дополнительная математика
\usepackage{ifthen}
\makeatletter
\newcounter{ticket}[subsection]
\newenvironment{ticket}[1][]{\item[Билет \ifthenelse{\equal{#1}{}}{}{\setcounter{ticket}{#1}}\theticket\refstepcounter{ticket}:]\phantom{}\begin{enumerate}}{\end{enumerate}}
\newcounter{Ticket}[subsection]
\newcommand{\Ticket}[1][]{\item[Билет \ifthenelse{\equal{#1}{}}{}{\setcounter{ticket}{#1}}\theticket\refstepcounter{ticket}:]}
\makeatother
\title{Билеты по математическому анализу \\ 2 семестр}
\date{Последнее обновление: \today}
\author{Национальный исследовательский \\
Нижегородский Государственный Университет \\
имени Н.И. Лобачевского \vspace{0.5em} \\
Высшая Школа Общей и Прикладной Физики \vspace{0.5em}}
\begin{document}
\maketitle
\begin{description}
	\begin{ticket}[1]
		\item Признаки сравнения сходимости несобственных интегралов.
		\item Формула Тейлора для функции нескольких переменных.
		\item Переход к пределу под знаком интеграла для
		      семейства фукнций.
	\end{ticket}
	\begin{ticket}[4]
		\item Компактные множества в метрических пространствах.
		      Необходимое условие компактности.
		\item Признак Раабе.
		\item Ряды Фурье. Коэффициенты тригонометрического ряда Фурье.
	\end{ticket}
	\begin{ticket}
		\item Связность и линейная связность. Образ связного множества
		      при непрерывном отображении.
		\item Признак Дирихле для числового ряда.
		\item Переход к пределу под знаком интеграла для
		      семейства функций.
	\end{ticket}
	\begin{ticket}
		\item Критерий компактности в $\mathbb{R}^n$.
		\item Признак Даламбера сходимости положительного ряда.
		\item Дифференцируемость интеграла, зависящего от параметра.
	\end{ticket}
	\begin{ticket}
		\item Образ компакта при непрерывном отображении.
		\item Формула Коши-Адамара.
		\item Бета-функция и её свойства.
	\end{ticket}
	\begin{ticket}
		\item Достаточное условие дифференцируемости в терминах
		      частных производных.
		\item Критерий сходимости положительного ряда.
		\item Равенство Парсеваля и неравенство Бесселя.
	\end{ticket}
	\begin{ticket}
		\item Теорема о дифференцируемости композиции дифференцируемых
		      функций.
		\item Равномерная сходимость и интегрирование.
		\item Разложение в ряд Тейлора: $ln(1 + x)$, $arc tg(x)$.
	\end{ticket}
	\begin{ticket}
		\item Связность и линейная связность. Образ связного множества
		      при непрерывном отображении.
		\item Достаточное условие абсолютного экстремума.
		\item Интегральный признак сходимости.
	\end{ticket}
	\begin{ticket}
		\item Инвариантность первого дифференциала.
		\item Совпадение смешанных частных производных.
		\item Равномерная сходимость несобственных интегралов,
		      зависящих от параметра. Аналог теоремы Вейерштрасса.
	\end{ticket}
	\begin{ticket}
		\item Равномерная непрерывность. Обобщение теоремы Кантора.
		\item Теорема Абеля о поведении степенного ряда на границе
		      интервала сходимости.
		\item Признак Дини.
	\end{ticket}
\end{description}
\end{document}