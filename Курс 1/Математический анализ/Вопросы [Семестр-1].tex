\documentclass{article}
\usepackage{mathtext} % использование кириллицы в формулах
\usepackage{cmap} % грамотное копирование кириллицы из pdf
\usepackage[T2A]{fontenc} % внутрення кодировка
\usepackage[utf8]{inputenc} % кодировка документа
\usepackage[russian]{babel} % язык документа
\usepackage{amssymb} % дополнительные символы
\usepackage{amsfonts} % математические шрифты
\usepackage{amsmath} % дополнительная математика
\usepackage{ifthen}
\makeatletter
\newcounter{ticket}[subsection]
\newenvironment{ticket}[1][]{\item[Билет \ifthenelse{\equal{#1}{}}{}{\setcounter{ticket}{#1}}\theticket\refstepcounter{ticket}:]\phantom{}\begin{enumerate}}{\end{enumerate}}
\newcounter{Ticket}[subsection]
\newcommand{\Ticket}[1][]{\item[Билет \ifthenelse{\equal{#1}{}}{}{\setcounter{ticket}{#1}}\theticket\refstepcounter{ticket}:]}
\makeatother
\title{Вопросы по математическому анализу \\ 1 семестр}
\date{Последнее обновление: \today}
\author{Национальный исследовательский \\
Нижегородский Государственный Университет \\
имени Н.И. Лобачевского \vspace{0.5em} \\
Высшая Школа Общей и Прикладной Физики \vspace{0.5em}}
\begin{document}
\maketitle
\begin{enumerate}
    \item Отношение порядка. Определения и примеры.
    \item Отношение эквивалентности. Определение и примеры.
    \item Множества и операции над множествами. 
    \item Мощность множеств. Счетные множества и множества мощности континуум. 
    \item Функции (отображения) и их виды: сюръекция, инъекция, биекция. Примеры. 
    \item Композиция отображений (сложная функция). Примеры. 
    \item Обратная функция. Примеры. 
    \item Натуральные числа: Математическая индукция, примеры. Операции на N и их свойства. 
    \item Целые числа. Модели множества целых чисел. Операции на Z и их свойства. 
    \item Рациональные числа. Операции в Q и их свойства. 
    \item Свойства неравенств. Соотношение для разности степеней двух чисел. 
    \item Бином Ньютона. 
    \item Неполнота множества рациональных чисел и алгебраическая незамкнутость. 
    \item Ограниченные и неограниченные множества. Примеры. 
    \item Определение точной верхней и точной нижней грани. 
    \item Действительные числа. Аксиоматическое определение, разные формы полноты R 
    \item Модель сечений Дедекинда. Определение операций и порядка в этой модели 
    \item Другие модели R (числовая прямая, бесконечные десятичные дроби). 
    \item Теорема о существовании арифметического корня . 
    \item Теорема о вложенных отрезках. 
    \item Графики функций. Преобразование графиков. 
    \item Типы числовых функций: ограниченные, монотонные. 
    \item Типы числовых функций: четные (нечетные), периодические. 
    \item Алгебраические свойства степеней с рациональным показателем. 
    \item Степенные функции и их графики. 
    \item Показательная и логарифмическая функции. 
    \item Тригонометрические и обратные тригонометрические функции. 
    \item Последовательности и их типы. 
    \item Формулы общего члена и суммы для арифметической и геометрической прогрессии 
    \item Определение предела последовательности. Примеры. Доказательства по определению. 
    \item Теорема о единственности предела последовательности. 
    \item Необходимое условие сходимости (теорема об ограниченности сходящейся последовательности)
    \item Бесконечно малые последовательности и бесконечно большие последовательности; их свойства. 
    \item Теорема об арифметических свойствах предела последовательности. 
    \item Переход к пределу в неравенствах. 
    \item Теорема «О двух милиционерах». 
    \item Теоремам Вейерштрасса о монотонной последовательности. 
    \item Число е. 
    \item Теорема Больцано-Вейерштрасса о под последовательностях. 
    \item Частичные пределы последовательности. Верхние и нижние пределы последовательности и их свойства. 
    \item Критерий сходимости Коши. 
    \item Предел функции. Определение на языке «эпсилон-дельта» (по Коши) и на языке последовательностей (по Гейне). Примеры. Доказательства по определению. 
    \item Эквивалентность определений предела по Коши и по Гейне. 
    \item Арифметические свойства предела функции. 
    \item Односторонние пределы. Бесконечные пределы. 
    \item Первый замечательный предел. 
    \item Второй замечательный предел. 
    \item Критерий Коши для предела функции. Переход к пределу функции в неравенствах. 
    \item Сравнение бесконечно малых функций. 
    \item Сравнение бесконечно больших функций и последовательностей. 
    \item Свойство предела монотонной функции. 
    \item Непрерывность функции. Определение и примеры. 
    \item Непрерывность арифметических операций и композиции непрерывных функций. 
    \item Точки разрыва и их классификация. 
    \item 1-ая Теорема Больцано –Коши. 
    \item 2-ая Теорема Больцано – Коши. 
    \item 1-ая Теорема Вейерштрасса. 
    \item 2-ая Теорема Вейерштрасса. 
    \item Теорема о непрерывности обратной функции. 
    \item Непрерывность основных элементарных функций. 
    \item Равномерная непрерывность. Теорема Кантора
    \item Задачи, приводящие к понятию производной. Определение производной. 
    \item Физический и геометрический смысл производной. Уравнение касательной. 
    \item Определение дифференцируемости функции в точке. Дифференциал и его геометрический смысл. 
    \item Дифференцируемость и существование производной. Эквивалентность этих понятий 
    \item Свойства производных (производная суммы, разности, произведения, частного). 
    \item Производная сложной функции. Производная обратной функции. Примеры
    \item Производная элементарных функций
    \item Лемма о возрастании функции в точке
    \item Теорема Ферма об экстремуме
    \item Теорема Ролля 
    \item Теорема (формула) Лагранжа 
    \item Следствия из теоремы Лагранжа (о монотонности) 
    \item Односторонние производные. Существование односторонних производных (следствие из теоремы Лагранжа).
    \item Теорема Дарбу 
    \item Теорема Коши
    \item Параметрический заданные функции и их производные
    \item Производные высших порядков для основных элементарных функций 
    \item Формула Лейбница 
    \item Инвариантность формы первого дифференциала и не инвариантность высших дифференциалов 
    \item Формула Тейлора для многочлена
    \item Формула Тейлора с остаточным членом в форме Пеано 
    \item Разложение Тейлора-Маклорена для $е^х$, $sin\ x$, $cos\ x$
    \item Разложение Тейлора-Маклорена для $ln(1+x)$, $(1+x)^a$
    \item Формула Тейлора с остаточным членом в форме Лагранжа и Кош 
    \item Правило Лопиталя для неопределённости
    \item Правило Лопиталя для неопределённости
    \item Выпуклые (вогнутые) функции. Определение и эквивалентные условия 
    \item Эквивалентные условия выпуклости для дифференцируемых функций 
    \item Условия выпуклости в терминах касательных 
    \item Неравенство Иенсена. Примеры 
    \item Асимптоты графика функции и их уравнения 
    \item Необходимые и достаточные условия экстремума (в терминах п. 
    \item Формула Тейлора с остаточным членом в форме Пеано 
    \item Разложение Тейлора-Маклорена для $e^x$, $sin\ x$, $cos\ x$. 
    \item Разложение Тейлора-Маклорена для $ln(1+x)$, $(1+x)^a$. 
    \item Формула Тейлора с остаточным членом в форме Лагранжа и Коши 
    \item Неопределенный интеграл и первообразная: определение, описание множества первообразных. 
    \item Свойства первообразных (линейность, замена переменных, интегрирование по частям). 
    \item Интегрирование рациональных функций. 
    \item Интегрирование некоторых иррациональных функций. 
    \item Интегралы с подстановками Эйлера. 
    \item Биномиальный дифференциал и его интегрирование. 
    \item Интегрирование тригонометрических функций. 
    \item Задачи, приводящие к понятию определенного интеграла. 
    \item Определение интеграла Римана. Необходимое условие интегрируемости. 
    \item Суммы Дарбу и их свойства. 
    \item Критерий интегрируемости (эквивалентные условия в терминах колебаний). 
    \item Достаточные условия интегрируемости (для непрерывных функций и функций с конечным множеством точек разрыва). 
    \item Достаточные условия интегрируемости для монотонных функций. 
    \item Множества меры 0 и их свойства. Критерий Лебега интегрируемости (без доказательства). 
    \item Теорема об интегрируемости суммы и произведения функций, интегрируемость модуля от функции. 
    \item Линейные свойства определенного интеграла. 
    \item Свойства неравенств для определенных интегралов. 
    \item Аддитивное свойство определенного интеграла. 
    \item Первая теорема о среднем и ее обобщение. Вторая теорема о среднем (без доказательства). 
    \item Свойства интеграла с переменным верхним пределом. 
    \item Формула Ньютона – Лейбница.
    \item Интегрирование по частям и замена переменных для определенного интеграла. 
    \item Формула Тейлора с интегральным остаточным членом. 
    \item Понятия квадрируемой фигуры и площади. Площадь в декартовых и полярных координатах. 
    \item Длина кривой, заданной параметрически и в явном виде. 
    \item Объем тела вращения. 
    \item Площадь поверхности вращения. 
    \item Физические приложения определенного интеграла.
\end{enumerate}
\end{document}