\documentclass{article}
\usepackage{mathtext} % использование кириллицы в формулах
\usepackage{cmap} % грамотное копирование кириллицы из pdf
\usepackage[T2A]{fontenc} % внутрення кодировка
\usepackage[utf8]{inputenc} % кодировка документа
\usepackage[russian]{babel} % язык документа
\usepackage{amssymb} % дополнительные символы
\usepackage{amsfonts} % математические шрифты
\usepackage{amsmath} % дополнительная математика
\usepackage{ifthen}
\makeatletter
\newcounter{ticket}[subsection]
\newenvironment{ticket}[1][]{\item[Билет \ifthenelse{\equal{#1}{}}{}{\setcounter{ticket}{#1}}\theticket\refstepcounter{ticket}:]\phantom{}\begin{enumerate}}{\end{enumerate}}
\newcounter{Ticket}[subsection]
\newcommand{\Ticket}[1][]{\item[Билет \ifthenelse{\equal{#1}{}}{}{\setcounter{ticket}{#1}}\theticket\refstepcounter{ticket}:]}
\makeatother
\title{Билеты по математическому анализу \\ 1 семестр}
\date{Последнее обновление: \today}
\author{Национальный исследовательский \\
Нижегородский Государственный Университет \\
имени Н.И. Лобачевского \vspace{0.5em} \\
Высшая Школа Общей и Прикладной Физики \vspace{0.5em}}
\begin{document}
\maketitle
\begin{description}
	\begin{ticket}[1]
		\item Мощность множеств. Счётные множества и множества мощности континуум.
		\item Число $e$.
		\item Интегралы с переменным верхним пределом и их свойства.
	\end{ticket}
	\begin{ticket}
		\item Определение точной верхней и точной нижней границы множества.
		\item Вторая теорема Больцано-Коши.
		\item Интеграл с переменным верхним пределом и его свойства.
		Определение интеграла Римана и его графический смысл.
	\end{ticket}
	\begin{ticket}
		\item Теорема о существовании арифметического корня.
		\item Первая теорема Больцано-Коши.
		\item Формула Ньютона-Лейбница.
	\end{ticket}
	\begin{ticket}
		\item Теорема о вложенных отрезках (в т.ч. в случае стремления к нулю длинны отрезков).
		\item Эквивалентность определений предела по Коши и по Гейне.
		\item Достаточное условие интегрируемости монотонной функции.
	\end{ticket}
	\begin{ticket}[6]
		\item Теорема о единственности предела последовательности.
		\item Второй замечательный предел.
		\item Интегралы с подстановками Эйлера.
	\end{ticket}
	\begin{ticket}
		\item Необходимое условие сходимости.
		\item Первый замечательный предел.
		\item Критерии интегрирования в терминах колебаний.
	\end{ticket}
	\begin{ticket}
		\item Бином Ньютона.
		\item Первая теорема Вейерштрасса (возможные вопросы: определения ограниченной и непрерывной функции).
		\item Эквивалентные условия интегрируемости в терминах колебаний.
	\end{ticket}
	\begin{ticket}
		\item Теорема Больцано-Вейерштрасса о подпоследовательностях.
		\item Эквивалентность дифференцируемости и существования производной в точке.
		\item Свойства сумм Дарбу.
	\end{ticket}
	\begin{ticket}
		\item Критерий Коши для последовательностей.
		\item Теорема о непрерывности обратной функции.
		\item Интеграл Римана. Необходимое условие интегрируемости.
	\end{ticket}
	\begin{ticket}
		\item Теорема Вейерштрасса для монотонной функции.
		\item Теорема Лагранжа.
		\item Линейные свойства определённого интеграла.
	\end{ticket}
	\begin{ticket}
		\item Арифметические свойства предела.
		\item Теорема Коши.
		\item Аддитивные свойства определённого интеграла.
	\end{ticket}
	\begin{ticket}
		\item Неполнота и алгебраическая незамкнутость поля рациональных чисел.
		\item Теорема Ролля (возможные вопросы: теорема Вейерштрасса и теорема Ферма).
		\item Множества меры ноль и их свойства. Критерий Лебега интегрируемости по Риманы.
	\end{ticket}
	\begin{ticket}
		\item Теорема о двух милиционерах.
		\item Вторая теорема Вейерштрасса.
		\item Условия интегрируемости для непрерывных функций и функций с конечным числом разрывов.
	\end{ticket}
	\begin{ticket}
		\item Сравнение бесконечно малых функций (возможные вопросы: определение предела функции по Коши и по Гейне).
		\item Параметрически заданные функции и их производные (возможные вопросы: определение производной, теорема Коши).
		\item Интегрируемость суммы и произведения интегрируемых функций.
	\end{ticket}
\end{description}
\end{document}