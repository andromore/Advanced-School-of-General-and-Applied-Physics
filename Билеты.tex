\documentclass{article}
\usepackage{mathtext} % использование кириллицы в формулах
\usepackage{cmap} % грамотное копирование кириллицы из pdf
\usepackage[T2A]{fontenc} % внутрення кодировка
\usepackage[utf8]{inputenc} % кодировка документа
\usepackage[russian]{babel} % язык документа
\usepackage{amssymb} % дополнительные символы
\usepackage{amsfonts} % математические шрифты
\usepackage{amsmath} % дополнительная математика
\usepackage{ifthen}
\makeatletter
\newcounter{ticket}[subsection]
\newenvironment{ticket}[1][]{\item[Билет \ifthenelse{\equal{#1}{}}{}{\setcounter{ticket}{#1}}\theticket\refstepcounter{ticket}:]\phantom{}\begin{enumerate}}{\end{enumerate}}
\newcounter{Ticket}[subsection]
\newcommand{\Ticket}[1][]{\item[Билет \ifthenelse{\equal{#1}{}}{}{\setcounter{ticket}{#1}}\theticket\refstepcounter{ticket}:]}
\makeatother
\title{Вопросы и билеты}
\date{Последнее обновление: \today}
\author{Национальный исследовательский \\
Нижегородский Государственный Университет \\
имени Н.И. Лобачевского \vspace{0.5em} \\
Высшая Школа Общей и Прикладной Физики \vspace{0.5em}}
\begin{document}
\maketitle
\section*{Курс 1 Семестр 1}
\subsection*{Математический анализ}
\begin{description}
	\begin{ticket}[1]
		\item Мощность множеств. Счётные множества и множества мощности континуум.
		\item Число $e$.
		\item Интегралы с переменным верхним пределом и их свойства.
	\end{ticket}
	\begin{ticket}
		\item Определение точной верхней и точной нижней границы множества.
		\item Вторая теорема Больцано-Коши.
		\item Интеграл с переменным верхним пределом и его свойства.
		Определение интеграла Римана и его графический смысл.
	\end{ticket}
	\begin{ticket}
		\item Теорема о существовании арифметического корня.
		\item Первая теорема Больцано-Коши.
		\item Формула Ньютона-Лейбница.
	\end{ticket}
	\begin{ticket}
		\item Теорема о вложенных отрезках (в т.ч. в случае стремления к нулю длинны отрезков).
		\item Эквивалентность определений предела по Коши и по Гейне.
		\item Достаточное условие интегрируемости монотонной функции.
	\end{ticket}
	\begin{ticket}[6]
		\item Теорема о единственности предела последовательности.
		\item Второй замечательный предел.
		\item Интегралы с подстановками Эйлера.
	\end{ticket}
	\begin{ticket}
		\item Необходимое условие сходимости.
		\item Первый замечательный предел.
		\item Критерии интегрирования в терминах колебаний.
	\end{ticket}
	\begin{ticket}
		\item Бином Ньютона.
		\item Первая теорема Вейерштрасса (возможные вопросы: определения ограниченной и непрерывной функции).
		\item Эквивалентные условия интегрируемости в терминах колебаний.
	\end{ticket}
	\begin{ticket}
		\item Теорема Больцано-Вейерштрасса о подпоследовательностях.
		\item Эквивалентность дифференцируемости и существования производной в точке.
		\item Свойства сумм Дарбу.
	\end{ticket}
	\begin{ticket}
		\item Критерий Коши для последовательностей.
		\item Теорема о непрерывности обратной функции.
		\item Интеграл Римана. Необходимое условие интегрируемости.
	\end{ticket}
	\begin{ticket}
		\item Теорема Вейерштрасса для монотонной функции.
		\item Теорема Лагранжа.
		\item Линейные свойства определённого интеграла.
	\end{ticket}
	\begin{ticket}
		\item Арифметические свойства предела.
		\item Теорема Коши.
		\item Аддитивные свойства определённого интеграла.
	\end{ticket}
	\begin{ticket}
		\item Неполнота и алгебраическая незамкнутость поля рациональных чисел.
		\item Теорема Ролля (возможные вопросы: теорема Вейерштрасса и теорема Ферма).
		\item Множества меры ноль и их свойства. Критерий Лебега интегрируемости по Риманы.
	\end{ticket}
	\begin{ticket}
		\item Теорема о двух милиционерах.
		\item Вторая теорема Вейерштрасса.
		\item Условия интегрируемости для непрерывных функций и функций с конечным числом разрывов.
	\end{ticket}
	\begin{ticket}
		\item Сравнение бесконечно малых функций (возможные вопросы: определение предела функции по Коши и по Гейне).
		\item Параметрически заданные функции и их производные (возможные вопросы: определение производной, теорема Коши).
		\item Интегрируемость суммы и произведения интегрируемых функций.
	\end{ticket}
\end{description}
\subsection*{Механика}
Вопросы коллоквиума по разделам <<Движение в неинерциальных системах отсчёта>>
и <<Колебательное движение>> (по состоянию первого курса 2022/2023 года):
\begin{description}
	\Ticket[1] Уравнение движения в неинерциальной системе отсчета: простейшие примеры и общий
	случай. Силы инерции.
	\Ticket	Система отсчета, связанная с Землей. Эффекты, связанные с суточным вращением Земли.
	\Ticket Система отсчета, связанная с Землей. Приливы: качественная картина, оценка высоты
	приливной волны.
	\Ticket Гармонические колебания (определение и основные характеристики; закон сохранения
	энергии; фазовый портрет; примеры).
	\Ticket Физический маятник (малые колебания: период, приведенная длина, зависимость периода
	колебаний от положения точки подвеса).
	\Ticket Физический маятник (произвольные колебания: фазовый портрет, качественный анализ
	движения).
	\Ticket Затухающие свободные колебания (определение и основные характеристики; фазовый
	портрет; примеры).
	\Ticket Линейный осциллятор под действием гармонической внешней силы (характеристики
	стационарных колебаний; влияние диссипации).
	\Ticket	Резонанс. Установление амплитуды колебаний. Зависимость от декремента затухания.
	\Ticket Гармонический осциллятор с медленно меняющейся частотой. Адиабатические
	инварианты.
\end{description}
\subsection*{Аналитическая геометрия}
Вопросы к экзамену (Разуваев):
\begin{enumerate}
	\item Операции над векторами и их свойства. Доказать свойство ассоциативности.
	\item Определение пропорциональности векторов. Доказать равносильность коллинеарности и пропорциональности векторов.
	\item Геометрическое определение базиса на прямой $V_1$, плоскости $V_2$ и пространстве $V_3$. Теоремы о разложении любого вектора по базису (случаи $V_1$, $V_2$, $V_3$).
	\item Определение координат вектора (рассмотреть случай пространства $V_3$). Операции над векторами в координатной форме. Доказать критерий коллинеарности векторов в координатной форме.
	\item Определение линейно зависимых и линейно независимых систем векторов. Алгебраическое определение базиса в векторном пространстве.
	\item Определение скалярного произведения векторов и его свойства. Выражение скалярного произведения векторов в координатной форме в общем случае и случае ортонормированного базиса.
	\item Понятие ориентации тройки векторов и циклической перестановки. Определение и свойства векторного произведения векторов.
	\item Определение смешанного произведения векторов. Доказать численное равенство модуля смешанного произведения векторов и значения объёма параллелепипеда, построенного на этих векторах.
	\item Критерий коллинеарности трёх векторов.
	\item Вывод формулы векторного произведения в координатной форме в общем случае и в случае ортонормированного базиса.
	\item Вывод формулы смешанного произведения в координатной форме.
	\item Матрица перехода в $V_2$. Невырожденность матрицы перехода. Закон изменения координат вектора при изменении базиса.
	\item Определение аффинного пространства и аффинной системы координат. Определение координат точки и координат вектора. Изменение координат точки при изменении системы координат.
	\item Уравнение линии на плоскости. Определение алгебраической линии $n$-ого порядка. Доказать инвариантность понятия алгебраической линии и её порядка относительно изменения декартовой системы координат для случая алгебраической линии первого порядка.
	\item Вывод векторного, параметрического и канонического уравнений прямой на плоскости.
	\item Вывод уравнения прямой по двум точкам и общего уравнения прямой.
	\item Доказать теорему о взаимном расположении прямых на плоскости.
	\item Определение пучка прямых на плоскости. Вывод уравнения пучка прямых на плоскости.
	\item Вывод формулы расстояния от точки до прямой на плоскости.
	\item Определение алгебраической поверхности и её порядка. Векторно-параметрическое и параметрическое уравнения плоскости.
	\item Теорема о задании линейным уравнением в произвольной системе координат и обратный случай. Общее уравнение плоскости.
	\item Определение вектора нормали к плоскости.Координаты вектора нормали к плоскости, заданной общим уравнением в прямоугольной системе координат.
	\item Вывод формулы расстояния от точки до плоскости.
	\item Уравнение линии в пространстве. Вывод уравнений прямых в пространстве.
	\item Вывод формулы расстояния между скрещивающимися прямыми.
	\item Эллипс и теорема о модулях радиус-векторов эллипса.
	\item Критерий принадлежности точки эллипсу ($r_1 + r_2 = 2a$).
	\item Гипербола и теорема о модуле радиус-векторов гиперболы.
	\item Критерий принадлежности точки гиперболе ($|r_1 - r_2| = 2a$).
	\item Парабола.
\end{enumerate}
\subsection*{История}
\begin{enumerate}
	\item Славяне. Восточные славяне.
	\item Киевская Русь и её соседи. Норманская теория. Обзор деятельности первых русских князей.
	\item Социально-экономическое и политическое развитие Киевской Руси в IX -- X веках.
	\item Княжение Владимира. Крещение Руси. Значение принятия христианства.
	\item Время Ярослава Мудрого. Киевская Русь в XI веке.
	\item Княжение Владимира Мономаха (1113 -- 1125). Киевская Русь в XII веке. Феодальная раздробленность Руси. Владимиро-Суздальская Русь.
	\item Невская битва (1240 год). Александр Невский. Ледовое побоище (1242 год).
	\item Русь и монголо-татары.
	\item Великое княжество Московское в XIV -- XV веках. Иван Калита.
	\item Борьба Московского княжества с Золотой Ордой. Дмитрий Иванович Донской. Куликовская битва.
	\item Объединение русских земель вокруг Московского княжества.
	\item Объединение русских земель в единое государство. Иван III.
	\item Россия в первой половине XVI века.
	\item Реформы Ивана IV Грозного. Опричнина.
	\item Внешняя политика Ивана IV.
	\item Русское государство в конце XVI века. Внутренняя и внешняя политика царя Фёдора (1584 -- 1598).
	\item <<Смута>> в России в начале XVII века. Самозванничество.
	\item Борьба русского народа с польской интервенцией. Нижегородское ополчение под предводительством Кузьмы (в последствии Козьмы) Минина и Дмитрия Пожарского.
	\item Правление царя Михаила Фёдоровича Романова (1613 -- 1645).
	\item Правление царя Алексея Михайловича (1645 -- 1676). Соборное уложение 1649 года.
	\item Церковь и раскол в XVII веке. Протопоп Аввакум. Патриарх Никон. Соловецкое восстание.
	\item Внешняя политика при Алексее Михайловиче. Воссоединение Украины с Россией.
	\item Социальные протесты XVII века. Крестьянская война под предводительством Степана Тимофеевича Разина.
	\item Царствование Петра I. Социально-экономическое развитие России на рубеже XVII -- XVIII веков.
	\item Внешняя политика Петра I.
	\item Реформы Петра I Великого и их историческое значение.
	\item Дворцовые перевороты.
	\item Внутренняя политика Екатерины II. <<Уложенная комиссия>> (1767 -- 1768 годов).
	\item Просвещённый абсолютизм.
	\item Социально-политическое развитие России во второй половине XVIII века. <<Золотой век дворянства>>.
	\item Внешняя политика России в период правления Екатерины II (1762 -- 1796 годы).
	\item Правление Павла I (1796 -- 1801 годы).
\end{enumerate}
\section*{Курс 1 Семестр 2}
\subsection*{Математический анализ}
\begin{description}
	\begin{ticket}[1]
		\item Признаки сравнения сходимости несобственных интегралов.
		\item Формула Тейлора для функции нескольких переменных.
		\item Переход к пределу под знаком интеграла для
		      семейства фукнций.
	\end{ticket}
	\begin{ticket}[4]
		\item Компактные множества в метрических пространствах.
		      Необходимое условие компактности.
		\item Признак Раабе.
		\item Ряды Фурье. Коэффициенты тригонометрического ряда Фурье.
	\end{ticket}
	\begin{ticket}
		\item Связность и линейная связность. Образ связного множества
		      при непрерывном отображении.
		\item Признак Дирихле для числового ряда.
		\item Переход к пределу под знаком интеграла для
		      семейства функций.
	\end{ticket}
	\begin{ticket}
		\item Критерий компактности в $\mathbb{R}^n$.
		\item Признак Даламбера сходимости положительного ряда.
		\item Дифференцируемость интеграла, зависящего от параметра.
	\end{ticket}
	\begin{ticket}
		\item Образ компакта при непрерывном отображении.
		\item Формула Коши-Адамара.
		\item Бета-функция и её свойства.
	\end{ticket}
	\begin{ticket}
		\item Достаточное условие дифференцируемости в терминах
		      частных производных.
		\item Критерий сходимости положительного ряда.
		\item Равенство Парсеваля и неравенство Бесселя.
	\end{ticket}
	\begin{ticket}
		\item Теорема о дифференцируемости композиции дифференцируемых
		      функций.
		\item Равномерная сходимость и интегрирование.
		\item Разложение в ряд Тейлора: $ln(1 + x)$, $arc tg(x)$.
	\end{ticket}
	\begin{ticket}
		\item Связность и линейная связность. Образ связного множества
		      при непрерывном отображении.
		\item Достаточное условие абсолютного экстремума.
		\item Интегральный признак сходимости.
	\end{ticket}
	\begin{ticket}
		\item Инвариантность первого дифференциала.
		\item Совпадение смешанных частных производных.
		\item Равномерная сходимость несобственных интегралов,
		      зависящих от параметра. Аналог теоремы Вейерштрасса.
	\end{ticket}
	\begin{ticket}
		\item Равномерная непрерывность. Обобщение теоремы Кантора.
		\item Теорема Абеля о поведении степенного ряда на границе
		      интервала сходимости.
		\item Признак Дини.
	\end{ticket}
\end{description}
\subsection*{Термодинамика и молекулярная кинетическая теория}
Известные билеты:
\begin{description}
	\begin{ticket}[3]
		\item Идеальный газ. Агрегатные состояния вещества.
		      Закон Дальтона.
		\item Молекулярно-кинетическая формулировка температуры
		      и теплового равновесия.
	\end{ticket}
	\begin{ticket}
		\item Теплоёмкость.
		\item Распределение Ферми-Дирака
	\end{ticket}
	\begin{ticket}[9]
		\item Статистика Бозе-Эйнштейна.
		\item Вязкость газа; внутреннее трение; коэффициент вязкости;
		      сила вязкого трения; оценка коэффициента вязкости.
	\end{ticket}
	\begin{ticket}
		\item Термодинамические процессы; квазистатические процессы
		      (обратимые); адиабатическое расширение и сжатие
		      (общий вид, примеры).
		\item Спектр излучения абсолютно чёрного тела; распределение
		      по степеням свободы для электромагнитного излучения;
		      формула Планка; закон Стефана-Больцмана; закон Вина.
	\end{ticket}
	\begin{ticket}[13]
		\item Второе начало термодинамики.
		\item Фазы вещества: классификация и условие равновесия.
	\end{ticket}
	\begin{ticket}
		\item Обратимые круговые процессы. Идеальный газ. Цикл Карно.
		\item Вывод уравнения Клайперона-Клаузиуса.
	\end{ticket}
	\begin{ticket}[17]
		\item Энтропия обратимых и необратимых процессов.
		      Неравенство Клаузиуса.
		\item Поверхностное натяжение и его термодинамический смысл
		      (связь коэффициента полезного действия с температурой).
	\end{ticket}
	\begin{ticket}
		\item Коэффициент полезного действия в необратимом круговом процессе
		      (физические причины необратимости; примеры).
		\item Смачивание и несмачивание; условие равновесия границы;
		      краевой угол (примеры).
	\end{ticket}
	\begin{ticket}
		\item Реальные газы (определение); газ Ван-дер-Ваальса
		      (уравнение; изотермы в координатах $P(V)$; внутренняя энергия;
		      теплоёмкость).
		\item Соотношение между давлением и кривизной поверхности;
		      формулы Лапласа; закон Жюрена.
	\end{ticket}
	\begin{ticket}
		\item Термодинамическое и статистическое определение
		      макропараметров физической системы. Эргодические системы.
		\item Характер движения отдельной частицы в газе.
		      Длина свободного пробега. Среднеквадратичное отклонение
		      частицы от начального положения.
	\end{ticket}
\end{description}
Вопросы к экзамену (по состоянию 1 курс 2022/2023 года):
\begin{enumerate}
	\item Способы описания систем многих частиц. Термодинамическое описание. Внутренние и
	      внешние параметры. Уравнение состояния.
	\item Понятие температуры. Абсолютная шкала температур. Абсолютный нуль. Давление.
	\item Агрегатные состояния вещества. Идеальный газ (уравнение состояния идеального газа:
	      вывод, пределы применимости). Закон Дальтона.
	\item Идеальный газ во внешнем силовом поле. Барометрическая формула: вывод, частные
	      случаи, условия применимости.
	\item Работа (элементарная работа; работа при конечном процессе; геометрическая
	      интерпретация; положительная и отрицательная работа; круговые процессы). Внешние
	      параметры и обобщенные силы.
	\item Адиабатическая оболочка. Внутренняя энергия. Количество тепла. Первый принцип
	      термодинамики.
	\item Звук как волна деформации. Скорость звука в газе. Изотермический и адиабатический
	      звук.
	\item Теплоемкость (теплоемкость физической системы; удельная и молярная теплоемкость;
	      зависимость теплоемкости от процесса; примеры).
	\item Термодинамическое равновесие. Функции состояния термодинамической системы.
	      Энтальпия (определение, физический смысл, примеры).
	\item Термодинамические процессы. Квазистатические (обратимые) процессы. Адиабатическое
	      расширение и сжатие (определение; общие соотношения для обратимого процесса;
	      примеры).
	\item Процесс Джоуля-Томсона. Закон Джоуля. Необратимые термодинамические процессы.
	\item Второй принцип термодинамики (формулировка; физический смысл; примеры: тепловые
	      двигатели с одним и двумя тепловыми резервуарами).
	\item Обратимый круговой процесс. Идеальный цикл Карно (прямой и обращенный цикл
	      Карно; второй принцип для обратимых круговых процессов).
	\item Теорема Карно. Абсолютная термодинамическая шкала температур. Следствие теоремы
	      Карно: связь калорического и термического уравнений состояния.
	\item Энтропия (определение, физический смысл, примеры). Второй принцип для обратимых
	      процессов. Равенство Клаузиуса.
	\item Второй принцип для необратимых процессов. Неравенство Клаузиуса: вывод;
	      физический смысл; следствия для адиабатических процессов.
	\item К.п.д. необратимых круговых процессов (физические причины необратимости, примеры
	      необратимых процессов).
	\item Реальные газы. Уравнение Ван-дер-Ваальса (вывод; изотермы В.д.В.; внутренняя энергия
	      и теплоемкость газа В.д.В.)
	\item Термодинамическое и статистическое определения макропараметров физической
	      системы. Эргодические системы.Высшая Школа Общей и Прикладной Физики
	\item Основные понятия теории вероятностей (событие, вероятность, статистический ансамбль;
	      непрерывные случайные величины, плотность вероятности; сложение вероятностей;
	      условная вероятность; независимые события; средние знaчения случaйных величин).
	\item Распределение молекул газа по скоростям (пространство скоростей, изображающая
	      точка, статистическая постановка задачи, функция распределения); закон распределения
	      скоростей Мaксвеллa-Больцмaнa: математическая запись и физический смысл. Условия
	      применимости мaксвелловского распределения.
	\item Закон Мaксвеллa-Больцмaнa: вывод; одномерное и трехмерное распределения;
	      распределение по модулю скорости; характерные скорости молекул.
	\item Давление газа с точки зрения молекулярно-кинетической теории. Давление фотонного
	      газа.
	\item Молекулярно-кинетический смысл температуры и теплового равновесия. Принцип
	      равномерного распределения энергии по степеням свободы: частные примеры и условия
	      применимости.
	\item Теплоемкость газов (классическая теория и пределы ее применимости). Степени свободы
	      электромагнитного излучения в полости; ультрафиолетовая катастрофа.
	\item Вероятностный смысл энтропии. Молекулярно-кинетическaя формулировкa второго
	      принципa термодинамики. Статистический вес.
	\item Микро- и мaкросостояния системы. Ансамбль различимых частиц. Рaспределение
	      Больцмaнa и условия его применимости.
	\item Ансамбль неразличимых частиц. Статистика Ферми-Дирaкa (фермионы; стaтистический
	      вес; распределение Ферми-Дирaкa; вырождение Ферми-гaзa; оценка теплоёмкости).
	\item Ансамбль неразличимых частиц. Стaтистика Бозе-Эйнштейнa (бозоны; стaтистический
	      вес; распределение Бозе-Эйнштейнa)
	\item Излучение абсолютно чёрного тела. Степени свободы электромагнитного излучения в
	      полости и их заполнение. Формула Планка. Зaкон Стефaнa-Больцмaнa, закон Вина.
	\item Теплоемкость кристаллических твердых тел. Закон Дюлонга-Пти. Ансамбль фононов;
	      теория теплоёмкости Дебая. Поведение теплоёмкости вблизи абсолютного нуля.
	\item Испaрение и конденсaция веществ, подчиняющихся урaвнению Вaн-дер-Вaaльсa.
	      Прaвило Мaксвеллa. Метaстaбильные состояния. Критическая точка.
	\item Общие условия термодинамического равновесия и устойчивости при различных
	      условиях. Энтропия, свободная энергия, термодинамический потенциал Гиббса.
	\item Фaзы, их классификация, фaзовые преврaщения, условия равновесия фаз. Фазовая
	      диаграмма.
	\item Испaрение и конденсaция в широком смысле. Урaвнение Клaйперонa-Клaузиусa (вывод,
	      физический смысл, примеры).
	\item Зaвисимость дaвления нaсыщенного пaрa от темперaтуры. Абсолютная и относительная
	      влажность воздуха.
	\item Фaзовые преврaщения второго родa (определения, физический смысл, примеры).
	      Соотношения Эренфестa.
	\item Природa поверхностного натяжения. Коэффициент поверхностного нaтяжения:
	      определение и физический смысл. Термодинaмикa поверхностного нaтяжения.
	\item Смaчивaемые и несмaчивaемые поверхности. Равновесие границы и крaевой угол.
	      Примеры.
	\item Соотношение между дaвлением и кривизной поверхности. Формулa Лaплaсa.
	      Кaпиллярные явления, зaкон Жюренa. Примеры.
	\item Характер движения отдельной частицы в газе. Длина свободного пробега.
	      Среднеквадратичное отклонение частицы от начального положения.
	\item  идеального газа. Поток тепла с точки зрения молекулярно-
	      кинетической теории. Оценка коэффициента теплопроводности. Уравнение
	      теплопроводности.
	\item Вязкость идеального газа. Внутреннее трение с точки зрения молекулярно-кинетической
	      теории. Оценка коэффициента вязкости. Сила вязкого трения; примеры проявления.
\end{enumerate}
\subsection*{Линейная алгебра}
\begin{description}
	\item[Билет 2:]
		Подпространства. Разложение подпространства в прямую сумму
		подпространств. Примеры разложения на подпространства.
	\item[Билет 5:]
		Обратная матрица и её вид.
	\item[Билет 17:]
		Унитарные операторы и их матрицы.
		(Всё, что знаете)
	\item[Билет 18:]
		Ортогональные операторы и их матрицы.
		(Всё, что знаете)
	\item[Билет 20:]
		Квадратичные формы: определение и диагонализация
		методом ортогональных преобразований. Закон инерции.
	\item[Билет 21:]
		Метод Якоби диагонализации квадратичных форм.
	\item[Билет 22:]
		Положительно определённые квадратичные формы
		и операторы в терминах квадратичных форм.
		(Определение; Необходимые и достаточные условия)
	\item[Билет 24:]
		Одновременная диагонализация квадратичных форм
\end{description}
\section*{Курс 2 Семестр 1}
\subsection*{Теория функций комплексного переменного}
\begin{description}
	\begin{ticket}[6]
		\item ?
		\item ?
		\item ?
		\item ?
		\item ?
		\item Найти оригинал по изображению: \begin{equation*}
			F(p) = \dfrac{1}{p^2 + 4p + 5}
		\end{equation*}
	\end{ticket}
	\begin{ticket}
		\item Множества на комплексной плоскости. Открытые, замкнутые, ограниченные и неограниченные множества. Граница множества. Предельные точки множества. Линейно-связное множество. Область. Односвязная и n-связная область. Компактное множество на комплексной плоскости. Расстояние между множества- ми. Доказательство факта, что расстояние между двумя ограниченными непересекающимися компактами больше нуля.
		\item Принцип аргумента. Теорема Руше. Основная теорема высшей алгебры (для нескольких корней).
		\item ?
		\item ?
		\item ?
		\item ?
	\end{ticket}
	\begin{ticket}
		\item Функции комплексного переменного, предел, непрерывность и т.д.
		\item Достаточные условия существования оригинала.
		\item ?
		\item ?
		\item Найти особые точки и определить их тип: \begin{equation*}
			f(z) = \dfrac{sh(\pi z)}{z^4 + 1}\cdot e^{\dfrac{1}{z}}
		\end{equation*}
		\item Решить задачу Коши: $x' + x = e^{-t}$, $x(0) = 1$.
	\end{ticket}
	\begin{ticket}[10]
		\item Дифференцируемость основных функций ($e^z$, $sin(z)$, $cos(z)$),
		рациональных и других элементарных. Теорема о существовании дифференцируемой функции
		комплексного переменного по данной действительной (мнимой) части (без доказательства).
		\item Формула Меллина.
		\item Решить уравнение $cos(z) + sin(z) = i$.
		\item Отобразить область $\{|z-1| > 1; |z - 2| < 2; Im\ z > 0\}$ на полуплоскость $Im\ \omega > 0$.
		\item Найти особые точки и определить их тип: \begin{equation*}
			f(z) = \dfrac{2z - sin(2z)}{z^2(z^2 + 1)}
		\end{equation*}
		\item Найти изображение функции $f(t) = t sh(4t) \eta(t)$.
	\end{ticket}
	\begin{ticket}[12]
		\item Интегрирование функций комплексного переменного. Формула для вычисления с помощью параметризации.
		Неравенства. Определение несобственного интеграла по лучу и по прямой. Теорема о возможности аппроксимировать
		интеграл по кривой интегралом по ломанной. Теорема об аппроксимации интеграла по границе области (без доказательства).
		\item Принцип соответствия границ (без доказательства). Критерий однолистности функции в области.
		\item Исследовать на дифференцируемость $f(z) = z \cdot \overline{z}$.
		\item Найти образ полуплоскости $Re\ z > 1$ при отображении: \begin{equation*}
			\omega(z) = \dfrac{z - 3 + i}{z + 1 + i}
		\end{equation*}
		\item Найти интеграл: \begin{equation*}
			\int\limits_0^{2\pi}\dfrac{d\varphi}{cos(\varphi) + 2}
		\end{equation*}
		\item Найти изображение функции: \begin{equation*}
			f(t) = \left\{\begin{matrix}
				1,  & 2n - 2 \leqslant t < 2n - 1 \\
				-1, & 2n - 1 \leqslant t < 2n
			\end{matrix}\right., n \in \mathbb{N}
		\end{equation*}
	\end{ticket}
	\begin{ticket}[15]
		\item Интеграл и первообразная. Теорема о первообразной.
		\item Свойства преобразования Лапласа.
		\item ?
		\item ?
		\item ?
		\item ?
	\end{ticket}
	\begin{ticket}[22]
		\item Целые функции. Теорема Лиувилля. Основная теорема высшей алгебры (для одного корня). Теорема Мореры.
		\item Лемма Жордана.
		\item ?
		\item ?
		\item ?
		\item ?
	\end{ticket}
\end{description}
Полный список теоретических вопросов:
\begin{enumerate}
	\item Комплексные числа. Действительная и мнимая части комплексного числа. Модуль и аргумент комплексного числа.
	      Арифметические действия с комплексными числами и их геометрический смысл.
	\item Корни из комплексных чисел. Алгебраическая, тригонометрическая и показательная формы записи комплексных чисел.
	      Возведение в комплексную степень.
	\item Последовательности комплексных чисел. Предел. Ограниченность. Покоординатная сходимость.
	      Бесконечно малые и бесконечно большие последовательности. Критерий Коши.
	\item Ряды комплексных чисел. Абсолютная и условная сходимость. Критерий Коши. Необходимое условие
	      сходимости. Сходимость абсолютно сходящегося ряда.
	\item Расширенная комплексная плоскость. Сфера Римана. Стереографическая проекция. Доказательство факта,
	      что прямым и окружностям комплексной плоскости при стереографической проекции соответствуют
	      окружности сферы Римана.
	\item Кривые на комплексной плоскости. Простая кривая. Замкнутая кривая. Кривая, проходящая через бесконечность.
	      Спрямляемая и кусочно-гладкая кривые.
	\item Множества на комплексной плоскости. Открытые, замкнутые, ограниченные и неограниченные множества.
	      Граница множества. Предельные точки множества. Линейно-связное множество. Область. Односвязная
	      и n-связная область. Компактное множество на комплексной плоскости. Расстояние между множествами.
	      Доказательство факта, что расстояние между двумя ограниченными непересекающимися компактами
	      больше нуля.
	\item Функции комплексной переменной. Предел. Непрерывность и равномерная непрерывность. Непрерывность
	      действительной, мнимой части и модуля непрерывной функции комплексного переменного. Теорема Кантора.
	\item Дифференцируемость функций комплексной переменной. Условия Коши-Римана. Примеры. Гармонические и сопряженные функции.
	\item Дифференцируемость основных функций ($e^z$ , $sin\ z$, $cos\ z$), рациональных и других элементарных функций.
	      Теорема о существовании дифференцируемой функции комплексного переменного по данной действительной (мнимой) части (без доказательства).
	\item Геометрический смысл производной. Теорема об обратной функции.
	\item Интегрирование функций комплексного переменного. Формула для вычисления с помощью параметризации.
	      Неравенства. Теорема о возможности аппроксимировать интеграл по кривой интегралом по ломаной.
	      Теорема об аппроксимации интеграла по границе области (без доказательства).
	\item Интегральная теорема Коши. Следствия.
	\item Интеграл и первообразная. Теорема о первообразной.
	\item Интегральная формула Коши. Теорема о среднем для функции комплексного переменного и гармонической
	      функции.
	\item Принцип максимума гармонической функции и принцип максимума модуля дифференцируемой функции.
	\item Функциональные ряды. Равномерная сходимость. Критерий Коши. Теорема о том,
	      что почленное домножение равномерно сходящегося функционального ряда на ограниченную функцию не изменяет равномерной
	      сходимости. Признак Вейерштрасса равномерной сходимости.
	      Непрерывность суммы равномерно сходящегося ряда непрерывных функций. Почленное интегрирование равномерно сходящегося ряда.
	\item Степенные ряды. Теорема Абеля. Радиус сходимости степенного ряда.
	\item Почленное дифференцирование степенного ряда. Единственность разложения функции в степенной ряд.
	      Ряд Тейлора. Формулы для коэффициентов ряда Тейлора. Основные разложения ($sin\ z$, $cos\ z$, $e^z$ , $(1+z)^k, k \in
		      \mathbb{Z}$).
	\item Основной критерий регулярности функции в области. Оценка радиуса сходимости степенного ряда регулярной функции.
	      Интегральное представление производных регулярной функции. Неравенство Коши.
	      Бесконечная дифференцируемость дифференцируемой и гармонической функций.
	\item Целые функции. Теорема Лиувилля. Основная теорема высшей алгебры (про один корень). Теорема Мореры.
	\item Теорема Вейерштрасса о почленном дифференцировании функционального ряда.
	\item Теорема единственности. Понятие аналитического продолжения. Принцип аналитического продолжения.
	      Аналитические продолжения элементарных функций и соотношений.
	\item Правильные и особые точки. Теорема о том, что на границе сходимости степенного ряда лежит особая
	      точка функции.
	\item Аналитическое продолжение вдоль кривой. Функции, аналитические на кривой и в области. Аналитическая функция $Ln\ z$. Точки ветвления.
	\item Аналитическое продолжение вдоль кривой. Функции, аналитические на кривой и в области.
	      Аналитическая функция $z^\alpha$ ($\alpha$ – действительное). Точки ветвления.
	\item Ряд Лорана. Регулярность суммы ряда Лорана. Теорема Лорана. Единственность разложения функции в
	      ряд Лорана. Неравенство Коши для коэффициентов ряда Лорана.
	\item Изолированные особые точки однозначного характера. Теорема о главной части ряда Лорана в окрестности
	      устранимой точки. Следствие. Классификация особых точек.
	\item Нули регулярных функций. Вид регулярной функции в окрестности нуля. Полюса. Порядок полюса. Вид
	      ряда Лорана в окрестности полюса.
	\item Существенно особые точки. Вид ряда Лорана в окрестности существенно особой точки. Теорема Сохоцкого-Вейерштрасса. Теорема Пикара (без доказательства).
	\item Мероморфные функции. Теорема о разложении мероморфной функции с конечным числом полюсов.
	\item Вычеты. Основная теорема о вычетах. Следствие. Теорема о вычетах для области, содержащей бесконечность.
	\item Лемма Жордана.
	\item Принцип аргумента. Теорема Руше. Основная теорема высшей алгебры (для нескольких корней).
	\item Определение конформного отображения (в точке, в области). Свойства конформных отображений. Теорема
	      о том, что при конформном отображении образом области является область.
	\item Однолистность функции в точке. Локальный критерий однолистности. Необходимые и достаточные условия однолистности функции в полюсе и в бесконечности.
	\item Принцип соответствия границ (без доказательства). Критерий однолистности функции в области.
	\item Линейная функция. Дробно-линейная функция. Теорема о том, что дробно-линейная функция --- единственная, конформно отображающая расширенную комплексную плоскость на себя.
	\item Круговое свойство дробно-линейных отображений. Единственность дробно-линейного отображения, переводящего три заданные точки в три заданные точки.
	\item Симметричные точки. Сохранение симметрии при дробно-линейных отображениях.
	\item Отображение верхней полуплоскости и единичного круга на единичный круг. Задача с нормировкой. Теорема Римана (без доказательства).
	\item Основные свойства степенной функции и функции Жуковского. Круговые луночки. Показательная функция.
	\item Задача Дирихле для уравнения Лапласа. Существование и единственность решения.
	\item Решение задачи Дирихле для уравнения Лапласа для круга и полуплоскости.
	\item Интегралы, зависящие от параметра. Теоремы о непрерывности, регулярности функции, заданной интегралом, зависящим от параметра. Интегрирование под знаком интеграла.
	\item Несобственные интегралы (по лучу или прямой). Несобственные интегралы, зависящие зависящие от параметра. Равномерная сходимость. Непрерывность функции, заданной таким интегралом.
	\item Преобразование Лапласа. Регулярность изображения. Стремление к нулю изображения при $Re\ p \rightarrow +\infty$.
	\item Свойства преобразования Лапласа (линейность, теорема подобия, изображение производной, интегрирование оригинала, интегрирование и дифференцирование изображения, теорема запаздывания, теорема
	      смещения, теорема свертки).
	\item Формула Меллина.
	\item Достаточные условия существования оригинала. Нахождение оригинала с помощью вычетов.
\end{enumerate}
\subsection*{Теоретическая механика}
\begin{description}
	\Ticket[1] Основные положения механики Ньютона, принцип относительности Галилея, принцип
	детерминизма Ньютона. I, II и III закон Ньютона. Инертная масса, импульс, момент импульса
	материальной точки.
	\Ticket Законы сохранения импульса и момента импульса системы материальных точек в механике
	Ньютона. Центр масс.
	\Ticket Кинетическая и потенциальная энергия материальной точки, уравнение для изменения
	механической энергии материальной точки во времени.
	\Ticket Закон сохранения энергии для системы материальных точек в механике Ньютона.
	\Ticket Теоpема о виpиале в механике Ньютона.
	\Ticket Связи в механике Лагранжа. Основная задача механики для систем с идеальными голономными
	связями.
	\Ticket Уравнения Лагранжа первого рода.
	\Ticket Пpинцип Д'Аламбеpа-Лагpанжа. Независимые обобщенные координаты. Вывод уравнений
	Лагpанжа второго рода для независимых обобщенных координат.
	\Ticket Обобщенные импульсы и обобщенные силы в механике Лагранжа.
	\Ticket Потенциальные и обобщенно-потенциальные силы в механике Лагранжа. Диссипативная функция
	Рэлея.
	\Ticket Сила Лоренца как обобщенно-потенциальная сила. Функция Лагpанжа для системы заряженных
	частиц во внешнем электpомагнитном поле.
	\Ticket Законы сохранения обобщенного импульса и обобщенной энергии в механике Лагранжа.
	\Ticket Линейные колебания в одномерных лагранжевых системах.
	\Ticket Линейные колебания в консервативных лагранжевых системах с несколькими степенями свободы.
	\Ticket Функционал действия. Вариационный принцип Гамильтона и принцип наименьшего действия для
	обобщенно-потенциальных систем. Вывод уравнений Лагранжа из вариационного принципа.
	\Ticket Свойства уравнений Лагранжа, следующие из вариационного принципа. Инвариантные
	преобразования функции Лагранжа, замена координат.
	\Ticket Симметрия механических систем и законы сохранения. Теоpема Hетеp.
	\Ticket Классификация состояний равновесия на двумерной фазовой плоскости. «Консервативные» и
	«диссипативные» состояния равновесия. Состояния равновесия типа «центp», «седло», «узел» и
	«фокус».
	\Ticket Качественный анализ систем с одной степенью свободы на фазовой плоскости (консервативные и
	диссипативные системы).
	\Ticket Интегpиpование уравнений движения одномерных консервативных систем.
	\Ticket Периодическое движение в одномерных натуральных консервативных системах. Вычисление
	периода колебаний для заданного потенциала.
	\Ticket Восстановление потенциала по заданной зависимости периода колебаний от полной энергии.
	\Ticket Определение возмущения периода колебаний при возмущении потенциальной энергии.
	\Ticket Колебания математического маятника.
\end{description}
Вопросы теоретического минимума:
\begin{description}
	\Ticket[1] Необходимые и достаточные условия потенциальности силы (без вывода).
	\Ticket Законы сохранения импульса и момента импульса для системы материальных точек.
	\Ticket Закон сохранения энергии для системы материальных точек.
	\Ticket Теоpема о виpиале (простейшая фоpмулиpовка для одной материальной точки).
	\Ticket Типы состояний равновесия систем с одной степенью свободы.
	\Ticket Интеграл энергии для движения материальной точки в одномерном потенциальном поле.
	\Ticket Независимые обобщенные координаты (опpеделение и нетривиальный пpимеp).
	\Ticket Уравнения Лагpанжа в независимых обобщенных координатах для систем с идеальными
	голономными связями (формулировки с учетом и без учета диссипативных сил).
	\Ticket Обобщенные силы (опpеделение и пpимеp).
	\Ticket Обобщенно-потенциальные силы (опpеделение и пpимеp).
	\Ticket Обобщенная энергия в механике Лагpанжа (опpеделение и условие сохранения).
	\Ticket Функция Лагpанжа натуральной механической системы.
	\Ticket Функция Лагранжа свободной материальной точки в декартовой, цилиндрической и сферической
	системах координат.
	\Ticket Функция Лагpанжа заpяженной частицы во внешнем электpомагнитном поле общего вида.
	\Ticket Функция Лагpанжа заpяженной частицы в постоянном магнитном поле
	\Ticket Циклические кооpдинаты в уpавнениях Лагpанжа, их связь с интегралами движения.
	\Ticket Ваpиационный пpинцип Гамильтона (фоpмулиpовка в механике Лагранжа).
	\Ticket Функция Лагpанжа и фазовая плоскость гармонического осциллятора.
	\Ticket Фазовая плоскость и зависимость периода от энергии (качественно) математического маятника.
\end{description}
\subsection*{Дифференциальные уравнения}
До коллоквиума (по состоянию 2 курса на 2023 год):
\begin{description}
	\Ticket[1] Понятие обыкновенного ДУ, порядок ДУ, решение ДУ. Понятие ДУ 1-го порядка, решение ДУ,
	задача Коши, геометрический смысл ДУ и его решения. Изоклины, метод изоклин.
	\Ticket Понятия общего, частного и особого решений для ДУ 1-го порядка, разрешенного относительно
	производной. Формулировка теоремы о существовании и единственности решения задачи Коши для
	ДУ 1-го порядка. Условие Липшица. Пример, демонстрирующий факт локальности теоремы. Пример,
	показывающий, что теорема дает лишь достаточные условия существования и единственности
	решения задачи Коши.
	\Ticket ДУ 1-го порядка с разделяющимися переменными: понятие, метод интегрирования. Особые
	решения, критерий для уравнения $y' = f(y)$, привести примеры. ДУ, сводящиеся к уравнению с
	разделяющимися переменными.
	\Ticket Однородные ДУ 1-го порядка и сводящиеся к ним: понятия и методы интегрирования.
	\Ticket Линейные ДУ 1-го порядка. Свойства линейного однородного уравнения (есть нулевое; всякое
	ненулевое целиком лежит выше или ниже оси х; сумма двух решений – решение; частное решение опр-ся с
	точностью до константы; структура общего решения; если известно частное, то общее – это с*на это
	частное). Теорема о структуре общего решения линейного неоднородного уравнения. Методы
	интегрирования линейного неоднородного уравнения (метод Лагранжа вариации произвольной
	постоянной, метод Бернулли). Уравнения Бернулли и Риккати.
	\Ticket Симметричная форма ДУ первого порядка. ДУ в полных дифференциалах. Необходимое и
	достаточное условие уравнения в полных дифференциалах. Способы восстановления функции по ее
	полному дифференциалу.
	\Ticket Некоторые способы решения уравнения, не являющегося уравнением в полных дифференциалах.
	Интегрирующий множитель. Способы нахождения интегрирующего множителя. Свойства
	интегрирующего множителя. Интегрирующий множитель и особые решения.
	\Ticket ДУ 1-го порядка, не разрешенные относительно производной. Поле направлений (в чем отличие от
	ур-я, разрешенного отн-но производной). Задача Коши. Теорема о существовании и единственности
	решения задачи Коши. Особые решения. Методы нахождения особых решений (дискриминантная
	кривая, огибающая).
	\Ticket Методы интегрирования уравнений, не разрешенных относительно производной.
	Метод разрешения отн-но производной; Метод введения параметра: \begin{enumerate}
		\item ур-е не содержит искомой ф-ции (разрешимо отн-но х, не разрешимо ни отн-но х, ни отн-но производной)
		\item ур-е не содержит х (ур-е разрешимо отн-но у, не разрешимо ни отн-но у, ни отн-но производной)
		\item ур-е общего вида (ур-е разрешимо отн-но х, отн-но у, не разрешимо ни отн-но х, ни отн-но у, ни отн-но производной)
	\end{enumerate}
	\Ticket Уравнения Лагранжа и Клеро.
	\Ticket ДУ высших порядков. Задача Коши. Формулировка теоремы существования и единственности
	решения задачи Коши. Классы ДУ высших порядков, допускающие понижение порядка. Методы
	интегрирования.
	Уравнение, не содержащее искомой ф-ции и ее производных до пор-ка (k-1) включительно, порядок ур-я
	можно понизить на k единиц. Ур-е, не содержащее независимой переменной, порядок ур-я можно понизить
	на единицу. Ур-е в точных производных допускает понижение пор-ка на единицу. Однородное (отн-но у и
	производных) уравнение – понижение порядка на единицу. Обобщенное однородное (отн-но х, у и ее
	производных) – понижение порядка на единицу.
	\Ticket Линейные дифференциальные уравнения произвольного порядка с переменными
	коэффициентами, однородные и неоднородные уравнения. Формулировка теоремы о существовании
	и единственности решений задачи Коши для линейного уравнения, глобальность решений задачи
	Коши. Понятие линейной зависимости и независимости произвольных функций, примеры,
	необходимое условие линейной зависимости. Необходимое и достаточное условие линейной
	зависимости. Общие свойства решений ЛОДУ. Вывод формулы Лиувилля-Остроградского.
	Опр-е ЛДУ n-го порядка, однородного и неоднородного. Сформулировать теорему, добавив, что в отличие
	от общей теоремы существования, решение задачи Коши для ЛДУ n-го порядка существует сразу на всём
	интервале определения коэффициентов уравнения. Опр-я ЛЗ и ЛНЗ функций. Необходимое условие ЛЗ
	функций (тожд-ое рав-во нулю определителя Вронского; достаточным это условие не является: привести
	пример, когда опр-ль Вронского тожд-но равен нулю, а функции ЛНЗ). Необходимое и достаточное условие
	ЛЗ (одна из функций линейно выражаетсяся через остальные). Общие свойства решений ЛОДУ:
	Предложение 1 (реш-е, полученное линейной комбинацией частных решений, тоже реш-е); Предложение 2
	(необходимые и достаточные условия ЛЗ и ЛНЗ функций, являющихся решениями ЛОДУ n-го порядка);
	Предложение 3 (Формула Лиувилля-Остроградского)
	\Ticket Общее решение линейного однородного ДУ произвольного порядка с переменными
	коэффициентами, пространство решений, фундаментальная система решений, размерность
	пространства решений. Задача о построении ЛОДУ по заданной ФСР. Понижение порядка линейного
	однородного ДУ при известном частном решении.
	Опр-е ФСР. Теорема (о существовании ФСР), Теорема (о структуре общего решения ЛОДУ), Утверждение
	(ЛОДУ n-го порядка не может иметь более чем n линейно независимых частных решений). Множество
	решений ЛОДУ n-го порядка образует n-мерное линейное пространство. Задача о построении ЛОДУ по
	заданной ФСР. Построение реш-я понижением порядка ЛОДУ при известном частном решении.
	\Ticket Линейное неоднородное ДУ n-го порядка с переменными коэффициентами, общее решение.
	Нахождение частного решения методом вариации произвольных постоянных. Предложение о
	структуре общего решения ЛНДУ. Метод вариации.
\end{description}
После коллоквиума (по состоянию 2 курса на 2023 год):
\begin{description}
	\Ticket Линейное однородное дифференциальное уравнение с постоянными коэффициентами, метод
	Эйлера, характеристическое уравнение, фундаментальная система решений в случае простых корней
	характеристического уравнения, фундаментальная система решений в случае кратных корней
	характеристического уравнения. Вещественные решения вещественного уравнения.
	Опр-е ЛОДУ с постоянными коэффициентами. Метод Эйлера. Характеристическое уравнение. Основная
	теорема алгебры (о числе корней многочлена n-ой степени). Лемма и следствие (о ФСР для ЛОДУ в случае
	простых корней хар.ур-я). Теорема (о ФСР для ЛОДУ в случае кратных корней). Лемма (случай
	вещественных коэф.ур-я, если оно имеет комплекснозначное реш-е, то его вещ-ая и мнимая части явл-ся
	реш-ми). Лемма (о вещественной ФСР, получающейся из набора комплексных компонент)
	\Ticket Линейное неоднородное дифференциальное уравнение с постоянными коэффициентами:
	получения частных решений в случае неоднородности типа квазиполинома. Линейное неоднородное
	дифференциальное уравнение с постоянными коэффициентами и неоднородностью произвольного
	вида: метод вариации произвольных постоянных. Уравнение Эйлера.
	Определение квазиполинома степени k, отвечающего комплексному числу. Теорема (о виде частного
	решения для ЛНДУ с неоднородностью типа квазиполинома). Метод вариации произвольных постоянных в
	случае неоднородности произвольного вида. Уравнение Эйлера (сведение к ЛДУ с постоянными
	коэффициентами).
	\Ticket Линейные дифференциальные системы с переменными коэффициентами: однородные и
	неоднородные системы, формулировка теоремы о существовании и елинственности решений задачи
	Коши, глобальность решений, общие свойства решений однородной системы. Линейная зависимость
	и независимость вектор-функций, необходимое условие линейной зависимости, определитель
	Вронского.
	Определение системы n ЛДУ 1-го пор-ка, однородной и неоднородной. Формулировка теоремы о сущ-ии и
	ед-ти решений з-чи Коши с указанием св-ва глобальности. Понятие ЛЗ и ЛНЗ в-р-функций. Необх.усл-е ЛЗ
	ф-ций, но не достаточное (пример). Лемма (линейная комбинация частных решений однор.сист- решение).
	Предложение (критерий линейной зависимости для n в-р-ф-ций, явл-ся решениями однор.системы – рав-во
	нулю их опред-ля Вронского).
	\Ticket Пространство решений линейной однородной дифференциальной системы с переменными
	коэффициентами, фундаментальные решения, фундаментальная матрица системы. Формула
	Лиувилля-Остроградского для системы. Описание всех фундаментальных матриц системы.
	Теорема (Пространство реш-й системы n лин.однор.ДУ 1-го пор-ка образует n-мерное векторное пр-во).
	Формула Лиувилля-Остроградского для системы. Фунд. система решений, фундаментальная матрица.
	Описание всех фундаментальных матриц системы.
	\Ticket Линейные неоднородные дифференциальной системы с переменными коэффициентами, общее
	решение неоднородной системы. Метод вариации постоянных для поиска частного решения.
	Предложение (о структуре общего реш-я неоднородной системы). Метод вариации.
	\Ticket Линейные однородные дифференциальные системы с постоянными коэффициентами. Метод
	Эйлера, характеристическое уравнение. Фундаментальная система решений в случае простых корней
	характеристического уравнения, фундаментальная система решений в случае кратных корней
	характеристического уравнения. Вещественные решения вещественной системы в случае
	комплексных корней характеристического уравнения.
	Метод Эйлера. Характеристическое уравнение. Предложение (о ФСР в случае простых корней хар.ур-я).
	Случай комплексных корней хар.ур-я (выделение вещественных реш-й). Случай кратных корней хар.ур-я
	(Теорема о ЖНФ, присоединенные векторы; метод неопределенных коэффициентов, понятие
	алгебраической и геометрической кратности собственного значения.). Общее реш-е системы для матрицы 2
	на 2, 3 на 3.
	\Ticket Нахождение частных решений неоднородной системы с постоянными коэффициентами и
	неоднородностью в виде векторного квазиполинома.
	\Ticket Матричная экспонента и ее свойства.
	\Ticket Понятие об устойчивости решения ДУ и ДС по Ляпунову. Примеры устойчивости и
	неустойчивости по Ляпунову. Условие устойчивости по Ляпунову нулевого решения линейной
	однородной системы. Критерий Рауса–Гурвица.
	Рассматривается неавтономная система n ур-й 1-го пор-ка. Опр-е реш-я сист, устойчивого и
	асимптотической устойчивости по Ляпунову. Примеры скалярных уравнений, имеющих устойчивое и
	неустойчивое по Ляпунову нулевое реш-е. Устойчивость состояния равновесия типа центр линейной
	автономной системы на плоскости (не является асимптотически устойчивым). Исследование устойчивости
	произвольного решения системы (всегда можно свести к исследованию устойчивости нулевого реш-я с
	помощью некоторой замены). Теорема об устойчивости по Ляпунову нулевого реш-я ЛОС. Матрица
	Гурвица. Критерий Рауса-Гурвица (все корни имеют отрицательные действительные части тогда и только
	тогда, когда все главные диагональные миноры матрицы Гурвица положительны).
	\Ticket Исследование устойчивости с помощью функций Ляпунова. Устойчивость по первому
	приближению.
	Производная функции в силу системы. Теорема Ляпунова об устойчивости (с док-вом). Теорема Ляпунова
	об асимптотической устойчивости (без док-ва). Теорема Ляпунова о неустойчивости (без док-ва). Теорема
	Четаева о неустойчивости (без док-ва). Теорема об устойчивости по 1-ому приближению (без док-ва).
	\Ticket Понятие об автономных системах, фазовое пространство, фазовые траектории и их свойства.
	Окрестность состояния равновесия автономной системы, фазовые портреты линейных систем на
	плоскости. Случаи узлов и седел. Случаи фокусов и центров. Узлы с кратными корнями и состояния
	равновесия с нулевым корнем. Теорема о типах состояний равновесия для двумерной автономной
	системы на плоскости с произвольными правыми частями. Проблема различения центра и фокуса.
	Предельные циклы. Признаки отсутствия ЗФТ.
	Опр-е автономной системы ДУ 1-го пор-ка. Понятия фазового пространства и фазовой траектории. Свойства
	фазовых траекторий (доказать любые два). Линейные двумерные автономные системы на плоскости, к
	изучению фазовых портретов которых приводит изучение автономных систем на плоскости в окрестности
	состояния равновесия. Теорема о типах состояний равновесия для двумерной автономной системы на
	плоскости с произвольными правыми частями. Достаточное условие существование центра. Теорема
	Ляпунова (необходимое и достаточное условие наличия центра). Изолированные и неизолированные
	фазовые траектории. Определение предельного цикла (ПЦ), устойчивого и неустойчивого ПЦ. Орбитальная
	устойчивость (в чем отличие от устойчивости по Ляпунову). Принцип кольца для «распознавания»
	предельных циклов. Признаки Дюлака и Бендиксона отсутствия ЗФТ.
	\Ticket Теорема о существовании и единственности решения задачи Коши для ДУ 1-го порядка: теорема
	Пикара, доказательство методом последовательных приближений Пикара. Лемма Гронуолла с
	доказательством.
	\Ticket Понятие о продолжении решений ДУ 1-го порядка. Доказательство теоремы о продолжении
	решения задачи Коши для уравнения первого порядка, разрешенного относительно производной.
	\Ticket Непрерывная зависимость решения задачи Коши от параметров и начальных условий.
	Формулировка теорем без доказательств.
	\Ticket Первые интегралы. Определение, физический и геометрический смыслы первого интеграла.
	Определение функционально независимой системы первых интегралов. Формулировки теорем о
	существовании функционально независимых первых интегралов (для неавтономной и автономной систем).
\end{description}
\end{document}