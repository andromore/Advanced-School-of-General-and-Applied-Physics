\documentclass{article}
\usepackage{mathtext} % использование кириллицы в формулах
\usepackage{cmap} % грамотное копирование кириллицы из pdf
\usepackage[T2A]{fontenc} % внутрення кодировка
\usepackage[utf8]{inputenc} % кодировка документа
\usepackage[russian]{babel} % язык документа
\usepackage{amssymb} % дополнительные символы
\usepackage{amsfonts} % математические шрифты
\usepackage{amsmath} % дополнительная математика
\usepackage{ifthen}
\makeatletter
\renewcommand{\section}{\@startsection{section}{1}{0pt}{1em}{1em}{\Large\bf}}
\renewcommand{\subsection}{\@startsection{subsection}{2}{0pt}{1em}{1em}{\Large\bf}}
\renewcommand{\subsubsection}{\@startsection{subsubsection}{3}{0pt}{1em}{1em}{\Large\bf}}
\newcounter{ticket}[subsection]
\newenvironment{ticket}[1][]{\item[Билет \ifthenelse{\equal{#1}{}}{}{\setcounter{ticket}{#1}}\theticket\refstepcounter{ticket}:]\phantom{}\begin{enumerate}}{\end{enumerate}}
\newcounter{Item}[subsection]
\newcommand{\Item}[1][]{\item[Билет \ifthenelse{\equal{#1}{}}{}{\setcounter{ticket}{#1}}\theticket\refstepcounter{ticket}:]}
\makeatother
\title{Вопросы и билеты}
\date{2023}
\begin{document}
\maketitle
\tableofcontents
\section{Семестр}
\subsection{Математический анализ}
\begin{description}
	\begin{ticket}[1]
		\item Мощность множеств. Счётные множества и множества мощности континуум.
		\item Число $e$.
		\item Интегралы с переменным верхним пределом и их свойства.
	\end{ticket}
	\begin{ticket}
		\item Определение точной верхней и точной нижней границы множества.
		\item Вторая теорема Больцано-Коши.
		\item Интеграл с переменным верхним пределом и его свойства.
			Определение интеграла Римана и его графический смысл.
	\end{ticket}
	\begin{ticket}
		\item Теорема о существовании арифметического корня.
		\item Первая теорема Больцано-Коши.
		\item Формула Ньютона-Лейбница.
	\end{ticket}
	\begin{ticket}
		\item Теорема о вложенных отрезках (в т.ч. в случае стремления к нулю длинны отрезков).
		\item Эквивалентность определений предела по Коши и по Гейне.
		\item Достаточное условие интегрируемости монотонной функции.
	\end{ticket}
	\begin{ticket}[6]
		\item Теорема о единственности предела последовательности.
		\item Второй замечательный предел.
		\item Интегралы с подстановками Эйлера.
	\end{ticket}
	\begin{ticket}
		\item Необходимое условие сходимости.
		\item Первый замечательный предел.
		\item Критерии интегрирования в терминах колебаний.
	\end{ticket}
	\begin{ticket}
		\item Бином Ньютона.
		\item Первая теорема Вейерштрасса (возможные вопросы: определения ограниченной и непрерывной функции).
		\item Эквивалентные условия интегрируемости в терминах колебаний.
	\end{ticket}
	\begin{ticket}
		\item Теорема Больцано-Вейерштрасса о подпоследовательностях.
		\item Эквивалентность дифференцируемости и существования производной в точке.
		\item Свойства сумм Дарбу.
	\end{ticket}
	\begin{ticket}
		\item Критерий Коши для последовательностей.
		\item Теорема о непрерывности обратной функции.
		\item Интеграл Римана. Необходимое условие интегрируемости.
	\end{ticket}
	\begin{ticket}
		\item Теорема Вейерштрасса для монотонной функции.
		\item Теорема Лагранжа.
		\item Линейные свойства определённого интеграла.
	\end{ticket}
	\begin{ticket}
		\item Арифметические свойства предела.
		\item Теорема Коши.
		\item Аддитивные свойства определённого интеграла.
	\end{ticket}
	\begin{ticket}
		\item Неполнота и алгебраическая незамкнутость поля рациональных чисел.
		\item Теорема Ролля (возможные вопросы: теорема Вейерштрасса и теорема Ферма).
		\item Множества меры ноль и их свойства. Критерий Лебега интегрируемости по Риманы.
	\end{ticket}
	\begin{ticket}
		\item Теорема о двух милиционерах.
		\item Вторая теорема Вейерштрасса.
		\item Условия интегрируемости для непрерывных функций и функций с конечным числом разрывов.
	\end{ticket}
	\begin{ticket}
		\item Сравнение бесконечно малых функций (возможные вопросы: определение предела функции по Коши и по Гейне).
		\item Параметрически заданные функции и их производные (возможные вопросы: определение производной, теорема Коши).
		\item Интегрируемость суммы и произведения интегрируемых функций.
	\end{ticket}
\end{description}
\subsection{Аналитическая геометрия}
\begin{description}
	\Item[1] Операции над векторами и их свойства. Доказать свойство ассоциативности.
	\Item Определение пропорциональности векторов. Доказать равносильность коллинеарности и пропорциональности векторов.
	\Item Геометрическое определение базиса на прямой $V_1$, плоскости $V_2$ и пространстве $V_3$. Теоремы о разложении любого вектора по базису (случаи $V_1$, $V_2$, $V_3$).
	\Item Определение координат вектора (рассмотреть случай пространства $V_3$). Операции над векторами в координатной форме. Доказать критерий коллинеарности векторов в координатной форме.
	\Item Определение линейно зависимых и линейно независимых систем векторов. Алгебраическое определение базиса в векторном пространстве.
	\Item Определение скалярного произведения векторов и его свойства. Выражение скалярного произведения векторов в координатной форме в общем случае и случае ортонормированного базиса.
	\Item Понятие ориентации тройки векторов и циклической перестановки. Определение и свойства векторного произведения векторов.
	\Item Определение смешанного произведения векторов. Доказать численное равенство модуля смешанного произведения векторов и значения объёма параллелепипеда, построенного на этих векторах.
	\Item Критерий коллинеарности трёх векторов.
	\Item Вывод формулы векторного произведения в координатной форме в общем случае и в случае ортонормированного базиса.
	\Item Вывод формулы смешанного произведения в координатной форме.
	\Item Матрица перехода в $V_2$. Невырожденность матрицы перехода. Закон изменения координат вектора при изменении базиса.
	\Item Определение аффинного пространства и аффинной системы координат. Определение координат точки и координат вектора. Изменение координат точки при изменении системы координат.
	\Item Уравнение линии на плоскости. Определение алгебраической линии $n$-ого порядка. Доказать инвариантность понятия алгебраической линии и её порядка относительно изменения декартовой системы координат для случая алгебраической линии первого порядка.
	\Item Вывод векторного, параметрического и канонического уравнений прямой на плоскости.
	\Item Вывод уравнения прямой по двум точкам и общего уравнения прямой.
	\Item Доказать теорему о взаимном расположении прямых на плоскости.
	\Item Определение пучка прямых на плоскости. Вывод уравнения пучка прямых на плоскости.
	\Item Вывод формулы расстояния от точки до прямой на плоскости.
	\Item Определение алгебраической поверхности и её порядка. Векторно-параметрическое и параметрическое уравнения плоскости.
	\Item Теорема о задании линейным уравнением в произвольной системе координат и обратный случай. Общее уравнение плоскости.
	\Item Определение вектора нормали к плоскости.Координаты вектора нормали к плоскости, заданной общим уравнением в прямоугольной системе координат.
	\Item Вывод формулы расстояния от точки до плоскости.
	\Item Уравнение линии в пространстве. Вывод уравнений прямых в пространстве.
	\Item Вывод формулы расстояния между скрещивающимися прямыми.
	\Item Эллипс и теорема о модулях радиус-векторов эллипса.
	\Item Критерий принадлежности точки эллипсу ($r_1 + r_2 = 2a$).
	\Item Гипербола и теорема о модуле радиус-векторов гиперболы.
	\Item Критерий принадлежности точки гиперболе ($|r_1 - r_2| = 2a$).
	\Item Парабола.
\end{description}
\subsection{История}
\begin{description}
	\Item[1] Славяне. Восточные славяне.
	\Item Киевская Русь и её соседи. Норманская теория. Обзор деятельности первых русских князей.
	\Item Социально-экономическое и политическое развитие Киевской Руси в IX -- X веках.
	\Item Княжение Владимира. Крещение Руси. Значение принятия христианства.
	\Item Время Ярослава Мудрого. Киевская Русь в XI веке.
	\Item Княжение Владимира Мономаха (1113 -- 1125). Киевская Русь в XII веке. Феодальная раздробленность Руси. Владимиро-Суздальская Русь.
	\Item Невская битва (1240 год). Александр Невский. Ледовое побоище (1242 год).
	\Item Русь и монголо-татары.
	\Item Великое княжество Московское в XIV -- XV веках. Иван Калита.
	\Item Борьба Московского княжества с Золотой Ордой. Дмитрий Иванович Донской. Куликовская битва.
	\Item Объединение русских земель вокруг Московского княжества.
	\Item Объединение русских земель в единое государство. Иван III.
	\Item Россия в первой половине XVI века.
	\Item Реформы Ивана IV Грозного. Опричнина.
	\Item Внешняя политика Ивана IV.
	\Item Русское государство в конце XVI века. Внутренняя и внешняя политика царя Фёдора (1584 -- 1598).
	\Item <<Смута>> в России в начале XVII века. Самозванничество.
	\Item Борьба русского народа с польской интервенцией. Нижегородское ополчение под предводительством Кузьмы (в последствии Козьмы) Минина и Дмитрия Пожарского.
	\Item Правление царя Михаила Фёдоровича Романова (1613 -- 1645).
	\Item Правление царя Алексея Михайловича (1645 -- 1676). Соборное уложение 1649 года.
	\Item Церковь и раскол в XVII веке. Протопоп Аввакум. Патриарх Никон. Соловецкое восстание.
	\Item Внешняя политика при Алексее Михайловиче. Воссоединение Украины с Россией.
	\Item Социальные протесты XVII века. Крестьянская война под предводительством Степана Тимофеевича Разина.
	\Item Царствование Петра I. Социально-экономическое развитие России на рубеже XVII -- XVIII веков.
	\Item Внешняя политика Петра I.
	\Item Реформы Петра I Великого и их историческое значение.
	\Item Дворцовые перевороты.
	\Item Внутренняя политика Екатерины II. <<Уложенная комиссия>> (1767 -- 1768 годов).
	\Item Просвещённый абсолютизм.
	\Item Социально-политическое развитие России во второй половине XVIII века. <<Золотой век дворянства>>.
	\Item Внешняя политика России в период правления Екатерины II (1762 -- 1796 годы).
	\Item Правление Павла I (1796 -- 1801 годы).
\end{description}
\section{Семестр}
\subsection{Математический анализ}
\begin{description}
	\item[Билет 1:]\phantom{}
	\begin{enumerate}
		\item Признаки сравнения сходимости несобственных интегралов.
		\item Формула Тейлора для функции нескольких переменных.
		\item Переход к пределу под знаком интеграла для
			семейства фукнций.
	\end{enumerate}
	\item[Билет 4:]\phantom{}
	\begin{enumerate}
		\item Компактные множества в метрических пространствах.
		Необходимое условие компактности.
		\item Признак Раабе.
		\item Ряды Фурье. Коэффициенты тригонометрического ряда Фурье.
	\end{enumerate}
	\item[Билет 5:]\phantom{}
	\begin{enumerate}
		\item Связность и линейная связность. Образ связного множества
		при непрерывном отображении.
		\item Признак Дирихле для числового ряда.
		\item Переход к пределу под знаком интеграла для
		семейства функций.
	\end{enumerate}
	\item[Билет 6:]\phantom{}
	\begin{enumerate}
		\item Критерий компактности в $\mathbb{R}^n$.
		\item Признак Даламбера сходимости положительного ряда.
		\itemДифференцируемость интеграла, зависящего от параметра.
	\end{enumerate}
	\item[Билет 7:]\phantom{}
	\begin{enumerate}
		\item Образ компакта при непрерывном отображении.
		\item Формула Коши-Адамара.
		\item Бета-функция и её свойства.
	\end{enumerate}
	\item[Билет 8:]\phantom{}
	\begin{enumerate}
		\item Достаточное условие дифференцируемости в терминах
		частных производных.
		\item Критерий сходимости положительного ряда.
		\item Равенство Парсеваля и неравенство Бесселя.
	\end{enumerate}
	\item[Билет 9:]\phantom{}
	\begin{enumerate}
		\item Теорема о дифференцируемости композиции дифференцируемых
		функций.
		\item Равномерная сходимость и интегрирование.
		\item Разложение в ряд Тейлора: $ln(1 + x)$, $arc tg(x)$.
	\end{enumerate}
	\item[Билет 10:]\phantom{}
	\begin{enumerate}
		\item Связность и линейная связность. Образ связного множества
		при непрерывном отображении.
		\item Достаточное условие абсолютного экстремума.
		\item Интегральный признак сходимости.
	\end{enumerate}
	\item[Билет 11:]\phantom{}
	\begin{enumerate}
		\item Инвариантность первого дифференциала.
		\item Совпадение смешанных частных производных.
		\item Равномерная сходимость несобственных интегралов,
		зависящих от параметра. Аналог теоремы Вейерштрасса.
	\end{enumerate}
	\item[Билет 12:]\phantom{}
	\begin{enumerate}
		\item Равномерная непрерывность. Обобщение теоремы Кантора.
		\item Теорема Абеля о поведении степенного ряда на границе
		интервала сходимости.
		\item Признак Дини.
	\end{enumerate}
\end{description}
\subsection{Физика}
\begin{description}
	\item[Билет 3:]\phantom{}
	\begin{enumerate}
		\item Идеальный газ. Агрегатные состояния вещества.
		Закон Дальтона.
		\item Молекулярно-кинетическая формулировка температуры
		и теплового равновесия.
	\end{enumerate}
	\item[Билет 4:]\phantom{}
	\begin{enumerate}
		\item Теплоёмкость.
		\item Распределение Ферми-Дирака
	\end{enumerate}
	\item[Билет 9:]\phantom{}
	\begin{enumerate}
		\item Статистика Бозе-Эйнштейна.
		\item Вязкость газа; внутреннее трение; коэффициент вязкости;
		сила вязкого трения; оценка коэффициента вязкости.
	\end{enumerate}
	\item[Билет 10:]\phantom{}
	\begin{enumerate}
		\item Термодинамические процессы; квазистатические процессы
		(обратимые); адиабатическое расширение и сжатие
		(общий вид, примеры).
		\item Спектр излучения абсолютно чёрного тела; распределение
		по степеням свободы для электромагнитного излучения;
		формула Планка; закон Стефана-Больцмана; закон Вина.
	\end{enumerate}
	\item[Билет 13:]\phantom{}
	\begin{enumerate}
		\item Второе начало термодинамики.
		\item Фазы вещества: классификация и условие равновесия.
	\end{enumerate}
	\item[Билет 14:]\phantom{}
	\begin{enumerate}
		\item Обратимые круговые процессы. Идеальный газ. Цикл Карно.
		\item Вывод уравнения Клайперона-Клаузиуса.
	\end{enumerate}
	\item[Билет 17:]\phantom{}
	\begin{enumerate}
		\item Энтропия обратимых и необратимых процессов.
		Неравенство Клаузиуса.
		\item Поверхностное натяжение и его термодинамический смысл
		(связь коэффициента полезного действия с температурой).
	\end{enumerate}
	\item[Билет 18:]\phantom{}
	\begin{enumerate}
		\item Коэффициент полезного действия в необратимом круговом процессе
		(физические причины необратимости; примеры).
		\item Смачивание и несмачивание; условие равновесия границы;
		краевой угол (примеры).
	\end{enumerate}
	\item[Билет 19:]\phantom{}
	\begin{enumerate}
		\item Реальные газы (определение); газ Ван-дер-Ваальса
		(уравнение; изотермы в координатах $P(V)$; внутренняя энергия;
		теплоёмкость).
		\item Соотношение между давлением и кривизной поверхности;
		формулы Лапласа; закон Жюрена.
	\end{enumerate}
	\item[Билет 20:]\phantom{}
	\begin{enumerate}
		\item Термодинамическое и статистическое определение
		макропараметров физической системы. Эргодические системы.
		\item Характер движения отдельной частицы в газе.
		Длина свободного пробега. Среднеквадратичное отклонение
		частицы от начального положения.
	\end{enumerate}
\end{description}
\subsection{Линейная алгебра}
\begin{description}
	\item[Билет 2:]
		Подпространства. Разложение подпространства в прямую сумму
		подпространств. Примеры разложения на подпространства.
	\item[Билет 17:]
		Унитарные операторы и их матрицы.
		(Всё, что знаете)
	\item[Билет 18:]
		Ортогональные операторы и их матрицы.
		(Всё, что знаете)
	\item[Билет 20:]
		Квадратичные формы: определение и диагонализация
		методом ортогональных преобразований. Закон инерции.
	\item[Билет 21:]
		Метод Якоби диагонализации квадратичных форм.
	\item[Билет 22:]
		Положительно определённые квадратичные формы
		и операторы в терминах квадратичных форм.
		(Определение; Необходимые и достаточные условия)
	\item[Билет 24:]
		Одновременная диагонализация квадратичных форм
\end{description}
\end{document}
