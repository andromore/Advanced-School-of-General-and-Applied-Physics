\documentclass{article}
\usepackage{mathtext} % использование кириллицы в формулах
\usepackage{cmap} % грамотное копирование кириллицы из pdf
\usepackage[T2A]{fontenc} % внутрення кодировка
\usepackage[utf8]{inputenc} % кодировка документа
\usepackage[russian]{babel} % язык документа
\usepackage{amssymb} % дополнительные символы
\usepackage{amsfonts} % математические шрифты
\usepackage{amsmath} % дополнительная математика
\makeatletter
\renewcommand{\section}{\@startsection{section}{1}{0pt}{1em}{1em}{\Large\bf}}
\renewcommand{\subsection}{\@startsection{subsection}{2}{0pt}{1em}{1em}{\Large\bf}}
\renewcommand{\subsubsection}{\@startsection{subsubsection}{3}{0pt}{1em}{1em}{\Large\bf}}
\makeatother
\title{Вопросы и билеты}
\date{2023}
\begin{document}
\maketitle
\tableofcontents
\section{Семестр}
\section{Семестр}
\subsection{Математический анализ}
\begin{description}
	\item[Билет 1:]\phantom{}
	\begin{enumerate}
		\item Признаки сравнения сходимости несобственных интегралов.
		\item Формула Тейлора для функции нескольких переменных.
		\item Переход к пределу под знаком интеграла для
			семейства фукнций.
	\end{enumerate}
	\item[Билет 4:]\phantom{}
	\begin{enumerate}
		\item Компактные множества в метрических пространствах.
		Необходимое условие компактности.
		\item Признак Раабе.
		\item Ряды Фурье. Коэффициенты тригонометрического ряда Фурье.
	\end{enumerate}
	\item[Билет 5:]\phantom{}
	\begin{enumerate}
		\item Связность и линейная связность. Образ связного множества
		при непрерывном отображении.
		\item Признак Дирихле для числового ряда.
		\item Переход к пределу под знаком интеграла для
		семейства функций.
	\end{enumerate}
	\item[Билет 6:]\phantom{}
	\begin{enumerate}
		\item Критерий компактности в $\mathbb{R}^n$.
		\item Признак Даламбера сходимости положительного ряда.
		\itemДифференцируемость интеграла, зависящего от параметра.
	\end{enumerate}
	\item[Билет 7:]\phantom{}
	\begin{enumerate}
		\item Образ компакта при непрерывном отображении.
		\item Формула Коши-Адамара.
		\item Бета-функция и её свойства.
	\end{enumerate}
	\item[Билет 8:]\phantom{}
	\begin{enumerate}
		\item Достаточное условие дифференцируемости в терминах
		частных производных.
		\item Критерий сходимости положительного ряда.
		\item Равенство Парсеваля и неравенство Бесселя.
	\end{enumerate}
	\item[Билет 9:]\phantom{}
	\begin{enumerate}
		\item Теорема о дифференцируемости композиции дифференцируемых
		функций.
		\item Равномерная сходимость и интегрирование.
		\item Разложение в ряд Тейлора: $ln(1 + x)$, $arc tg(x)$.
	\end{enumerate}
	\item[Билет 10:]\phantom{}
	\begin{enumerate}
		\item Связность и линейная связность. Образ связного множества
		при непрерывном отображении.
		\item Достаточное условие абсолютного экстремума.
		\item Интегральный признак сходимости.
	\end{enumerate}
	\item[Билет 11:]\phantom{}
	\begin{enumerate}
		\item Инвариантность первого дифференциала.
		\item Совпадение смешанных частных производных.
		\item Равномерная сходимость несобственных интегралов,
		зависящих от параметра. Аналог теоремы Вейерштрасса.
	\end{enumerate}
	\item[Билет 12:]\phantom{}
	\begin{enumerate}
		\item Равномерная непрерывность. Обобщение теоремы Кантора.
		\item Теорема Абеля о поведении степенного ряда на границе
		интервала сходимости.
		\item Признак Дини.
	\end{enumerate}
\end{description}
\subsection{Физика}
\begin{description}
	\item[Билет 3:]\phantom{}
	\begin{enumerate}
		\item Идеальный газ. Агрегатные состояния вещества.
		Закон Дальтона.
		\item Молекулярно-кинетическая формулировка температуры
		и теплового равновесия.
	\end{enumerate}
	\item[Билет 4:]\phantom{}
	\begin{enumerate}
		\item Теплоёмкость.
		\item Распределение Ферми-Дирака
	\end{enumerate}
	\item[Билет 9:]\phantom{}
	\begin{enumerate}
		\item Статистика Бозе-Эйнштейна.
		\item Вязкость газа; внутреннее трение; коэффициент вязкости;
		сила вязкого трения; оценка коэффициента вязкости.
	\end{enumerate}
	\item[Билет 10:]\phantom{}
	\begin{enumerate}
		\item Термодинамические процессы; квазистатические процессы
		(обратимые); адиабатическое расширение и сжатие
		(общий вид, примеры).
		\item Спектр излучения абсолютно чёрного тела; распределение
		по степеням свободы для электромагнитного излучения;
		формула Планка; закон Стефана-Больцмана; закон Вина.
	\end{enumerate}
	\item[Билет 13:]\phantom{}
	\begin{enumerate}
		\item Второе начало термодинамики.
		\item Фазы вещества: классификация и условие равновесия.
	\end{enumerate}
	\item[Билет 14:]\phantom{}
	\begin{enumerate}
		\item Обратимые круговые процессы. Идеальный газ. Цикл Карно.
		\item Вывод уравнения Клайперона-Клаузиуса.
	\end{enumerate}
	\item[Билет 17:]\phantom{}
	\begin{enumerate}
		\item Энтропия обратимых и необратимых процессов.
		Неравенство Клаузиуса.
		\item Поверхностное натяжение и его термодинамический смысл
		(связь коэффициента полезного действия с температурой).
	\end{enumerate}
	\item[Билет 18:]\phantom{}
	\begin{enumerate}
		\item Коэффициент полезного действия в необратимом круговом процессе
		(физические причины необратимости; примеры).
		\item Смачивание и несмачивание; условие равновесия границы;
		краевой угол (примеры).
	\end{enumerate}
	\item[Билет 19:]\phantom{}
	\begin{enumerate}
		\item Реальные газы (определение); газ Ван-дер-Ваальса
		(уравнение; изотермы в координатах $P(V)$; внутренняя энергия;
		теплоёмкость).
		\item Соотношение между давлением и кривизной поверхности;
		формулы Лапласа; закон Жюрена.
	\end{enumerate}
	\item[Билет 20:]\phantom{}
	\begin{enumerate}
		\item Термодинамическое и статистическое определение
		макропараметров физической системы. Эргодические системы.
		\item Характер движения отдельной частицы в газе.
		Длина свободного пробега. Среднеквадратичное отклонение
		частицы от начального положения.
	\end{enumerate}
\end{description}
\subsection{Линейная алгебра}
\begin{description}
	\item[Билет 2:]
		Подпространства. Разложение подпространства в прямую сумму
		подпространств. Примеры разложения на подпространства.
	\item[Билет 17:]
		Унитарные операторы и их матрицы.
		(Всё, что знаете)
	\item[Билет 18:]
		Ортогональные операторы и их матрицы.
		(Всё, что знаете)
	\item[Билет 20:]
		Квадратичные формы: определение и диагонализация
		методом ортогональных преобразований. Закон инерции.
	\item[Билет 21:]
		Метод Якоби диагонализации квадратичных форм.
	\item[Билет 22:]
		Положительно определённые квадратичные формы
		и операторы в терминах квадратичных форм.
		(Определение; Необходимые и достаточные условия)
	\item[Билет 24:]
		Одновременная диагонализация квадратичных форм
\end{description}
\end{document}
