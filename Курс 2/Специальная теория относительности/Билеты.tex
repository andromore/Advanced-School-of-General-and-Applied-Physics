\documentclass{article}
\usepackage{mathtext} % использование кириллицы в формулах
\usepackage{cmap} % грамотное копирование кириллицы из pdf
\usepackage[T2A]{fontenc} % внутрення кодировка
\usepackage[utf8]{inputenc} % кодировка документа
\usepackage[russian]{babel} % язык документа
\usepackage{amssymb} % дополнительные символы
\usepackage{amsfonts} % математические шрифты
\usepackage{amsmath} % дополнительная математика
\usepackage{indentfirst} % отступ и у первого абзаца
\author{Можаров А.Р.}
\date{19 мая 2024}
\title{Расписанные билеты по СТО}
\begin{document}
\maketitle
\begin{enumerate}
    \item Опыт Майкельсона-Морли

          Трудно представить себе абсолютную пустоту — полный вакуум, не содержащий чего бы то ни было. Человеческое сознание стремится заполнить его хоть чем-то материальным, и на протяжении долгих веков человеческой истории считалось, что мировое пространство заполнено эфиром. Идея состояла в том, что межзвездное пространство заполнено какой-то невидимой и неосязаемой тонкой субстанцией. Когда была получена система уравнений Максвелла, предсказывающая, что свет распространяется в пространстве с конечной скоростью, даже сам автор этой теории полагал, что электромагнитные волны распространяются в среде, подобно тому, как акустические волны распространяются в воздухе, а морские — в воде. В первой половине XIX столетия ученые даже тщательно проработали теоретическую модель эфира и механику распространения света, включая всевозможные рычаги и оси, якобы способствующие распространению колебательных световых волн в эфире.

          В 1887 году два американских физика — Альберт Майкельсон и Генри Морли — решили совместно провести эксперимент, призванный раз и навсегда доказать скептикам, что светоносный эфир реально существует, наполняет Вселенную и служит средой, в которой распространяются свет и прочие электромагнитные волны. Майкельсон обладал непререкаемым авторитетом как конструктор оптических приборов, а Морли славился как неутомимый и непогрешимый физик-экспериментатор. Придуманный ими опыт проще описать, чем провести практически.

          Майкельсон и Морли использовали интерферометр — оптический измерительный прибор,
          в котором луч света расщепляется надвое полупрозрачным зеркалом
          (стеклянная пластина посеребрена с одной стороны ровно настолько,
          чтобы частично пропускать поступающие на нее световые лучи,
          а частично отражать их;
          аналогичная технология сегодня используется в зеркальных фотоаппаратах).
          В итоге луч расщепляется и два получившихся когерентных луча расходятся
          под прямым углом друг к другу, после чего отражаются от двух равноудаленных
          от полупрозрачного зеркала зеркал-отражателей и возвращаются
          на полупрозрачное зеркало, результирующий пучок света
          от которого позволяет наблюдать интерференционную картину
          и выявлять малейшую разность хода двух лучей.

          Опыт Майкельсона—Морли был принципиально направлен на то, чтобы подтвердить (или опровергнуть) существование мирового эфира посредством выявления «эфирного ветра» (или факта его отсутствия). Действительно, двигаясь по орбите вокруг Солнца, Земля совершает движение относительно гипотетического эфира полгода в одном направлении, а следующие полгода в другом. Следовательно, полгода «эфирный ветер» должен обдувать Землю и, как следствие, смещать показания интерферометра в одну сторону, полгода — в другую. Итак, наблюдая в течение года за своей установкой, Майкельсон и Морли не обнаружили никаких смещений в интерференционной картине: полный эфирный штиль! (Современные эксперименты подобного рода, проведенные с максимально возможной точностью, включая эксперименты с лазерными интерферометрами, дали аналогичные результаты.) Итак: эфирного ветра, а, стало быть, и эфира не существует.

          В отсутствие эфирного ветра и эфира, как такового, стал очевиден неразрешимый конфликт между классической механикой Ньютона (подразумевающей некую абсолютную систему отсчета) и уравнениями Максвелла (согласно которым скорость света имеет предельное значение, не зависящее от выбора системы отсчета), что и привело в итоге к появлению теории относительности. Опыт Майкельсона—Морли окончательно показал, что «абсолютной системы отсчета» в природе не существует. И, сколько бы Эйнштейн впоследствии ни утверждал, что вообще не обращал внимания на результаты экспериментальных исследований при разработке теории относительности, сомневаться в том, что результаты опытов Майкельсона — Морли способствовали быстрому восприятию столь радикальной теории научной общественностью всерьез, вряд ли приходится.
    \item
\end{enumerate}
\end{document}