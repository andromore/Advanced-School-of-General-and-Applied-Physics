\documentclass{article}
\usepackage{mathtext} % использование кириллицы в формулах
\usepackage{cmap} % грамотное копирование кириллицы из pdf
\usepackage[T2A]{fontenc} % внутрення кодировка
\usepackage[utf8]{inputenc} % кодировка документа
\usepackage[russian]{babel} % язык документа
\usepackage{amssymb} % дополнительные символы
\usepackage{amsfonts} % математические шрифты
\usepackage{amsmath} % дополнительная математика
\usepackage{ifthen}
\makeatletter
\newcounter{ticket}[subsection]
\newenvironment{ticket}[1][]{\item[Билет \ifthenelse{\equal{#1}{}}{}{\setcounter{ticket}{#1}}\theticket\refstepcounter{ticket}:]\phantom{}\begin{enumerate}}{\end{enumerate}}
\newcounter{Ticket}[subsection]
\newcommand{\Ticket}[1][]{\item[Билет \ifthenelse{\equal{#1}{}}{}{\setcounter{ticket}{#1}}\theticket\refstepcounter{ticket}:]}
\makeatother
\title{Вопросы к экзамену по курсу \mbox{<<Колебания и волны, оптика>>}}
\date{Последнее обновление: \today}
\author{Национальный исследовательский \\
Нижегородский Государственный Университет \\
имени Н.И. Лобачевского \vspace{0.5em} \\
Высшая Школа Общей и Прикладной Физики \vspace{0.5em} \\
Анашкина Елена Александровна \vspace{0.5em}}
\begin{document}
\maketitle
\begin{enumerate}
    \Ticket[1] Основные характеристики скалярного гармонического колебания. Примеры механических и электрических колебаний. Уравнение гармонического осциллятора, его решение, интеграл энергии.
    \Ticket Свободные колебания в консервативных системах. Анализ движения при помощи фазовой плоскости. Фазовый портрет гармонического осциллятора, фазовый портрет нелинейного осциллятора на примере физического маятника. 
    \Ticket Линейный осциллятор с затуханием, частота и декремент слабозатухающих колебаний, добротность, апериодические колебания. Фазовый портрет, энергетические соотношения. 
    \Ticket Сложение двух синхронных скалярных гармонических колебаний. Сложение двух взаимно ортогональных векторных колебаний. Сложение двух скалярных гармонических колебаний с близкими частотами. Биения.
    \Ticket Метод векторных диаграмм. Метод комплексных амплитуд. 
    \Ticket Сложение произвольного числа синхронных гармонических колебаний. Сложение колебаний равной амплитуды, фазы которых образуют арифметическую прогрессию.
    \Ticket Действие периодической импульсной силы на гармонический осциллятор без затухания. Принцип суперпозиции на примере действия периодической импульсной силы.
    \Ticket Вынужденные колебания гармонического осциллятора (без затухания, с затуханием) под действием периодической импульсной вынуждающей силы.
    \Ticket Движение гармонического осциллятора (без затухания, с затуханием) под действием внешней синусоидальной силы. Резонанс, резонансные кривые, добротность, установление колебаний.
    \Ticket Метод Лагранжа вариации произвольной постоянной. Применение метода Лагранжа для гармонического осциллятора с произвольной вынуждающей силой.
    \Ticket Параметрический резонанс. Теорема Флоке для уравнения Хилла. Уравнение Матье, параметрический резонанс в основной зоне Матье. 
    \Ticket Колебания систем в быстро осциллирующем поле, высокочастотный потенциал. Маятник Капицы. Фазовый портрет при вертикальных колебаниях точки подвеса. Фазовый портрет при горизонтальных колебаниях точки подвеса 
    \Ticket Колебания систем с медленно меняющимися параметрами. ВКБ приближение и адиабатические инварианты.
    \Ticket Автоколебания. Маятниковые часы, ламповый генератор. Предельный цикл. Релаксационные колебания.
    \Ticket Собственные колебания системы с двумя степенями свободы. Парциальные и нормальные частоты в системе двух связанных линейных осцилляторов, нормальные колебания. 
    \Ticket Вынужденные колебания в системе двух связанных осцилляторов. Динамическое демпфирование. Резонансная кривая.
    \Ticket Собственные колебания системы трех связанных линейных осцилляторов (на примере одинаковых масс). Нормальные частоты. Собственные моды.
    \Ticket Общие свойства свободных колебаний в системе N связанных линейных осцилляторов. Коэффициенты распределения амплитуд, нормальные координаты. Колебания в однородных цепочках гармонических осцилляторов. Дисперсионная характеристика.
    \Ticket Колебания в цепочках гармонических осцилляторов с частицами двух сортов. Дисперсионная  характеристика.
    \Ticket Спектральное разложение периодического колебания. Ряд Фурье в действительной и комплексной форме. Примеры спектральных разложений. Колебательный контур как анализатор спектра.
    \Ticket Спектральное разложение непериодического колебания. Спектры видеоимпульса и радиоимпульса. Свойства спектрального преобразования. Соотношение неопределенностей.
    \Ticket Модулированные колебания, виды модуляции. Амплитудная модуляция, демодуляция АМ колебаний. Преобразование частоты, комбинационные частоты.
    \Ticket Плоские скалярные волны, скорость распространения. Монохроматические волны, длина волны. Векторные волны, основные типы поляризации волн.
    \Ticket Стоячие волны. Интерференция двух плоских волн. Интерференция двух сферических волн.
    \Ticket Продольные упругие волны в стержнях (пластинках). Волновое уравнение. Плотность энергии в упругой волне. Вектор Умова.
    \Ticket Упругие волны в газах и жидкостях. Скорость звука. Прохождение и отражение упругой волны на границе двух сред.
    \Ticket Стоячие волны в стержнях. Собственные колебания стержня. Поперечные волны на струне. Собственные колебания струны.
    \Ticket Электромагнитные волны. Волновое уравнение. Плоские электромагнитные волны, связь между векторами электрического и магнитного поля, волновое сопротивление среды. Энергетические соотношения в плоской электромагнитной волне, вектор Пойнтинга. 
    \Ticket Плоские монохроматические электромагнитные волны. Типы поляризации. Стоячие электромагнитные волны.
    \Ticket Отражение и преломление плоских электромагнитных волн на границе раздела двух диэлектриков. Граничные условия. Законы Снеллиуса. Формулы Френеля для интенсивности отраженной и преломленной волны. Явление Брюстера. Коэффициенты отражения и прохождения при нормальном падении.
    \Ticket Электромагнитные волны в анизотропных средах. Распространение волн в одноосных кристаллах, двулучепреломление. Распространение волн в гиротропных средах, вращение плоскости поляризации.
    \Ticket Элементарная теория дисперсии электромагнитных волн. Распространение волновых пакетов в диспергирующих средах. Групповая скорость. Частотная модуляция. Мгновенная частота. Распространение импульсов в среде с квадратичной дисперсией. 
    \Ticket Излучение точечного диполя. Диаграмма направленности. Сопротивление излучения. 
    \Ticket Классическая модель излучающего атома. Излучение естественных источников света. Временной и пространственный масштаб когерентности. Условия наблюдения интерференции света от естественных источников. Интерферометр Майкельсона. 
    \Ticket Излучение антенны, состоящей из двух параллельных диполей: а) с одинаковой фазой колебаний; б) с разностью фаз $\pi$/2 при расстоянии между диполями, равном четверти длины волны. Излучение одномерной решетки из диполей. Диаграммы направленности.
    \Ticket Дифракция света. Принцип Гюйгенса – Френеля. Зоны Френеля. Фокусировка при помощи зонной пластинки, фазовой зонной пластинки. Идеальная линза. Дифракционные ограничения на размеры области фокусировки.
    \Ticket Метод Френеля решения дифракционных задач. Спираль Френеля. Зоны дифракции Френеля и Фраунгофера. Интеграл Гюйгенса-Френеля. Дифракция Френеля от прямолинейного края полуплоскости. Дифракция Френеля от круглого отверстия, от прямоугольного отверстия, от круглого непрозрачного экрана.
    \Ticket Дифракция Фраунгофера на щели, на круглом отверстии, на прямоугольном отверстии. Дифракционная решетка, ее характеристики как спектрального прибора.
\end{enumerate}
\end{document}


