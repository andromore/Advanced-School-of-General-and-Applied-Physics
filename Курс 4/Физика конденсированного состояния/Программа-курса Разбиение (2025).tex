\documentclass[a4paper]{article}

\usepackage{mathtext} % использование кириллицы в формулах
\usepackage{cmap} % грамотное копирование кириллицы из pdf
\usepackage[T2A]{fontenc} % внутрення кодировка
\usepackage[utf8]{inputenc} % кодировка документа
\usepackage[russian]{babel} % язык документа
\usepackage{amssymb} % дополнительные символы
\usepackage{amsfonts} % математические шрифты
\usepackage{amsmath} % дополнительная математика
\usepackage{indentfirst} % отступ и у первого абзаца
\usepackage{tikz}
\usepackage{tikz-3dplot}
\usepackage[hidelinks]{hyperref} % ссылки

\title{Учебная программа по курсу\\
<<Физика конденсированного состояния>>}

\begin{document}

\maketitle

\begin{enumerate}
    \item Введение в электронную теорию твёрдого тела. Модель свободных
    электронов
    \begin{enumerate}
        \item Определение и общие свойства конденсированного состояния вещества
        \item Классификация твёрдых тел по их структуре и электрофизическим
        свойствам
        \item Электро- и теплопроводность твёрдых тел
        \begin{enumerate}
            \item Закон Ома
            \item Температурная зависимость удельного электросопротивления
            \item Закон Фурье
            \item Коэффициент теплопроводности
            \item Закон Видемана-Франца для металлов
        \end{enumerate}
        \item Модель свободных электронов Друде
        \begin{enumerate}
            \item Электроны проводимости как
            идеальный газ классических частиц
            \item Электрон-ионные столкновения, сечение
            рассеяния и среднее время свободного пробега электронов
            \item Расчёт
            статической электропроводности и коэффициента теплопроводности
            металлов в рамках элементарной кинетической теории разреженных газов
        \end{enumerate}
        \item Понятие о термоэлектрических эффектах
        \begin{enumerate}
            \item Эффект Зеебека
            \item Эффект Пельтье
        \end{enumerate}
        \item Квантовая теория свободных электронов
        \begin{enumerate}
            \item Принцип неразличимости
            тождественных частиц, фермионы и бозоны
            \item Принцип запрета Паули, спин
            электрона, структура волновой функции газа невзаимодействующих
            электронов
            \item Модель свободных электронов Зоммерфельда, периодические
            граничные условия
            \item Свойства идеального газа свободных электронов в
            основном состоянии
            \begin{enumerate}
                \item Импульс и энергия Ферми
                \item Плотность разрешённых волновых векторов и плотность одноэлектронных уровней энергии
                \item Энергия основного состояния электронного газа
            \end{enumerate}
            \item Распределение Ферми-Дирака
            \item Термодинамические свойства газа свободных электронов
            \begin{enumerate}
                \item Температура Ферми
                \item Условие вырождения электронного газа
                \item Температурное разложение Зоммерфельда
                \item Расчёт удельной теплоёмкости вырожденного электронного газа
            \end{enumerate}
        \end{enumerate}
        \item Теория металлической проводимости Зоммерфельда
        \begin{enumerate}
            \item Квазиклассическое
            уравнение Больцмана для неравновесной функции распределения электронов
            и условия его применимости
            \item Общая структура интеграла столкновений с
            учётом принципа запрета Паули и принцип детального баланса
            \item Дифференциальное сечение рассеяния и транспортное время свободного
            пробега электронов
            \item Расчёт статической электропроводности, коэффициента
            теплопроводности и дифференциальной термо-э.д.с. металлов в рамках
            квазиклассического уравнения Больцмана в случае упругих электрон-ионных
            столкновений
            \item Особенности измерения дифференциальной термо-э.д.с.
            металлов, биметаллический контур, понятие об электрохимическом
            потенциале
            \item Число Лоренца и закон Видемана-Франца в рамках
            зоммерфельдовской теории свободных электронов
        \end{enumerate}
        \item Магнетизм электронного газа
        \begin{enumerate}
            \item Определение собственного и орбитального магнитного моментов электрона
            \item Определение магнитной восприимчивости
        \end{enumerate}
        \item Диамагнетизм и парамагнетизм. Теорема Бора -- Ван Леевен о нулевой
        магнитной восприимчивости газа классических заряженных частиц.
        \item Магнитная восприимчивость больцмановского газа электронов с учётом их
        собственного магнитного момента
        \begin{enumerate}
            \item Закон Кюри
            \item Трудности классических теорий электронного газа
        \end{enumerate}
        \item Магнитная восприимчивость вырожденного электронного газа
        \begin{enumerate}
            \item Парамагнетизм Паули
            \item Понятие о диамагнетизме Ландау
            \item Недостатки модели свободных электронов
        \end{enumerate}
    \end{enumerate}
    \item Свойства кристаллических решеток
    \begin{enumerate}
        \item Трансляционная симметрия кристаллов
        \begin{enumerate}
            \item Решётка Бравэ и основные векторы трансляций
            \item Простая, объёмно-центрированная и гранецентрированная кубические решётки
            \item Примитивная (элементарная) ячейка, ячейка Вигнера-Зейтца и условная ячейка
            \item Кристаллические структуры и решётки с базисом
        \end{enumerate}
        \item Гексагональная плотноупакованная структура
        \begin{enumerate}
            \item Координационное число и коэффициент компактности (упаковочный множитель)
            \item Алгоритм построения различных плотноупакованных структур
            \item Структуры типа хлорида натрия, алмаза и цинковой обманки
        \end{enumerate}
        \item Понятие об элементах симметрии кристаллических решёток и группе
        симметрии (пространственной группе) решётки Бравэ
        \begin{enumerate}
            \item Поворотные оси симметрии
            \item Теорема о симметрии кристаллических решёток по отношению к поворотам
        \end{enumerate}
        \item Прямая и обратная решётки
        \begin{enumerate}
            \item Свойства обратной решётки
            \item Примеры обратных решёток
            \item Первая зона Бриллюэна
            \item Атомные плоскости и семейства атомных плоскостей
            \item Теорема о связи семейств атомных плоскостей с векторами обратной решётки
            \item Индексы Миллера
        \end{enumerate}
        \item Условия конструктивной интерференции рентгеновских лучей в кристалле
        \begin{enumerate}
            \item Формулировка Брэгга
            \item Формулировка Лауэ
            \item Доказательство эквивалентности формулировок
        \end{enumerate}
        \item Геометрические формулировки условий конструктивной интерференции
        \begin{enumerate}
            \item Построение Бриллюэна
            \item Брэгговские плоскости
            \item Построение Эвальда
        \end{enumerate}
        \item Экспериментальные методы определения кристаллических структур
        \begin{enumerate}
            \item Метод Лауэ
            \item Метод вращающегося кристалла
            \item Порошковый метод (метод Дебая-Шеррера)
        \end{enumerate}
        \item Геометрический структурный фактор кристаллических структур
    \end{enumerate}
    \item Состояния электронов в кристаллической решётке. Основы зонной
    теории твёрдых тел
    \begin{enumerate}
        \item Квантовые состояния электрона в периодическом потенциале
        \begin{enumerate}
            \item Теорема Блоха
            \item Квазиимпульс электрона и его свойства
            \item Граничные условия Борна-Кармана
            для кристаллов и число разрешённых значений квазиимпульса
            \item Периодичность волновых функций и энергетического спектра в обратном
            пространстве
            \item Энергетические зоны и их свойства
            \item Запрещённые зоны энергий,
            описание электронных состояний в схемах приведённых и повторяющихся зон
        \end{enumerate}
        \item Понятие о <<$\vec{k}\vec{p}$>> методе
        \item Теорема о средней скорости блоховского электрона
        \item Свойства энергетического спектра вблизи экстремумов энергии в зоне Бриллюэна
        \item Тензор обратных эффективных масс электрона и его свойства
        \item Отсутствие вклада в электрический ток от
        полностью заполненных зон (инертность заполненных зон), критерии
        металла и диэлектрика
        \begin{enumerate}
            \item Диэлектрики
            \item Полупроводники
            \item Металлы
            \item Полуметаллы
        \end{enumerate}
    \end{enumerate}
\end{enumerate}

\end{document}