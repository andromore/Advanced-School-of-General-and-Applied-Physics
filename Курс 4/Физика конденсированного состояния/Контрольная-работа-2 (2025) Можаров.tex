\documentclass[a4paper]{article}

\usepackage{mathtext} % использование кириллицы в формулах
\usepackage{cmap} % грамотное копирование кириллицы из pdf
\usepackage[T2A]{fontenc} % внутрення кодировка
\usepackage[utf8]{inputenc} % кодировка документа
\usepackage[russian]{babel} % язык документа
\usepackage{amssymb} % дополнительные символы
\usepackage{amsfonts} % математические шрифты
\usepackage{amsmath} % дополнительная математика
\usepackage{indentfirst} % отступ и у первого абзаца
\usepackage{tikz}
\usepackage{tikz-3dplot}
\usepackage[hidelinks]{hyperref} % ссылки

\title{Контрольная работа 2 по физике конденсированного состояния}
\author{Можаров Андрей}

\begin{document}

\maketitle

\section*{\centering Задача 1}

\subsection*{Условие}

Пользуясь кинетическим уравнением Больцмана в приближении упругого рассеяния
электронов на сферически симметричных дефектах, рассчитайте ток, вызываемый в
металле зависящим от времени однородным электрическим полем $\vec{E}(t)$.
Получите явное выражение для частотно-зависящей проводимости $\sigma(\omega)$
в случае периодически зависящего от времени поля
$\vec{E}(t) = Re(\vec{E}(\omega) \cdot e^{-i\omega t})$.

\subsection*{Решение}

Запишем кинетическое уравнение Больцмана в общем виде:
$$
    St\{f\} = \dfrac{\partial f}{\partial t} + (\vec{v}, grad_{\vec{r}}\ f) + (\vec{F}, grad_{\vec{p}}\ f)
$$
Для его решения, рассмотрим линеаризуем функцию распределения
$$
    f = f_0 + f_1
$$
где $f_0 = f_0(\varepsilon(\vec{p}))$ --- равновесная функция распределения.
В приближении упругого рассеяния имеем $|\vec{p}_i| = |\vec{p}_f|$
(следствие закона сохранения энергии), откуда следует
$$
    \varepsilon(\vec{p}_i) = \varepsilon(\vec{p}_f) \hspace{2em}
    f_0(\varepsilon(\vec{p}_i)) = f_0(\varepsilon(\vec{p}_f))
$$
Тогда рассмотрев общий вид интеграла столкновений, получим:
$$
    St\{f\} = \int W(\vec{p}_i, \vec{p}_f) (f(\vec{p}_f) - f(\vec{p}_i))\dfrac{d^3\vec{p}_f}{(2\pi\hbar)^3} = \int W(\vec{p}_i, \vec{p}_f) (f_1(\vec{p}_f) - f_1(\vec{p}_i))\dfrac{d^3\vec{p}_f}{(2\pi\hbar)^3}
$$
где $W(\vec{p}_i, \vec{p}_f)$ --- вероятность перехода из состояния с импульсом $\vec{p}_i$
в состояние с импульсом $\vec{p}_f$.

Ввиду сферической симметрии дефектов, а также упругости рассеяния,
имеем
$$
    W(\vec{p}_i, \vec{p}_f) = W(\varepsilon, |\theta_f - \theta_i|)
$$
где углы $\theta_i$ и $\theta_f$ --- углы между осью $Oz$,
выбранной по направлению поля, и соответствующим импульсом.
Тогда также получим, что $f_1(\vec{p_f}) = f_1(\vec{p_i})cos(\theta_f - \theta_i)$,
тогда получим интеграл столкновений в виде
$$
    St\{f\} = - f_1(\vec{p}_i)\int W(\varepsilon, |\theta_f - \theta_i|)(1 - cos(\theta_f - \theta_i)) \dfrac{d^3\vec{p}_f}{(2\pi\hbar)^3} = - \dfrac{f_1}{\tau_{tr}}
$$
где $\tau_{tr}$ --- транспортное время релаксации (далее просто $\tau$).

Используя полученные результаты, запишем кинетическое уравнение в случае данной задачи
в однородном поле:
$$
    \dfrac{\partial f_1}{\partial t} + e (\vec{E}(t), \vec{v})\dfrac{\partial f_0}{\partial \varepsilon} = - \dfrac{\partial f_1}{\partial \tau} 
$$
Тогда, переходя к комплексным амплитудам для поля и функции распределения, получим
$$
    f_1(\omega) = - \dfrac{e(\vec{E}(\omega), \vec{v})\tau}{1 - i\omega\tau}\dfrac{\partial f_0}{\partial \varepsilon}
$$
Тогда плотность тока $\vec{j}(\omega)$:
$$
    \vec{j}(\omega) = \int e\vec{v}f_1(\omega)\dfrac{2d^3\vec{p}}{(2\pi\hbar)^3}
$$
Подставляя, переходя к сферическим координатам, интегрируя по углам,
после чего переходя от интегрирования по модулю импульса к интегрированию по энергии
и учитывая
$$
    -\dfrac{\partial f_0}{\partial \varepsilon} \approx \delta(\varepsilon - \varepsilon_F) \hspace{1em} g(\varepsilon_F) = \dfrac{3n}{m v_F^2}
$$
получим из $\vec{j}(\omega) = \sigma(\omega)\vec{E}(\omega)$
$$
    \sigma(\omega) = \dfrac{ne^2\tau}{m(1 - i\omega\tau)} = \dfrac{\sigma_0}{1 - i\omega\tau}
$$
где $\sigma_0 = ne^2\tau/m$ --- статическая удельная электропроводимость.

\subsection*{Ответ}

$$
    \sigma(\omega) = \dfrac{\sigma_0}{1 - i\omega\tau}
$$
(Также, фактически, решение данной задачи может быть найдено в книге <<Физика твёрдого тела>> (Том 1)
за авторством Ашкрофта и Мермина \mbox{на стр. 31-32};
Конкретно ответ данной задачи даётся выражением (1.29))

\section*{\centering Задача 2}

\subsection*{Условие}

Считая, что источником электрического поля в металле является неравновесная
электронная плотность и, используя локальный закон Ома
в высокочастотном пределе ($\omega >> 1/\tau_{tr}$),
$$
    \vec{j}(\omega, \vec{r}) = \sigma(\omega) \cdot \vec{E}(\omega, \vec{r})
$$
где $\sigma(\omega)$ --- частотно-зависящая проводимость,
вычисленная в предыдущей задаче, найдите частоту колебаний неравновесной
электронной плотности $n_e(t, \vec{r})$.
Оцените величину этой частоты, используя типичные
значения электронной плотности в металлах.

\subsection*{Решение}

Для решения данной задачи используем следующую ниже систему уравнений:
\begin{equation*}
    \begin{cases}
        \dfrac{\partial \rho}{\partial t} + div\ \vec{j} = 0 \hspace{1.8em} \text{ (Уравнение непрерывности)} \\
        div\ \vec{E} = 4\pi\rho \hspace{3em} \text{ (Закон Гаусса/одно из уравнений Максвелла)} \\
        \vec{j}(\omega) = \sigma(\omega) \vec{E}(\omega) \hspace{1em} \text{ (Закон Ома/материальное соотношение)}
    \end{cases}
\end{equation*}
Перейдя к Фурье-компонентам и подставив закон Ома в уравнение непрерывности, получим
$$
    -i\omega\rho + \sigma(\omega) div\ \vec{E} = 0
$$
Подставив закон Гаусса, получим
$$
    -i\omega\rho + 4\pi\sigma(\omega)\rho = 0 \hspace{1em} \sigma(\omega) = \dfrac{i\omega}{4\pi}
$$
Используем выражение для $\sigma(\omega)$, вычисленное в предыдущей задаче,
учтя высокочастотный предел $\omega >> 1/\tau$ ($\omega\tau >> 1$):
$$
    \sigma(\omega) = \dfrac{ine^2}{m\omega}
$$
Приравняв два выражения для $\sigma(\omega)$, получим:
$$
    \omega = \sqrt{\dfrac{4\pi ne^2}{m}}
$$

Сделаем численную оценку для $\omega$, взяв следующие значения:
\begin{description}
    \item[Концентрация электронов в металлах] $n \approx 10^{22} - 10^{23} \text{ см}^{-3}$
    \item[Элементарный заряд] $e \approx 4,8 \cdot 10^{-10} \text{ ед. зар. СГСЭ}$
    \item[Масса электрона] $m \approx 9,1 \cdot 10^{-28} \text{ г}$ 
\end{description}
Для данных значений имеем $\omega \approx 5,64 \cdot 10^{15} - 1,78 \cdot 10^{16} \text{ с}^{-1}$.

\subsection*{Ответ}

$$
    \omega = \sqrt{\dfrac{4\pi ne^2}{m}} \approx 5,64 \cdot 10^{15} - 1,78 \cdot 10^{16} \text{ с}^{-1}
$$
(Также, фактически, решение данной задачи может быть найдено в книге <<Физика твёрдого тела>> (Том 1)
за авторством Ашкрофта и Мермина \mbox{на стр. 32-33}; конкретно,
формула из ответа данной задачи получается извлечением квадратного корня
из выражения (1.38))

\section*{\centering Задача 3}

\subsection*{Условие}

Используя стационарное уравнение Больцмана в приближении времени релаксации
(т.е. $St\{f\} = - (f - f_0(\varepsilon)) / \tau$,
где $f_0(\varepsilon)$ --- равновесная функция распределения, $\tau$ --- время
релаксации, которое будем считать независящим от энергии $\varepsilon$) для электронов в
однородном электрическом поле $\vec{E}$, найдите нелинейные по напряжённости поля
поправки к закону Ома: $\delta j(E) = j(E) - \sigma \cdot E = \alpha_2 \cdot E^2 + \alpha_3 \cdot E^3 + \ldots$, здесь $j$ --- плотность
электрического тока, $\sigma$ --- удельная проводимость.

\subsection*{Решение}

Запишем стационарное кинетическое уравнение Больцмана в общем виде
$$
    St\{f\} = \dfrac{\partial f}{\partial t} + (\vec{v}, grad_{\vec{r}}\ f) + (\vec{F}, grad_{\vec{p}}\ f)
$$
которое в приближении времени релаксации для электронов в электрическом поле $\vec{E}$ примет вид
$$
    \dfrac{f - f_0(\varepsilon)}{\tau} = e \left(\vec{E}, grad_{\vec{p}}\ f\right)
$$
или, выражая неравновесную функцию распределения $f$
$$
    f = f_0(\varepsilon) + e\tau (\vec{E}, grad_{\vec{p}}f)
$$
Считая $E = |\vec{E}|$ малым параметром,
будем данное уравнение методом последовательных приближений
$$
    f(\vec{p}) = f_0(\varepsilon(\vec{p})) + \sum_{n=1}^{\infty} f_n(\vec{p})
$$
где $f_n(\vec{p}) \sim E^n$. Тогда, приравнивая равные порядки малости,
получим (номер указывает порядок малости)
\begin{enumerate}
    \item $f_1 = e\tau(\vec{E}, grad_{\vec{p}}f_0)$
    \item $f_2 = e\tau(\vec{E}, grad_{\vec{p}}f_1) = (e\tau)^2(\vec{E}, \nabla_{\vec{p}})^2f_0$
    \item $f_3 = e\tau(\vec{E}, grad_{\vec{p}}f_2) = (e\tau)^3(\vec{E}, \nabla_{\vec{p}})^3f_0$
\end{enumerate}
что в общем случае принимает вид
$$
    f_n = e\tau(\vec{E}, grad_{\vec{p}})f_{n-1} = (e\tau)^n(\vec{E}, \nabla_{\vec{p}})^n f_0
$$

Рассмотрим плотность тока, которая в общем случае,
определяется нижеследующим интегралом
$$
    \vec{j} = - e \int \vec{v}(\vec{p}) f(\vec{p})\dfrac{2d^3\vec{p}}{(2\pi\hbar)^3}
$$
тогда можно перейти от вектора плотности тока $\vec{j}$ к рассмотрению его модуля $j = |\vec{j}|$
и определить поправки $j_n$ по соответствующим порядкам малости (выразив их,
соответственно, $j_n$ через $f_n$)
$$
    j = \sum_n j_n, \hspace{1em} \text{ где } j_n = - \dfrac{e}{m}(e\tau)^n\int (\vec{E}, \nabla_{\vec{p}})^n f_0 \dfrac{2pd^3\vec{p}}{(2\pi\hbar)^3}
$$
Тогда $j_0 = 0$ (ток в отсутствие внешнего поля),
а линейный член $j_1 = \sigma E$ (проводимость Друде).

Квадратичный по полю член $j_2 = 0$ (и все чётные поправки по полю, аналогично),
т.к. подынтегральное выражение является нечётной функцией по импульсу,
а интеграл берётся в симметричных пределах.

Вычислим $j_3$ <<по честному>>. Для этого найдём $f_3$
(однако возьмём только члены, нечётные по импульсу, т.е. не зануляющиеся при интегрировании):
\begin{gather*}
    \nabla_{\vec{p}} = \dfrac{\partial}{\partial \vec{p}} = \dfrac{\partial \varepsilon}{\partial \vec{p}} \dfrac{\partial}{\partial \varepsilon} = \dfrac{\vec{p}}{m} \dfrac{\partial}{\partial \varepsilon} \\
    f_1 = e\tau \dfrac{(\vec{E}, \vec{p})}{m} \dfrac{\partial f_0}{\partial \varepsilon} \\
    f_2 = (e\tau E)^2 \left(\dfrac{1}{m} \dfrac{\partial f_0}{\partial \varepsilon} + \dfrac{p^2}{m^2} \dfrac{\partial^2 f_0}{\partial \varepsilon^2}\right) \\
    f_3 = (e\tau)^3 E^2 \left(\dfrac{3(\vec{E}, \vec{p})}{m^2}\dfrac{\partial^2 f_0}{\partial \varepsilon^2} + \dfrac{(\vec{E}, \vec{p})p^2}{m^3}\dfrac{\partial^3 f_0}{\partial \varepsilon^3}\right)
\end{gather*}
Использовав правила усреднения по углам и переходя к интегрированию по энергии, получим
$$
    j_3 = -\frac{e^4 \tau^3 E^3}{m^2} \int_0^\infty \left[ 2\varepsilon \frac{\partial^2 f_0}{\partial \varepsilon^2} + \frac{4}{5} \varepsilon^2 \frac{\partial^3 f_0}{\partial \varepsilon^3} \right] g(\varepsilon) d\varepsilon
$$
Подавляя $f_0$, получим
$$
    j_3 = -\frac{e^4 \tau^3 E^3}{m^2} \int_0^\infty \left[ \frac{2\varepsilon}{(kT)^2} - \frac{4\varepsilon^2}{5(kT)^3} \right] f_0 g(\varepsilon) d\varepsilon
$$
Вспомним, что
$$
    n = \int f_0 g(\varepsilon) d\varepsilon \hspace{1em} \langle \varepsilon \rangle = \frac{1}{n} \int \varepsilon f_0 g(\varepsilon) d\varepsilon = \frac{3}{2} kT \hspace{1em} \langle \varepsilon^2 \rangle = \frac{15}{4} (kT)^2
$$
Подставляем средние значения:
\begin{gather*}
    j_3 = -\frac{e^4 \tau^3 E^3 n}{m^2} \left[ \frac{2 \langle \varepsilon \rangle}{(kT)^2} - \frac{4 \langle \varepsilon^2 \rangle}{5(kT)^3} \right] = n \left[ \frac{2 \cdot \frac{3}{2} kT}{(kT)^2} - \frac{4 \cdot \frac{15}{4} (kT)^2}{5(kT)^3} \right] \\
    j_3 = -\frac{e^4 \tau^3 E^3 n}{m^2} \left[ \frac{3}{kT} - \frac{3}{kT} \right] = 0
\end{gather*}

\subsection*{Ответ}

При данных приближениях
$$
    \delta j(E) = 0
$$
что будет неверно в общем случае при $\tau \not\equiv const$ и при больших импульсах.

\section*{\centering Задача 4}

\subsection*{Условие}

Используя кинетическое уравнение Больцмана для свободных электронов в металле,
найдите дифференциальную термо-э.д.с. металла в области температур $kT << \varepsilon_F$ в
случаях, когда транспортное время свободного пробега электронов равно:
\begin{enumerate}
    \item 
    $\tau_{tr}(\varepsilon) = \tau \equiv const$
    \item
    $\tau_{tr}(\varepsilon) = 1 / (v(\varepsilon)\sigma_{tr}n_i)$, где $n_i$ --- концентрация рассеивателей,
    $v(\varepsilon) = \sqrt{2\varepsilon/m}$ --- скорость электрона с энергией,
    $\sigma_{tr} = \pi a^2$ --- полное транспортное сечение рассеяния на твёрдой сфере радиуса.
\end{enumerate}

\newcommand{\bracets}[1]{\langle #1\rangle}

\subsection*{Решение}

Выражение для дифференциальной термо-ЭДС из лекций имеет следующий вид:
$$
    Q' = \dfrac{1}{eT}\dfrac{\bracets{\varepsilon} - \mu\bracets{1}}{\bracets{1}}
$$
где оператор усреднения $\bracets{F(\varepsilon)}$ определяется следующим образом
$$
    \bracets{F(\varepsilon)} = - \int_0^\infty K_0(\varepsilon) \dfrac{\partial f_0}{\partial \varepsilon} F(\varepsilon) \hspace{1em} K_0(\varepsilon) = \dfrac{1}{3} g(\varepsilon)v^2(\varepsilon)\tau_{tr}(\varepsilon)
$$

Применим разложение Зоммерфельда, приведённого в общем случае ниже,
для вычисления интегралов $\bracets{1}$ и $\bracets{\varepsilon}$.
$$
    - \int_0^\infty F(\varepsilon) \dfrac{\partial f_0}{\partial \varepsilon} d\varepsilon \approx F(\mu) + \left.\dfrac{\pi^2}{6}(kT)^2\dfrac{d^2F}{d\varepsilon^2}\right|_{\varepsilon=\mu}
$$
Тогда
$$
    \bracets{1} = - \int_0^\infty K_0(\varepsilon) \dfrac{\partial f_0}{\partial \varepsilon} \approx K_0(\mu) + \left.\dfrac{(\pi kT)^2}{6} \dfrac{d^2K_0(\varepsilon)}{d\varepsilon^2}\right|_{\varepsilon = \mu}
$$
$$
    \bracets{\varepsilon} = - \int_0^\infty \varepsilon K_0(\varepsilon) \dfrac{\partial f_0}{\partial \varepsilon} \approx \mu K_0(\mu) + \left.\dfrac{(\pi kT)^2}{6} \dfrac{d^2}{d\varepsilon^2}\left(\varepsilon K_0(\varepsilon)\right)\right|_{\varepsilon = \mu}
$$
И применим это к искомым случаям
\begin{enumerate}
    \item $\tau_{tr} = \tau \equiv const$:
    
    Вычисляем $K_0(\varepsilon)$:
    \begin{gather*}
        K_0(\varepsilon) = g(\varepsilon) \dfrac{v^2(\varepsilon)}{3} \tau_{tr}(\varepsilon) = \dfrac{3}{2} \dfrac{n}{\varepsilon_F}\sqrt{\dfrac{\varepsilon}{\varepsilon_F}} \cdot \dfrac{2\varepsilon}{3m} \cdot \tau = \dfrac{n\tau}{m} \left(\dfrac{\varepsilon}{\varepsilon_F}\right)^{3/2} := \alpha \varepsilon^{3/2}
    \end{gather*}
    Вычисляем производные:
    \begin{gather*}
        \dfrac{d}{d\varepsilon}K_0(\varepsilon) = \dfrac{3}{2} \alpha\varepsilon^{1/2} \hspace{2em}
        \dfrac{d^2}{d\varepsilon^2}K_0(\varepsilon) = \dfrac{3}{4} \alpha\varepsilon^{-1/2} \\
        \dfrac{d^2}{d\varepsilon^2} \varepsilon K_0(\varepsilon) = \varepsilon \dfrac{d^2}{d\varepsilon^2}K_0(\varepsilon) + 2 \dfrac{d}{d\varepsilon}K_0(\varepsilon) = \dfrac{15}{4}\alpha\varepsilon^{1/2}
    \end{gather*}
    Вычисляем усреднённые величины:
    \begin{gather*}
        \bracets{1} = \alpha\mu^{3/2} + \dfrac{(\pi kT)^2}{8} \alpha \mu^{-1/2} \hspace{2em} \bracets{\varepsilon} = \alpha \mu^{5/2} + \dfrac{5}{8} (\pi kT)^2 \alpha \mu^{1/2}
    \end{gather*}
    Вычисляем термо-ЭДС:
    \begin{gather*}
        Q' = \dfrac{1}{eT} \dfrac{\dfrac{(\pi kT)^2}{2}\alpha\mu^{1/2}}{\alpha\mu^{3/2} + \dfrac{(\pi kT)^2}{8} \alpha \mu^{-1/2}} = \dfrac{1}{eT} \dfrac{\mu}{\dfrac{2\mu^{2}}{(\pi kT)^2} + \dfrac{1}{4}} \approx \dfrac{\pi^2}{2} \dfrac{k^2T}{e\varepsilon_F}
    \end{gather*}
    \item $\tau_{tr}(\varepsilon) = 1 / (v(\varepsilon)\sigma_{tr}n_i)$:
    
    Вычисляем $K_0(\varepsilon)$:
    \begin{gather*}
        K_0(\varepsilon) = g(\varepsilon) \dfrac{v^2(\varepsilon)}{3} \tau_{tr}(\varepsilon) = \dfrac{3}{2} \dfrac{n}{\varepsilon_F}\sqrt{\dfrac{\varepsilon}{\varepsilon_F}} \cdot \dfrac{2\varepsilon}{3m} \cdot \dfrac{1}{\sqrt{\dfrac{2\varepsilon}{m}}\pi a^2 n_i} = \\
        = \dfrac{\sqrt{2}}{2} \dfrac{n \varepsilon}{n_i \varepsilon^{3/2}_F \sqrt{m} \pi a^2} := \alpha \varepsilon
    \end{gather*}
    Вычисляем производные:
    \begin{gather*}
        \dfrac{d}{d\varepsilon} K_0(\varepsilon) = \alpha \hspace{2em}
        \dfrac{d^2}{d\varepsilon^2} K_0(\varepsilon) = 0 \hspace{2em} \dfrac{d^2}{d\varepsilon^2} \varepsilon K_0(\varepsilon) = 2\alpha
    \end{gather*}
    Вычисляем усреднённые величины:
    \begin{gather*}
        \bracets{1} = \alpha \mu \hspace{2em} \bracets{\varepsilon} = \alpha \mu^2 + \dfrac{(\pi kT)^2}{3} \alpha
    \end{gather*}
    Вычисляем термо-ЭДС:
    \begin{gather*}
        Q' = \dfrac{1}{eT} \dfrac{\dfrac{(\pi kT)^2}{3}\alpha}{\alpha \mu} = \dfrac{(\pi kT)^2}{3eT\mu} \approx \dfrac{\pi^2}{3} \dfrac{k^2T}{e\varepsilon_F}
    \end{gather*}
\end{enumerate}

\subsection*{Ответ}

$$
    Q_1' \approx \dfrac{\pi^2}{2} \dfrac{k^2T}{e\varepsilon_F} \hspace{5em} Q_2' \approx \dfrac{\pi^2}{3} \dfrac{k^2T}{e\varepsilon_F}
$$

\section*{\centering Задача 5}

\subsection*{Условие}

Найти коэффициенты компактности (относительный объём идентичных
соприкасающихся шаров с центрами в узлах решётки) для г.ц.к. и о.ц.к. решёток Бравэ
и структуры типа алмаза.

\subsection*{Решение}

Расчётная формула коэффициента компактности,
который показывает какую долю объёма ячейки занимают атомы,
имеет следующий вид
$$
    f = \dfrac{N \cdot V_\text{шара}}{V_\text{ячейки}}
$$
где $f$ --- коэффициент компактности,
$N$ --- количество атомов в одной ячейке,
$V_\text{шара}$ --- объём одного шара,
$V_\text{ячейки}$ --- объём одной ячейки.

\begin{description}
    \item[Гранецентрированная кубическая решётка] (приведена на рис. \ref{ГЦК})
    
    \begin{figure}[h]
        \centering
        \begin{tikzpicture}[scale=3, every node/.style={shading=ball, ball color=blue!60}]
            % Вершины нижнего основания
            \node (A) at (0,0,0) [circle, minimum size=10pt] {};
            \node (B) at (1,0,0) [circle, minimum size=10pt] {};
            \node (C) at (1,0,1) [circle, minimum size=10pt] {};
            \node (D) at (0,0,1) [circle, minimum size=10pt] {};
            
            % Центры граней (нижняя, боковые, передняя)
            \node at (0.5,0,0.5) [circle, minimum size=10pt, ball color=red!70] {}; % Нижняя
            \node at (1,0.5,0.5) [circle, minimum size=10pt, ball color=red!70] {}; % Правая
            \node at (0,0.5,0.5) [circle, minimum size=10pt, ball color=red!70] {}; % Левая
            
            % Вершины верхнего основания
            \node (E) at (0,1,0) [circle, minimum size=10pt] {};
            \node (F) at (1,1,0) [circle, minimum size=10pt] {};
            \node (G) at (1,1,1) [circle, minimum size=10pt] {};
            \node (H) at (0,1,1) [circle, minimum size=10pt] {};
            
            % Центры граней (верхняя, задняя, передняя)
            \node at (0.5,1,0.5) [circle, minimum size=10pt, ball color=red!70] {}; % Верхняя
            \node at (0.5,0.5,0) [circle, minimum size=10pt, ball color=red!70] {}; % Задняя
            \node at (0.5,0.5,1) [circle, minimum size=10pt, ball color=red!70] {}; % Передняя
            
            % Ребра куба для наглядности
            \draw[thick, opacity=0.3] (0,0,0) -- (1,0,0) -- (1,0,1) -- (0,0,1) -- cycle;
            \draw[thick, opacity=0.3] (0,1,0) -- (1,1,0) -- (1,1,1) -- (0,1,1) -- cycle;
            \draw[thick, opacity=0.3] (0,0,0) -- (0,1,0) (1,0,0) -- (1,1,0) (1,0,1) -- (1,1,1) (0,0,1) -- (0,1,1);
        \end{tikzpicture}
        \caption{Элементарная ячейка г. ц. к. решётки}
        \label{ГЦК}
    \end{figure}

    Здесь количество атомов в одной ячейке будет равным
    $$
        N = 8 \cdot \dfrac{1}{8} + 6 \cdot \dfrac{1}{2} = 4
    $$
    Связь радиуса шара $R$ с периодом решётки $a$:
    можно выразить через диагональ грани $a\sqrt{2} = 4R$.
    Тогда
    $$
        f = \dfrac{4 \cdot \dfrac{4}{3}\pi\left(\dfrac{a\sqrt{2}}{4}\right)^3}{a^3} = \dfrac{\pi\sqrt{2}}{6} \approx 0.74
    $$

    \item[Объемноцентрированная кубическая решётка] (приведена на рис. \ref{ОЦК})
    
    \begin{figure}[h]
        \centering
        \begin{tikzpicture}[scale=3, every node/.style={shading=ball, ball color=blue!60}]
            % Вершины
            \foreach \x in {0,1} \foreach \y in {0,1} \foreach \z in {0,1}
            \node at (\x,\y,\z) [circle, minimum size=10pt] {};
            
            % Центральный атом
            \node at (0.5,0.5,0.5) [circle, minimum size=10pt, ball color=green!70] {};
            
            % Ребра куба
            \draw[thick, opacity=0.3] (0,0,0) -- (1,0,0) -- (1,0,1) -- (0,0,1) -- cycle;
            \draw[thick, opacity=0.3] (0,1,0) -- (1,1,0) -- (1,1,1) -- (0,1,1) -- cycle;
            \draw[thick, opacity=0.3] (0,0,0) -- (0,1,0) (1,0,0) -- (1,1,0) (1,0,1) -- (1,1,1) (0,0,1) -- (0,1,1);
        \end{tikzpicture}
        \caption{Элементарная ячейка о. ц. к. решётки}
        \label{ОЦК}
    \end{figure}

    Здесь количество атомов в одной ячейке будет равным
    $$
        N = 8 \cdot \dfrac{1}{8} + 1 = 2
    $$
    Связь радиуса шара $R$ с периодом решётки $a$:
    можно выразить через диагональ куба $a\sqrt{3} = 4R$.
    Тогда
    $$
        f = \dfrac{2 \cdot \dfrac{4}{3}\pi\left(\dfrac{a\sqrt{3}}{4}\right)^3}{a^3} = \dfrac{\pi\sqrt{3}}{8} \approx 0.68
    $$

    \item[Структура типа алмаза] (приведена на рис. \ref{Алмаз})
    
    \begin{figure}[h]
        \centering
        \tdplotsetmaincoords{70}{120}
        \begin{tikzpicture}[tdplot_main_coords, scale=3, line join=round, line cap=round, every node/.style={shading=ball}]
            \foreach \x/\y/\z in {0/0/0, 1/0/0, 1/1/0, 0/1/0, 0/0/1, 1/0/1, 1/1/1, 0/1/1}
                \node[circle, minimum size=10pt, ball color=blue!60] at (\x,\y,\z) {};
            \foreach \x/\y/\z in {0.5/0.5/0, 0.5/0.5/1, 0.5/0/0.5, 0.5/1/0.5, 0/0.5/0.5, 1/0.5/0.5}
                \node[circle, minimum size=10pt, ball color=red!70] at (\x,\y,\z) {};
            \node[circle, ball color=green!70] at (0.25,0.75,0.25) {};
            \node[circle, ball color=green!70] at (0.75,0.25,0.25) {};
            \node[circle, ball color=green!70] at (0.75,0.75,0.75) {};
            \node[circle, ball color=green!70] at (0.25,0.25,0.75) {};
    
            \draw[thick, color=black] (0.25,0.25,0.75) -- (0,0,1);
            \draw[thick, color=black] (0.25,0.25,0.75) -- (0.5,0,0.5);
            \draw[thick, color=black] (0.25,0.25,0.75) -- (0,0.5,0.5);
            \draw[thick, color=black] (0.25,0.25,0.75) -- (0.5,0.5,1);
            \draw[thick, color=black] (0.75,0.25,0.25) -- (0.5,0,0.5);
            \draw[thick, color=black] (0.75,0.25,0.25) -- (1,0,0);
            \draw[thick, color=black] (0.75,0.25,0.25) -- (1,0.5,0.5);
            \draw[thick, color=black] (0.75,0.25,0.25) -- (0.5,0.5,0);
            \draw[thick, color=black] (0.25,0.75,0.25) -- (0,1,0);
            \draw[thick, color=black] (0.25,0.75,0.25) -- (0,0.5,0.5);
            \draw[thick, color=black] (0.25,0.75,0.25) -- (0.5,0.5,0);
            \draw[thick, color=black] (0.25,0.75,0.25) -- (0.5,1,0.5);
            \draw[thick, color=black] (0.75,0.75,0.75) -- (1,1,1);
            \draw[thick, color=black] (0.75,0.75,0.75) -- (0.5,0.5,1);
            \draw[thick, color=black] (0.75,0.75,0.75) -- (1,0.5,0.5);
            \draw[thick, color=black] (0.75,0.75,0.75) -- (0.5,1,0.5);
    
            % 3. Контур куба (пунктиром)
            \draw[thick, opacity=0.3] (0,0,0) -- (1,0,0) -- (1,1,0) -- (0,1,0) -- cycle;
            \draw[thick, opacity=0.3] (0,0,1) -- (1,0,1) -- (1,1,1) -- (0,1,1) -- cycle;
            \draw[thick, opacity=0.3] (0,0,0) -- (0,0,1) (1,0,0) -- (1,0,1) (1,1,0) -- (1,1,1) (0,1,0) -- (0,1,1);
        \end{tikzpicture}
        \caption{Элементарная ячейка структуры типа алмаза}
        \label{Алмаз}
    \end{figure}
    
    Здесь количество атомов в одной ячейке будет равным
    $$
        N = 8 \cdot \dfrac{1}{8} + 6 \cdot \dfrac{1}{2} + 4 \cdot 1 = 8
    $$
    Рассмотрим двух <<ближайших соседей>> в точках $(0, 0, 0)$ и $(a/4, a/4, a/4)$.
    Расстояние между ними равно $2R$, но с другой стороны оно же равно $\sqrt{3 \cdot (a/4)^2} = a\sqrt{3}/4$.
    Тогда
    $$
        f = \dfrac{8 \cdot \dfrac{4}{3}\pi\left(\dfrac{a\sqrt{3}}{8}\right)^3}{a^3} = \dfrac{\pi\sqrt{3}}{16} \approx 0.34
    $$

\end{description}

\subsection*{Ответ}

$$
    f_\text{г.ц.к.} \approx 0.74 \hspace{1em} f_\text{о.ц.к.} \approx 0.68 \hspace{1em} f_\text{стр. алмаза} \approx 0.34
$$

\section*{\centering Задача 6}

\subsection*{Условие}

Докажите, что идеальное отношение $c/a$ для гексагональной плотноупакованной
структуры равно $\sqrt{8/3} \approx 1,633$. Найдите коэффициент компактности для этой
структуры.

\subsection*{Решение}

\begin{figure}[h]
    \centering
    \tdplotsetmaincoords{70}{135}
    \begin{tikzpicture}[tdplot_main_coords, scale=2, line join=round, line cap=round, 
        every node/.style={circle, minimum size=10pt, ball color=blue!60}]

        % Константы
        \pgfmathsetmacro{\a}{1}
        \pgfmathsetmacro{\c}{sqrt(8/3)}
        \pgfmathsetmacro{\rMid}{\a/sqrt(3)} % Расстояние до центра треугольной лунки

        % 1. Нижний слой A (z=0)
        \node[ball color=red!70] (center) at (0,0,0) {}; % Центр
        \foreach \ang in {0,60,120,180,240,300} {
            \node (down\ang) at ({\a*cos(\ang)}, {\a*sin(\ang)}, 0) {};
        }

        % 2. Средний слой B (z = c/2)
        % Эти три атома должны иметь другой цвет, как в вашем примере
        \foreach \ang in {30,150,270} {
            \node[ball color=green!70] (ball\ang) at ({\rMid*cos(\ang)}, {\rMid*sin(\ang)}, \c/2) {};
        }

        % 3. Верхний слой A (z = c)
        \node[opacity=0.8, ball color=red!70] at (0,0,\c) {}; % Центр
        \foreach \ang in {0,60,120,180,240,300} {
            \node[opacity=0.8] at ({\a*cos(\ang)}, {\a*sin(\ang)}, \c) {};
        }

        % 4. Каркас (аналогично вашему draw[thick, opacity=0.3])
        \begin{scope}[thick, opacity=0.3, black]
            % Основания
            \foreach \ang in {0,60,...,300} {
                \draw ({\a*cos(\ang)}, {\a*sin(\ang)}, 0) -- ({\a*cos(\ang+60)}, {\a*sin(\ang+60)}, 0);
                \draw ({\a*cos(\ang)}, {\a*sin(\ang)}, \c) -- ({\a*cos(\ang+60)}, {\a*sin(\ang+60)}, \c);
                % Вертикальные ребра
                \draw ({\a*cos(\ang)}, {\a*sin(\ang)}, 0) -- ({\a*cos(\ang)}, {\a*sin(\ang)}, \c);
            }
        \end{scope}

        % 5. Тетраэдр
        \draw[thick, color=black] (ball30) -- (down60);
        \draw[thick, color=black] (ball30) -- (down0);
        \draw[thick, color=black] (ball30) -- (center);
        \draw[thick, color=black] (down60) -- (down0);
        \draw[thick, color=black] (down0) -- (center);
        \draw[thick, color=black] (center) -- (down60);
    \end{tikzpicture}
    \caption{Элементарная ячейка гексагональной плотноупакованной структуры}
    \label{ИГЦК}
\end{figure}

Атомы в базисной плоскости касаются друг друга,
поэтому сторона шестиугольника $a = 2R$.

Центральный слой атомов располагается в «выемках» между атомами нижнего слоя.
Рассмотрим тетраэдр (выделенный контур на рис. \ref{ИГЦК}), образованный тремя касающимися атомами в нижнем слое и одним атомом из среднего слоя,
который лежит в углублении между ними.
\begin{enumerate}
    \item Все ребра этого тетраэдра равны $a$ (так как все атомы касаются друг друга).
    \item Высота этого тетраэдра составляет ровно половину высоты всей ячейки,
    то есть $c/2$.
    \item Расстояние от вершины до центра основания (медиана треугольника в основании):
    $r_d = \frac{a}{\sqrt{3}}$.
\end{enumerate}
Тогда по теореме Пифагора для высоты тетраэдра ($h = c/2$):
\begin{gather*}
    (c/2)^2 + (a/\sqrt{3})^2 = a^2 \\
    \dfrac{c^2}{a^2} = \dfrac{8}{3}
\end{gather*}
Что и требовалось доказать.

Найдём коэффициент компактности $f$.
Число атомов в ячейке $N$ состоит из суммы:
\begin{itemize}
    \item 12 атомов в вершинах (каждый принадлежит 6 ячейкам): $12 \cdot 1/6 = 2$.
    \item 2 атома в центрах оснований (каждый принадлежит 2 ячейкам): \mbox{$2 \cdot 1/2 = 1$}.
    \item 3 атома внутри ячейки: $3 \cdot 1 = 3$.
\end{itemize}
Итого $N = 6$. Объём элементарной ячейки равен объёму прямоугольной призмы
с правильным шестиугольником со стороной $a$ в основании.
Площадь правильного шестиугольника будет равна
шести площадям равносторонних треугольников со стороной $a$
$$
    S_\text{осн.} = 6 \cdot \dfrac{a^2 \sqrt{3}}{4} = \dfrac{3\sqrt{3}}{2}a^2
$$
Объём прямоугольной призмы будет равен произведению площади основания на высоту
$$
    V_\text{ячейки} = \dfrac{3\sqrt{3}}{2}a^2 \cdot a\sqrt{\dfrac{8}{3}} = 3\sqrt{2}a^2
$$
Тогда
$$
    f = \dfrac{8 \cdot \dfrac{4}{3}\pi\left(\dfrac{a}{2}\right)^3}{3\sqrt{2}a^3} = \dfrac{\pi}{3\sqrt{2}} \approx 0.74
$$

\subsection*{Ответ}

$$
    f = \dfrac{\pi}{3\sqrt{2}} \approx 0.74
$$

\section*{\centering Задача 7}

\subsection*{Условие}

\begin{enumerate}
    \item Найдите плотность точек решётки в атомной плоскости (число узлов на единицу
    площади), если расстояние между соседними плоскостями в семействе, которому
    принадлежит данная плоскость, равно $d$, а объёмная плотность точек данной
    решётки Бравэ равна $n$;
    \item Найдите и изобразите семейство плоскостей в г.ц.к. и о.ц.к.
    решётках Бравэ, для которых плотность точек максимальна (плоскости скола).
\end{enumerate}

\subsection*{Решение}

\begin{enumerate}
    \item Рассмотрим семейство параллельных кристаллографических плоскостей.
    Пусть расстояние между соседними плоскостями равно $d$.
    По определению семейства плоскостей,
    каждая точка решётки Бравэ принадлежит какой-то одной плоскости
    из этого семейства.

    Выделим на одной из плоскостей участок площадью $S$.
    Если мы продлим этот участок перпендикулярно плоскостям,
    мы получим объём
    $$
        V = S \cdot d
    $$
    ограниченный цилиндрической поверхностью.

    В объёме $V$, ограниченном расстоянием $d$, содержится ровно столько точек,
    сколько их приходится на площадь $S$ одной плоскости.
    Число точек в этом объёме
    $$
        N = n \cdot V = n \cdot S \cdot d
    $$
    Поверхностная плотность $\sigma$ --- это количество точек на единицу площади:
    $$
        \sigma = \frac{N}{S}
    $$
    Тогда
    $$
        \sigma = n \cdot d
    $$
    Т.е. чем больше межплоскостное расстояние,
    тем выше плотность упаковки атомов в самой плоскости.

    \item Согласно формуле $\sigma = nd$, плоскости с максимальной плотностью точек
    --- это те плоскости, у которых межплоскостное расстояние $d$ максимально.

    Пусть $\vec{a}_1, \vec{a}_2, \vec{a}_3$ --- основные векторы трансляции решётки Бравэ.
    Любое семейство параллельных плоскостей характеризуется вектором обратной решётки $\vec{g}_{hkl}$:
    $$
        \vec{g}_{hkl} = h\vec{b}_1 + k\vec{b}_2 + l\vec{b}_3
    $$
    где $\vec{b}_i$ — базисные векторы обратной решётки.
    Из свойств обратной решётки известно, что:
    \begin{itemize}
        \item Вектор $\vec{g}_{hkl}$ перпендикулярен плоскостям $(hkl)$.
        \item Межплоскостное расстояние $d_{hkl}$ обратно пропорционально длине этого вектора:
        $$
            d_{hkl} = \frac{2\pi}{|\vec{g}_{hkl}|}
        $$
    \end{itemize}
    Так как из пункта (а) мы знаем, что $\sigma = n \cdot d$,
    то для поиска максимальной плотности $\sigma$ нам нужно найти минимальный по длине
    ненулевой вектор обратной решётки $\vec{g}$.

    Чтобы минимизировать $|\vec{g}_{hkl}|$, мы ищем такие целые индексы Миллера $(h, k, l)$,
    при которых комбинация $h\vec{b}_1 + k\vec{b}_2 + l\vec{b}_3$
    даст кратчайший вектор.
    \begin{description}
        \item[ГЦК] Решётка, обратная к ГЦК с параметром $a$,
        является ОЦК решёткой с параметром $4\pi/a$.
        Базисные векторы обратной решётки для ГЦК:
        $$
            \vec{b}_1 = \frac{4\pi}{a}(-\vec{i} + \vec{j} + \vec{k}), \quad \vec{b}_2 = \frac{4\pi}{a}(\vec{i} - \vec{j} + \vec{k}), \quad \vec{b}_3 = \frac{4\pi}{a}(\vec{i} + \vec{j} - \vec{k})
        $$
        В ОЦК решётке кратчайшими векторами являются векторы в узлы типа
        $(\pm 1, \pm 1, \pm 1)$ в стандартных индексах (соответствует векторам
        в вершины «куба» обратной решётки).
        Это соответствует семейству плоскостей $(1, 1, 1)$. Приведено на рис. \ref{ГЦК-обратная}.
        
        \begin{figure}[h]
            \centering
            \begin{tikzpicture}[scale=3, every node/.style={shading=ball, ball color=blue!60}]
                % Вершины нижнего основания
                \node (A) at (0,0,0) [circle, minimum size=10pt] {};
                \node (B) at (1,0,0) [circle, minimum size=10pt] {};
                \node (C) at (1,0,1) [circle, minimum size=10pt] {};
                \node (D) at (0,0,1) [circle, minimum size=10pt] {};
                
                \fill[yellow!70, opacity=0.5] (1,0,0) -- (0,1,0) -- (0,0,1) -- cycle;
                \draw[yellow!70, very thick] (1,0,0) -- (0,1,0) -- (0,0,1) -- cycle;
                
                % Центры граней (нижняя, боковые, передняя)
                \node at (0.5,0,0.5) [circle, minimum size=10pt, ball color=red!70] {}; % Нижняя
                \node at (1,0.5,0.5) [circle, minimum size=10pt, ball color=red!70] {}; % Правая
                \node at (0,0.5,0.5) [circle, minimum size=10pt, ball color=red!70] {}; % Левая
                
                % Вершины верхнего основания
                \node (E) at (0,1,0) [circle, minimum size=10pt] {};
                \node (F) at (1,1,0) [circle, minimum size=10pt] {};
                \node (G) at (1,1,1) [circle, minimum size=10pt] {};
                \node (H) at (0,1,1) [circle, minimum size=10pt] {};
                
                % Центры граней (верхняя, задняя, передняя)
                \node at (0.5,1,0.5) [circle, minimum size=10pt, ball color=red!70] {}; % Верхняя
                \node at (0.5,0.5,0) [circle, minimum size=10pt, ball color=red!70] {}; % Задняя
                \node at (0.5,0.5,1) [circle, minimum size=10pt, ball color=red!70] {}; % Передняя
                
                % Ребра куба для наглядности
                \draw[thick, opacity=0.3] (0,0,0) -- (1,0,0) -- (1,0,1) -- (0,0,1) -- cycle;
                \draw[thick, opacity=0.3] (0,1,0) -- (1,1,0) -- (1,1,1) -- (0,1,1) -- cycle;
                \draw[thick, opacity=0.3] (0,0,0) -- (0,1,0) (1,0,0) -- (1,1,0) (1,0,1) -- (1,1,1) (0,0,1) -- (0,1,1);
            \end{tikzpicture}
            \caption{Семейство плоскостей ГЦК с минимальной плотностью точек}
            \label{ГЦК-обратная}
        \end{figure}

        \item[ОЦК] Решётка, обратная к ОЦК с параметром $a$,
        является ГЦК решёткой с параметром $4\pi/a$.
        Базисные векторы обратной решётки для ОЦК:
        $$
            \vec{b}_1 = \frac{4\pi}{a}(\vec{j} + \vec{k}), \quad \vec{b}_2 = \frac{4\pi}{a}(\vec{i} + \vec{k}), \quad \vec{b}_3 = \frac{4\pi}{a}(\vec{i} + \vec{j})
        $$
        В ГЦК решётке кратчайшими векторами являются векторы к центрам граней,
        то есть векторы типа $(\pm 1, \pm 1, 0)$ и их перестановки.
        Это соответствует семейству плоскостей $(1, 1, 0)$. Приведено на рис. \ref{ОЦК-обратная}.

        \begin{figure}[h]
            \centering
            \begin{tikzpicture}[scale=3, every node/.style={shading=ball, ball color=blue!60}]
                % Вершины
                \foreach \x in {0,1} \foreach \y in {0,1} \foreach \z in {0,1}
                \node at (\x,\y,\z) [circle, minimum size=10pt] {};

                \fill[yellow!70, opacity=0.5] (1,0,0) -- (1,0,1) -- (0,1,1) -- (0,1,0) -- cycle;
                \draw[yellow!70, very thick] (1,0,0) -- (1,0,1) -- (0,1,1) -- (0,1,0) -- cycle;
                
                % Центральный атом
                \node at (0.5,0.5,0.5) [circle, minimum size=10pt, ball color=green!70] {};
                
                % Ребра куба
                \draw[thick, opacity=0.3] (0,0,0) -- (1,0,0) -- (1,0,1) -- (0,0,1) -- cycle;
                \draw[thick, opacity=0.3] (0,1,0) -- (1,1,0) -- (1,1,1) -- (0,1,1) -- cycle;
                \draw[thick, opacity=0.3] (0,0,0) -- (0,1,0) (1,0,0) -- (1,1,0) (1,0,1) -- (1,1,1) (0,0,1) -- (0,1,1);
            \end{tikzpicture}
            \caption{Семейство плоскостей ОЦК с минимальной плотностью точек}
            \label{ОЦК-обратная}
        \end{figure}
    \end{description}
\end{enumerate}

\subsection*{Ответ}

\begin{enumerate}
    \item $\sigma = n \cdot d$
    \item Для ГЦК: $(1, 1, 1)$: Для ОЦК: $(1, 1, 0)$ (или другая, полученная перестановкой).
\end{enumerate}

\section*{\centering Задача 8}

Найдите геометрический структурный фактор моноатомной г.ц.к. решётки Бравэ,
представленной в виде простой кубической решётки с четырёхточечным базисом.
Установите, какую структуру образуют точки обратной решётки с ненулевым
структурным фактором. Почему можно было ожидать такой результат?

\subsection*{Решение}

Гранецентрированная кубическая (ГЦК) решетка может быть описана
как простая кубическая решетка с ребром $a$,
в узлах которой находится базис из 4 атомов.
Координаты узлов базиса имеют следующий вид:
$$
    \begin{cases}
        \vec{r}_1 = (0, 0, 0) \\
        \vec{r}_2 = (1/2, 1/2, 0) \\
        \vec{r}_3 = (0, 1/2, 1/2) \\
        \vec{r}_4 = (1/2, 0, 1/2)
    \end{cases}
$$
Геометрический структурный фактор определяется формулой
$$
    S(\vec{b}) = \sum_{j=1}^n e^{-i(\vec{b}, \vec{r}_j)}
$$
где $\vec{b}$ --- вектор обратной решётки, который имеет вид
$$
    \vec{b} = \dfrac{2\pi}{a} (h, k, l)
$$
где $h, k, l \in \mathbb{Z}$ --- индексы Миллера. Подставим и получим
$$
    S(h, k, l) = 1 + (-1)^{h + k} + (-1)^{k + l} + (-1)^{h + l}
$$
Рассмотрим различные случаи чётности степеней экспонент:
\begin{enumerate}
    \item 3 чётные: тогда $S = 4$ --- подходит.
    
    Этот случай выполняется, если ($h, k, l, a, b, c \in \mathbb{Z}$):
    $$
        \begin{cases}
            h + k = 2a \\
            k + l = 2b \\
            h + l = 2c
        \end{cases} \Rightarrow \begin{cases}
            h = a + c - b \\
            k = a + b - c \\
            l = b + c - a
        \end{cases}
    $$
    \item 2 четные, 1 нечётная: тогда $S = 2$ --- подходит.
    
    Этот случай выполняется, если ($h, k, l, a, b, c \in \mathbb{Z}$):
    $$
        \begin{cases}
            h + k = 2a + 1 \\
            k + l = 2b \\
            h + l = 2c
        \end{cases} \Rightarrow \begin{cases}
            h = a + c - b + 1/2 \\
            k = a + b - c + 1/2 \\
            l = b + c - a
        \end{cases}
    $$
    Не удовлетворяет условиям целых $h, k, l$.
    \item 1 чётная, 2 нечётные: тогда $S = 0$ --- не подходит.
    \item 3 нечётные: тогда $S = -2$ --- подходит.
    
    Этот случай выполняется, если ($h, k, l, a, b, c \in \mathbb{Z}$):
    $$
        \begin{cases}
            h + k = 2a + 1 \\
            k + l = 2b + 1 \\
            h + l = 2c + 1
        \end{cases} \Rightarrow \begin{cases}
            h = a + c - b + 1/2 \\
            k = a + b - c + 1/2 \\
            l = b + c - a + 1/2
        \end{cases}
    $$
    Не удовлетворяет условиям целых $h, k, l$.
\end{enumerate}
Таким образом, все три показателя экспоненты должны быть чётными.
Это условие выполняется когда либо все числа из набора $(h, k, l)$
чётные, либо все нечётные.

Этот результат ожидаем из следующих соображений:
рассмотрим примеры точек, удовлетворяющих полученному правилу.
Это будут точки $(0, 0, 0)$, $(\pm 2, 0, 0)$ с перестановками,
и т.д. Если построить эти точки на кубе со стороной 2,
то можно видеть, что была получена ОЦК решётка,
что можно было ожидать ввиду существования соответствующей теоремы.

(Также, фактически, решение данной задачи может быть найдено в книге <<Физика твёрдого тела>> (Том 1)
за авторством Ашкрофта и Мермина \mbox{на стр. 114-115} с той лишь оговоркой,
что в данной книге оно проводится в обратную сторону относительно поставленной задачи)

\section*{\centering Задача 9}

\subsection*{Условие}

Измельчённые в порошок образцы трёх различных моноатомных кубических
кристаллов анализируются с помощью метода Дебая-Шеррера. Известно, что один из
образцов --- гранецентрированный кубический, другой --- объёмноцентрированный
кубический и третий имеет структуру типа алмаза. Примерные положения первых
четырёх дифракционных колец характеризуются в каждом случае следующими углами
отклонения от направления падения исходного пучка (в градусах):
\begin{description}
    \item[Образец A:] 42.2, 49.2, 72.0, 87.3;
    \item[Образец B:] 28.8, 41.0, 50.8, 59.6;
    \item[Образец C:] 42.8, 73.2, 89.0, 115.0;
\end{description}
Определите кристаллическую структуру образцов A, B и C.
Если длина волны падающих рентгеновских лучей равна
$1.5\ \overset{\circ }{\text{A}}$, чему равна длина стороны условной кубической ячейки в каждом из указанных
случаев?

\subsection*{Решение}

Уравнение Вульфа-Брэгга для межплоскостного расстояния $d$ имеет вид:
$$
    2d \sin\left(\dfrac{\theta}{2}\right) = \lambda
$$
Для кубического кристалла с параметром решетки $a$ расстояние
$d$ между плоскостями с индексами Миллера $(hkl)$ равно:
$$
    d = \frac{a}{\sqrt{h^2 + k^2 + l^2}}
$$
Объединяя данные две формулы, получаем выражение для углов дифракции:
$$
    \sin^2 \left(\dfrac{\theta}{2}\right) = \frac{\lambda^2}{4a^2} (h^2 + k^2 + l^2) := \dfrac{\lambda^2}{4a^2} \cdot S
$$
Тогда имеем:
\begin{description}
    \item[ОЦК:]
    
    Условие максимума: $h + k + l$ --- чётное. Тогда последовательность для $S$
    примет вид: 2, 4, 6, 8, \ldots (Т.е. отношение имеет вид: 1 : 2 : 3 : 4 : 5 : \ldots)
    \item[ГЦК:]
    
    Условие максимума: $h, k, l$ либо все чётные, либо все нечётные.
    Тогда последовательность $S$ примет вид: 3, 4, 8, 11, 12, \ldots

    \item[Структура типа алмаза:]
    Условие максимума: $h, k, l$ либо все нечётные, либо все чётные и $h + k + l = 4n$.
    Тогда последовательность значений $S$ примет вид: 3, 8, 11, 16, 19, \ldots
\end{description}
Тогда требуется преобразовать экспериментальные данные, которые приведены в таблице \ref{Экспериментальные данные}.
\begin{table}[h]
    \begin{tabular}{|c|c|c|c|}
        \hline
        Образец & Образец А & Образец B & Образец C \\
        \hline
        $\theta$ & 42.2, 49.2, 72.0, 87.3 & 28.8, 41.0, 50.8, 59.6 & 42.8, 73.2, 89.0, 115.0 \\
        \hline
        $sin^2(\theta/2)$ & 0.129, 0.173, 0.345, 0.476 & 0.062, 0.123, 0.184, 0.247 & 0.133, 0.355, 0.491, 0.711 \\
        \hline
        Соотношение & 1 : 1.33 : 2.67 : 3.67 & 1 : 2 : 3 : 4 & 1 : 2.67 : 3.67 : 5.33 \\
        \hline
        Вывод & ГЦК & ОЦК & Структура алмаза \\
        \hline
    \end{tabular}
    \caption{Экспериментальные данные}
    \label{Экспериментальные данные}
\end{table}

\subsection*{Ответ}

\begin{description}
    \item[Образец A:] ГЦК
    \item[Образец B:] ОЦК
    \item[Образец C:] Структура типа алмаза
\end{description}
(Задача 1, Глава 6 в <<Физика твёрдого тела>> (Том 1) за авторством Ашкрофта и Мермина)

\end{document}