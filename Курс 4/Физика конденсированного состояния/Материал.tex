\documentclass[a4paper]{report}

\usepackage{mathtext} % использование кириллицы в формулах
\usepackage{cmap} % грамотное копирование кириллицы из pdf
\usepackage[T2A]{fontenc} % внутрення кодировка
\usepackage[utf8]{inputenc} % кодировка документа
\usepackage[russian]{babel} % язык документа
\usepackage{amssymb} % дополнительные символы
\usepackage{amsfonts} % математические шрифты
\usepackage{amsmath} % дополнительная математика
\usepackage{indentfirst} % отступ и у первого абзаца
\usepackage{tikz}
\usepackage{tikz-3dplot}
\usepackage[hidelinks]{hyperref} % ссылки

\title{Материал по курсу\\
<<Физика конденсированного состояния>>}

\begin{document}

\maketitle

\tableofcontents

\chapter{Введение в электронную теорию твёрдого тела. Модель свободных
  электронов}

Различные состояния вещества:
\begin{itemize}
    \item Твёрдое тело (сохраняет и форму, и объём)
    \item Жидкость (сохраняет объём)
    \item Газ
\end{itemize}
(Объём может изменяться в зависимости от температуры)

\section{Определение и общие свойства конденсированного состояния вещества}

К конденсированному состоянию вещества относятся твёрдое тело и жидкость
(хотя разность между твёрдым телом и жидкостью больше,
чем между жидкостью и газом, между которыми в т.ч.
можно перейти без фазового перехода)

\section{Классификация твёрдых тел по их структуре и электрофизическим
  свойствам}

\section{Электро- и теплопроводность твёрдых тел}

\subsection{Закон Ома}

\subsection{Температурная зависимость удельного электросопротивления}

\subsection{Закон Фурье}

\subsection{Коэффициент теплопроводности}

\subsection{Закон Видемана-Франца для металлов}

\section{Модель свободных электронов Друде}

В 1897 г. Томсон открыл электрон. Это открытие оказало громадное
и непосредственное воздействие на теорию материи и позволило также объяснить проводимость металлов. Через три года после открытия Томсона Друде
разработал свою теорию электро- и теплопроводности. При этом он рассматривал электроны в металле как электронный газ и применил к нему кинетическую
теорию газов, оказавшуюся весьма плодотворной.

В кинетической теории, в ее самой простой форме, считают, что молекулы
газа представляют собой одинаковые твердые сферы, которые движутся по прямым линиям до тех пор, пока не столкнутся друг с другом. Предполагается,
что продолжительность отдельного столкновения пренебрежимо мала и что
между молекулами не действует никаких иных сил, кроме возникающих в момент столкновения.

В простейших газах имеются лишь частицы одного сорта, в металлах же
их должно быть по меньшей мере два: электроны заряжены отрицательно,
а металл в целом электрически нейтрален. Друде предположил, что компенсирующий положительный заряд принадлежит гораздо более тяжелым частицам, которые он считал неподвижными. В то время, однако, еще не понимали
точно, почему в металле имеются подобные легкие подвижные электроны и более
тяжелые неподвижные положительно заряженные ионы. Решение этой проблемы стало одним из фундаментальных достижений современной квантовой теории
твердого тела. При обсуждении модели Друде, однако, нам будет достаточно
просто предположить (для многих металлов это предположение оправдано), что
когда атомы металлического элемента объединяются, образуя металл, валентные
электроны освобождаются и получают возможность свободно передвигаться
по металлу, тогда как металлические ионы остаются неизменными и играют
роль неподвижных положительных частиц теории Друде. Эта модель схематически изображена на фиг. 1.1. Каждый отдельный атом металлического элемента имеет ядро с зарядом $eZ_a$, где $Z_a$ — атомный номер и $e$ — величина заряда
электрона: $e = 4,80 \cdot 10^{-10}$ ед. СГСЭ $= 1,60 \cdot 10^{-19}$ Кл.
Вокруг ядра расположено $Z_a$ электронов с полным зарядом — $eZ_a$.
Некоторое число $Z$ из них — это
слабо связанные валентные электроны. Остающиеся $Z_a$ — $Z$ электронов довольно сильно связаны с ядром; они играют меньшую роль в химических реакциях
и носят название электронов атомного остова. Когда изолированные атомы
объединяются, образуя металл, электроны атомного остова остаются связанными
с ядрами, т. е. возникают металлические ионы. Валентные же электроны,
наоборот, приобретают возможность далеко уходить от «родительских» атомов.
В металлах эти электроны называют электронами проводимости.

К такому «газу», состоящему из электронов с массой т, которые (в отличие
от молекул обычного газа) движутся на фоне тяжелых неподвижных ионов,
Друде применил кинетическую теорию. Плотность электронного газа можно
рассчитать следующим образом.

Плотность газа электронов проводимости примерно в 1000 раз больше
плотности классического газа при нормальных температуре и давлении.
Несмотря на это и несмотря на наличие сильного электрон-электронного и
электронионного взаимодействия в модели Друде для рассмотрения электронного газа
в металлах почти без изменений применяются методы кинетической теории
нейтральных разреженных газов. Приведем основные предположения теории
Друде.
\begin{enumerate}
    \item В интервале между столкновениями не учитывается взаимодействие
электрона с другими электронами и ионами. Иными словами, принимается,
что в отсутствие внешних электромагнитных полей каждый электрон движется
с постоянной скоростью по прямой линии. Далее, считают, что в присутствии
внешних полей электрон движется в соответствии с законами Ньютона; при
этом учитывают влияние только этих полей, пренебрегая сложными дополнительными полями, порождаемыми другими электронами и ионами х). Приближе-
ние, в котором пренебрегают электрон-электронным взаимодействием в промежутках между столкновениями, известно под названием приближения независимых электронов. Соответственно приближение, в котором пренебрегают
электрон-ионным взаимодействием, называется приближением свободных электронов. В последующих главах мы обнаружим, что приближение независимых
электронов оказывается неожиданно удачным во многих отношениях, тогда
как от приближения свободных электронов приходится отказаться, даже если
мы хотим достичь лишь качественного
понимания поведения металлов.
    \item В модели Друде, как и в кине-
тической теории, столкновения — это
мгновенные события, внезапно меняю-
щие скорость электрона. Друде связывал их с тем, что электроны отскакивают от
непроницаемых сердцевин ионов (а не считал их электрон-электронными столкновениями по аналогии с
доминирующим механизмом столкновений в обычном газе). Позднее мы увидим, что при обычных
условиях рассеяние электронов на электронах действительно является одним
из наименее существенных механизмов рассеяния в металле. Однако простая
механическая модель (фиг. 1.2), согласно которой электрон отскакивает от
иона к иону, весьма далека от действительности. К счастью, во многих
задачах это не важно: для качественного (и даже количественного) понимания
проводимости металлов достаточно просто предположить существование какого-то механизма рассеяния, не вдаваясь в подробности относительно того,
каков именно этот механизм. Используя в своем анализе лишь несколько общих
свойств процесса столкновения, мы можем не связывать себя конкретной картиной столкновений. Эти общие характерные черты описываются следующими
двумя предположениями.
    \item Будем предполагать, что за единицу времени электрон испытывает
столкновение (т. е. внезапное изменение скорости) с вероятностью, равной
1/т. Имеется в виду, что для электрона вероятность испытать столкновение
в течение бесконечно малого промежутка времени dt равна просто dtlx. Время т называют временем релаксации, или временем свободного пробега; оно
играет фундаментальную роль в теории проводимости металлов. Из этого
предположения следует, что электрон, выбранный наугад в настоящий момент
времени, будет двигаться в среднем в течение времени т до его следующего
столкновения и уже двигался в среднем в течение времени т с момента предыдущего столкновения. В простейших приложениях модели Друде считают, что
время релаксации т не зависит от пространственного положения электрона
и его скорости. Позднее мы увидим, что во многих, но не во всех задачах такое
предположение оказывается удивительно хорошим.
    \item Предполагается, что электроны приходят в состояние теплового равно
весия со своим окружением исключительно благодаря столкновениям. Считается, что столкновения поддерживают локальное термодинамическое равновесие чрезвычайно простым способом: скорость электрона сразу же после
столкновения не связана с его скоростью до столкновения, а направлена случайным образом, причем ее величина соответствует той температуре, которая
превалирует в области, где происходило столкновение. Поэтому чем более
горячей является область, где происходит столкновение, тем большей скоростью
обладает электрон после столкновения.
\end{enumerate}

\subsection{Электроны проводимости как
    идеальный газ классических частиц}

\subsection{Электрон-ионные столкновения, сечение
    рассеяния и среднее время свободного пробега электронов}

\subsection{Расчёт
    статической электропроводности и коэффициента теплопроводности
    металлов в рамках элементарной кинетической теории разреженных газов}

\section{Понятие о термоэлектрических эффектах}

\subsection{Эффект Зеебека}

\subsection{Эффект Пельтье}

\section{Квантовая теория свободных электронов}

\subsection{Принцип неразличимости
    тождественных частиц, фермионы и бозоны}

\subsection{Принцип запрета Паули, спин
    электрона, структура волновой функции газа невзаимодействующих
    электронов}

\subsection{Модель свободных электронов Зоммерфельда, периодические
    граничные условия}

\subsection{Свойства идеального газа свободных электронов в
    основном состоянии}

\subsubsection{Импульс и энергия Ферми}

\subsubsection{Плотность разрешённых волновых векторов и плотность одноэлектронных уровней энергии}

\subsubsection{Энергия основного состояния электронного газа}

\subsection{Распределение Ферми-Дирака}

\subsection{Термодинамические свойства газа свободных электронов}

\subsubsection{Температура Ферми}

\subsubsection{Условие вырождения электронного газа}

\subsubsection{Температурное разложение Зоммерфельда}

\subsubsection{Расчёт удельной теплоёмкости вырожденного электронного газа}

\section{Теория металлической проводимости Зоммерфельда}

Во времена Друде и затем в течение многих лет вполне разумным казалось
предположение, что распределение электронов по скоростям совпадает с рас-
пределением в обычном классическом газе с плотностью $n = N/V$ и описывается
в состоянии равновесия при температуре $Т$ формулой Максвелла — Больцмана.

\subsection{Квазиклассическое
    уравнение Больцмана для неравновесной функции распределения электронов
    и условия его применимости}

\subsection{Общая структура интеграла столкновений с
    учётом принципа запрета Паули и принцип детального баланса}

\subsection{Дифференциальное сечение рассеяния и транспортное время свободного
    пробега электронов}

\subsection{Расчёт статической электропроводности, коэффициента
    теплопроводности и дифференциальной термо-э.д.с. металлов в рамках
    квазиклассического уравнения Больцмана в случае упругих электрон-ионных
    столкновений}

\subsection{Особенности измерения дифференциальной термо-э.д.с.
    металлов, биметаллический контур, понятие об электрохимическом
    потенциале}

\subsection{Число Лоренца и закон Видемана-Франца в рамках
    зоммерфельдовской теории свободных электронов}

\section{Магнетизм электронного газа}

\subsection{Определение собственного и орбитального магнитного моментов электрона}

\subsection{Определение магнитной восприимчивости}

\section{Диамагнетизм и парамагнетизм. Теорема Бора -- Ван Леевен о нулевой
  магнитной восприимчивости газа классических заряженных частиц}

\section{Магнитная восприимчивость больцмановского газа электронов с учётом их
  собственного магнитного момента}

\subsection{Закон Кюри}

\subsection{Трудности классических теорий электронного газа}

\section{Магнитная восприимчивость вырожденного электронного газа}

\subsection{Парамагнетизм Паули}

\subsection{Понятие о диамагнетизме Ландау}

\subsection{Недостатки модели свободных электронов}

\chapter{Свойства кристаллических решеток}

\section{Трансляционная симметрия кристаллов}

\subsection{Решётка Бравэ и основные векторы трансляций}

\subsection{Простая, объёмно-центрированная и гранецентрированная кубические решётки}

\subsection{Примитивная (элементарная) ячейка, ячейка Вигнера-Зейтца и условная ячейка}

\subsection{Кристаллические структуры и решётки с базисом}

\section{Гексагональная плотноупакованная структура}

\subsection{Координационное число и коэффициент компактности (упаковочный множитель)}

\subsection{Алгоритм построения различных плотноупакованных структур}

\subsection{Структуры типа хлорида натрия, алмаза и цинковой обманки}

\section{Понятие об элементах симметрии кристаллических решёток и группе
  симметрии (пространственной группе) решётки Бравэ}

\subsection{Поворотные оси симметрии}

\subsection{Теорема о симметрии кристаллических решёток по отношению к поворотам}

\section{Прямая и обратная решётки}

\subsection{Свойства обратной решётки}

\subsection{Примеры обратных решёток}

\subsection{Первая зона Бриллюэна}

\subsection{Атомные плоскости и семейства атомных плоскостей}

\subsection{Теорема о связи семейств атомных плоскостей с векторами обратной решётки}

\subsection{Индексы Миллера}

\section{Условия конструктивной интерференции рентгеновских лучей в кристалле}

\subsection{Формулировка Брэгга}

\subsection{Формулировка Лауэ}

\subsection{Доказательство эквивалентности формулировок}

\section{Геометрические формулировки условий конструктивной интерференции}

\subsection{Построение Бриллюэна}

\subsection{Брэгговские плоскости}

\subsection{Построение Эвальда}

\section{Экспериментальные методы определения кристаллических структур}

\subsection{Метод Лауэ}

\subsection{Метод вращающегося кристалла}

\subsection{Порошковый метод (метод Дебая-Шеррера)}

\section{Геометрический структурный фактор кристаллических структур}

\chapter{Состояния электронов в кристаллической решётке. Основы зонной
  теории твёрдых тел}

\section{Квантовые состояния электрона в периодическом потенциале}

\subsection{Теорема Блоха}

\subsection{Квазиимпульс электрона и его свойства}

\subsection{Граничные условия Борна-Кармана
    для кристаллов и число разрешённых значений квазиимпульса}

\subsection{Периодичность волновых функций и энергетического спектра в обратном
    пространстве}

\subsection{Энергетические зоны и их свойства}

\subsection{Запрещённые зоны энергий,
    описание электронных состояний в схемах приведённых и повторяющихся зон}

\section{Понятие о <<$\vec{k}\vec{p}$>> методе}

\section{Теорема о средней скорости блоховского электрона}

\section{Свойства энергетического спектра вблизи экстремумов энергии в зоне Бриллюэна}

\section{Тензор обратных эффективных масс электрона и его свойства}

\section{Отсутствие вклада в электрический ток от
  полностью заполненных зон (инертность заполненных зон), критерии
  металла и диэлектрика}

\subsection{Диэлектрики}

\subsection{Полупроводники}

\subsection{Металлы}

\subsection{Полуметаллы}

\end{document}