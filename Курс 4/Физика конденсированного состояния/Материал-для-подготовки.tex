\documentclass[a4paper]{report}

\usepackage{mathtext} % использование кириллицы в формулах
\usepackage{cmap} % грамотное копирование кириллицы из pdf
\usepackage[T2A]{fontenc} % внутрення кодировка
\usepackage[utf8]{inputenc} % кодировка документа
\usepackage[russian]{babel} % язык документа
\usepackage{amssymb} % дополнительные символы
\usepackage{amsfonts} % математические шрифты
\usepackage{amsmath} % дополнительная математика
\usepackage{indentfirst} % отступ и у первого абзаца
\usepackage{tikz}
\usepackage{tikz-3dplot}
\usepackage[hidelinks]{hyperref} % ссылки

\title{Учебный материал по курсу\\
<<Физика конденсированного состояния>>}
\author{Google Gemmini 3 \and Андрей Можаров}

\begin{document}

\maketitle

\tableofcontents

\chapter{Введение в электронную теорию твёрдого тела. Модель свободных электронов.}

\section{Определение и общие свойства конденсированного состояния вещества}

Конденсированное состояние вещества — это макроскопическая система, состоящая из огромного числа частиц ($N \sim 10^{23}$ см$^{-3}$), между которыми существуют сильные взаимодействия, сопоставимые по величине с энергией теплового движения.

\subsection*{Основные признаки конденсированного состояния:}

\begin{itemize}
    \item \textbf{Высокая плотность:} В отличие от газов, межатомные расстояния в конденсированных средах соизмеримы с размерами самих атомов. Это делает их практически несжимаемыми[cite: 1, 3].
    \item \textbf{Сильное взаимодействие:} Потенциальная энергия взаимодействия частиц $U$ значительно превышает их среднюю кинетическую энергию теплового движения $k_B T$. Это приводит к возникновению корреляций в пространственном расположении частиц[cite: 1, 3].
    \item \textbf{Наличие упорядоченности:}
    \begin{itemize}
        \item \textit{Дальний порядок:} Характерен для кристаллов, где структура строго периодична на макроскопических расстояниях[cite: 1, 3].
        \item \textit{Ближний порядок:} Характерен для жидкостей и аморфных тел, где упорядоченность сохраняется только для ближайших соседей[cite: 1, 3].
    \end{itemize}
\end{itemize}

\subsection*{Классификация по типу связи:}
Ашкрофт и Мермин выделяют следующие основные типы связи, определяющие физические свойства состояния[cite: 1, 3]:
\begin{enumerate}
    \item \textbf{Ионная связь:} Обусловлена кулоновским притяжением разноименно заряженных ионов (например, $NaCl$).
    \item \textbf{Ковалентная связь:} Возникает за счет обобществления электронных пар (например, алмаз, кремний).
    \item \textbf{Металлическая связь:} Характеризуется наличием «электронного газа», удерживающего положительно заряженные ионы в узлах решетки.
    \item \textbf{Ван-дер-ваальсова связь:} Слабое притяжение между нейтральными атомами или молекулами за счет флуктуаций дипольных моментов (например, твердые инертные газы).
    \item \textbf{Водородная связь:} Специфический тип связи, важный для органических соединений и льда.
\end{enumerate}

\subsection*{Макроскопические свойства:}
\begin{itemize}
    \item \textbf{Механические:} Упругость, пластичность и твердость[cite: 1, 3].
    \item \textbf{Термические:} Теплоемкость и теплопроводность, определяемые как электронной подсистемой, так и колебаниями решетки[cite: 1, 2, 3].
    \item \textbf{Электрические:} Огромный диапазон удельной проводимости (от металлов до диэлектриков)[cite: 1, 3].
\end{itemize}

\section{Классификация твёрдых тел по структуре и электрофизическим свойствам}

Классификация твёрдых тел проводится по двум основным признакам: атомному упорядочению (структуре) и способности проводить электрический ток (электронным свойствам).

\subsection*{1. Классификация по структуре}
С точки зрения пространственного расположения атомов выделяют:
\begin{itemize}
    \item \textbf{Кристаллы:} Характеризуются наличием \textit{дальнего порядка}. Атомы расположены в узлах периодической решетки Бравэ. Свойства часто анизотропны. 
    \item \textbf{Аморфные тела:} Обладают только \textit{ближним порядком} (упорядоченность только в ближайшем окружении атома). Макроскопически изотропны (например, стекла). [cite: 1, 3]
    \item \textbf{Квазикристаллы:} Обладают дальним порядком, но не имеют трансляционной симметрии (запрещенные для классической кристаллографии оси симметрии, например, 5-го порядка). 
\end{itemize}

\subsection*{2. Классификация по электрофизическим свойствам}
Основным критерием является величина удельного электрического сопротивления $\rho$ или проводимости $\sigma = 1/\rho$. Ашкрофт и Мермин выделяют следующие группы: [cite: 2, 3]

\begin{itemize}
    \item \textbf{Металлы:} 
    \begin{itemize}
        \item Имеют высокую концентрацию свободных носителей ($n \sim 10^{22} \text{--} 10^{23} \text{ см}^{-3}$). 
        \item Сопротивление растет с температурой: $\rho(T) \propto T$ при высоких температурах. 
        \item Удельное сопротивление крайне мало: $\rho \sim 10^{-8} \text{ Ом}\cdot\text{м}$. 
    \end{itemize}
    
    \item \textbf{Диэлектрики (изоляторы):}
    \begin{itemize}
        \item Характеризуются полностью заполненными энергетическими зонами и большой шириной запрещенной зоны $E_g > 3 \text{ эВ}$. 
        \item Проводимость практически отсутствует при нормальных условиях ($\rho > 10^{10} \text{ Ом}\cdot\text{м}$). 
    \end{itemize}

    \item \textbf{Полупроводники:}
    \begin{itemize}
        \item Имеют узкую запрещенную зону ($E_g < 3 \text{ эВ}$). 
        \item Сопротивление \textit{экспоненциально уменьшается} при росте температуры: $\sigma \propto \exp(-E_g / 2k_B T)$. 
    \end{itemize}

    \item \textbf{Полуметаллы:}
    \begin{itemize}
        \item Имеют небольшое перекрытие валентной зоны и зоны проводимости, что дает малую, но отличную от нуля концентрацию носителей при $T = 0 \text{ К}$ (например, висмут). [cite: 1, 3]
    \end{itemize}
\end{itemize}

\subsection*{Сводная таблица типичных значений $\rho$ при 300 K}
\begin{center}
\begin{tabular}{|l|c|r|}
\hline
Тип вещества & Удельное сопротивление $\rho$, Ом$\cdot$м & Пример \\
\hline
Металлы & $10^{-8} \div 10^{-6}$ & Cu, Al, Ag \\
Полупроводники & $10^{-5} \div 10^{7}$ & Si, Ge, GaAs \\
Диэлектрики & $10^{8} \div 10^{18}$ & Кварц, Алмаз \\
\hline
\end{tabular}
\end{center}

\section{Электро- и теплопроводность твёрдых тел}

Данный раздел рассматривает феноменологическое описание переноса заряда и энергии в твердых телах.

\subsection*{i. Закон Ома}
В дифференциальной форме закон Ома устанавливает линейную связь между плотностью тока $\vec{j}$ и напряженностью приложенного электрического поля $\vec{E}$:
$$ \vec{j} = \sigma \vec{E}, $$
где $\sigma$ — удельная электрическая проводимость[cite: 3]. Также часто используется удельное сопротивление $\rho = 1/\sigma$[cite: 3].

\subsection*{ii. Температурная зависимость удельного электросопротивления}
Для большинства металлов в широком диапазоне температур наблюдается характерная зависимость:
\begin{itemize}
    \item \textbf{При высоких температурах ($T \gg \Theta_D$):} сопротивление линейно растет с температурой $\rho(T) \propto T$ из-за рассеяния электронов на фононах[cite: 3, 27].
    \item \textbf{При низких температурах ($T \to 0$):} согласно правилу Матиссена, полное сопротивление складывается из остаточного сопротивления на примесях $\rho_0$ и температурно-зависимой части $\rho_{id}(T) \propto T^5$ (закон Блоха-$T^5$)[cite: 27].
\end{itemize}

\subsection*{iii. Закон Фурье}
Перенос тепла в твердом теле описывается законом Фурье, согласно которому плотность теплового потока $\vec{q}$ пропорциональна градиенту температуры:
$$ \vec{q} = -\kappa \nabla T, $$
где $\kappa$ — коэффициент теплопроводности[cite: 6, 9]. Отрицательный знак указывает на то, что тепло течет от горячих областей к холодным[cite: 9].

\subsection*{iv. Коэффициент теплопроводности}
В твердых телах общая теплопроводность $\kappa$ складывается из вклада электронной подсистемы $\kappa_e$ и вклада кристаллической решетки (фононов) $\kappa_{ph}$:
$$ \kappa = \kappa_e + \kappa_{ph} $$
В хороших металлах доминирует электронный вклад $\kappa_e$, в то время как в диэлектриках тепло переносится исключительно фононами[cite: 6, 10, 15].

\subsection*{v. Закон Видемана-Франца для металлов}
Эмпирический закон, связывающий электронную теплопроводность $\kappa$ и электрическую проводимость $\sigma$ металла:
$$ \frac{\kappa}{\sigma T} = L, $$
где $L$ — число Лоренца[cite: 11, 34]. 
Экспериментально установлено, что для большинства металлов при комнатной температуре $L \approx 2.44 \times 10^{-8} \text{ Вт}\cdot\text{Ом}/\text{К}^2$[cite: 11]. Это свидетельствует о том, что в металлах одни и те же носители (электроны) отвечают за перенос как заряда, так и тепла[cite: 15].

\section{Модель свободных электронов Друде}

Модель Друде (1900 г.) — это классическая кинетическая теория электронов в металлах, основанная на представлении об электронном газе[cite: 1, 12, 13].

\subsection*{i. Электроны проводимости как идеальный газ классических частиц}
\begin{itemize}
    \item Электроны проводимости рассматриваются как классический идеальный газ, подчиняющийся статистике Максвелла-Больцмана[cite: 1, 13].
    \item Валентные атомы металла полностью ионизированы, и их электроны становятся свободными носителями заряда[cite: 1, 13].
    \item Между столкновениями электроны движутся прямолинейно, а их взаимодействие друг с другом и с ионами решётки (за исключением мгновенных столкновений) игнорируется[cite: 1, 13].
\end{itemize}

\subsection*{ii. Электрон-ионные столкновения и среднее время свободного пробега}
\begin{itemize}
    \item Основным механизмом изменения импульса электрона являются столкновения с ионами решётки[cite: 1, 14].
    \item Вводится параметр $\tau$ — \textit{время релаксации} (среднее время между столкновениями)[cite: 1, 14].
    \item Вероятность столкновения за малый интервал времени $dt$ полагается равной $dt/\tau$[cite: 1, 14].
    \item Среднее время свободного пробега связано со средней скоростью теплового движения $\langle v \rangle$ соотношением $l = \langle v \rangle \tau$[cite: 1, 14].
\end{itemize}

\subsection*{iii. Расчёт статической электропроводности и теплопроводности}
В рамках элементарной кинетической теории получены ключевые макроскопические коэффициенты[cite: 1, 15]:

\textbf{1. Статическая электропроводность ($\sigma$):}
Уравнение движения электрона в поле $\vec{E}$ с учётом силы трения:
$$ m \frac{d\vec{v}}{dt} = -e\vec{E} - \frac{m\vec{v}}{\tau} $$
В стационарном случае ($d\vec{v}/dt = 0$) дрейфовая скорость $\vec{v}_d = -\frac{e\vec{E}\tau}{m}$. Плотность тока $\vec{j} = -en\vec{v}_d = \frac{ne^2\tau}{m}\vec{E}$. Отсюда:
$$ \sigma = \frac{ne^2\tau}{m} $$
где $n$ — концентрация электронов, $e$ — заряд, $m$ — масса электрона[cite: 1, 15].

\textbf{2. Теплопроводность ($\kappa$):}
Используя результат кинетической теории газов для коэффициента теплопроводности $\kappa = \frac{1}{3} v^2 \tau c_v$, где $c_v = \frac{3}{2} n k_B$ — классическая теплоёмкость, Друде получил:
$$ \kappa = \frac{1}{2} n v^2 \tau k_B = \frac{3}{2} \frac{nk_B^2 T \tau}{m} $$
(Примечание: в классическом расчёте допущена ошибка в коэффициенте 2, которую позже исправил Зоммерфельд) [cite: 1, 15].

\section{Понятие о термоэлектрических эффектах}

Термоэлектрические эффекты описывают взаимосвязь между электрическим током и переносом тепла в проводниках.

\subsection*{i. Эффект Зеебека}
Эффект Зеебека заключается в возникновении термоэлектродвижущей силы (термо-э.д.с.) в замкнутой цепи, состоящей из разнородных проводников, контакты между которыми поддерживаются при разных температурах[cite: 16, 17, 31].

\begin{itemize}
    \item Если вдоль проводника существует градиент температуры $\nabla T$, то носители заряда на «горячем» конце обладают большей кинетической энергией и диффундируют к «холодному» концу.
    \item Это приводит к разделению зарядов и возникновению внутреннего электрического поля $\vec{E}$, которое в разомкнутой цепи компенсирует поток носителей[cite: 31, 33].
    \item Соотношение определяется формулой:
    $$ \vec{E} = Q \nabla T $$
    где $Q$ (или $S$) — коэффициент Зеебека (дифференциальная термо-э.д.с.)[cite: 31].
\end{itemize}

\subsection*{ii. Эффект Пельтье}
Эффект Пельтье является обратным по отношению к эффекту Зеебека: при прохождении электрического тока через контакт двух различных проводников, в зависимости от направления тока, происходит выделение или поглощение тепла[cite: 16, 18].

\begin{itemize}
    \item Количество выделяемого/поглощаемого тепла $dQ$ пропорционально протекающему заряду $dq$:
    $$ dQ = \Pi_{ab} \, dq = (\Pi_a - \Pi_b) \, dq $$
    где $\Pi_{ab}$ — коэффициент Пельтье для контакта проводников $a$ и $b$[cite: 18].
    \item Физическая причина заключается в том, что средняя энергия электронов, участвующих в переносе тока, различна в разных материалах. При переходе через контакт электрон либо отдает избыточную энергию решетке (нагрев), либо забирает её (охлаждение)[cite: 31, 33].
\end{itemize}

\subsection*{Связь между коэффициентами (Соотношение Томсона)}
Коэффициенты Зеебека и Пельтье связаны вторым соотношением Кельвина (Томсона), вытекающим из принципа симметрии кинетических коэффициентов Онзагера[cite: 28, 31]:
$$ \Pi = Q T $$
где $T$ — абсолютная температура[cite: 31].

\section{Квантовая теория свободных электронов}

Квантовая теория, разработанная Зоммерфельдом, учитывает волновые свойства электронов и принцип запрета Паули. [cite: 19, 22]

\subsection*{i-ii. Принцип Паули и структура волновой функции}
\begin{itemize}
    \item Электроны являются \textbf{фермионами} (частицами с полуцелым спином $s=1/2$) и описываются антисимметричной волновой функцией. [cite: 20, 21]
    \item \textbf{Принцип запрета Паули:} в каждом квантовом состоянии может находиться не более одного электрона. [cite: 21]
    \item Волновая функция системы $N$ невзаимодействующих электронов представляется в виде определителя Слэтера из одноэлектронных функций. [cite: 21]
\end{itemize}

\subsection*{iii. Модель Зоммерфельда и граничные условия}
Электроны рассматриваются как частицы в потенциальном ящике объемом $V = L^3$. [cite: 22]
\begin{itemize}
    \item Применяются \textbf{периодические граничные условия Борна-Кармана}: $\psi(x+L, y, z) = \psi(x, y, z)$. [cite: 22]
    \item Решением уравнения Шрёдингера являются плоские волны $\psi_{\vec{k}}(\vec{r}) = \frac{1}{\sqrt{V}} e^{i\vec{k}\cdot\vec{r}}$ с энергией $\varepsilon(\vec{k}) = \frac{\hbar^2 k^2}{2m}$. [cite: 22, 23]
\end{itemize}

\subsection*{iv. Свойства основного состояния (при $T=0$)}
В основном состоянии электроны заполняют все уровни с наименьшей энергией внутри \textbf{сферы Ферми} в $k$-пространстве. [cite: 23]
\begin{itemize}
    \item \textbf{Волновой вектор Ферми:} $k_F = (3\pi^2 n)^{1/3}$, где $n = N/V$. [cite: 23]
    \item \textbf{Энергия Ферми:} $\varepsilon_F = \frac{\hbar^2 k_F^2}{2m}$. [cite: 23]
    \item \textbf{Плотность состояний} (количество уровней на единицу энергии): 
    $$g(\varepsilon) = \frac{m}{\pi^2 \hbar^3} \sqrt{2m\varepsilon} = \frac{3}{2} \frac{n}{\varepsilon_F} \sqrt{\frac{\varepsilon}{\varepsilon_F}}$$ [cite: 23]
\end{itemize}

\subsection*{v-vi. Термодинамические свойства при $T > 0$}
\begin{itemize}
    \item Вероятность заполнения уровня энергией $\varepsilon$ дается \textbf{распределением Ферми-Дирака}:
    $$f(\varepsilon) = \frac{1}{e^{(\varepsilon-\mu)/k_B T} + 1}$$ [cite: 25]
    \item \textbf{Температура Ферми:} $T_F = \varepsilon_F / k_B$. Для металлов $T_F \sim 10^4$ К, поэтому при комнатных температурах электронный газ является \textit{вырожденным}. [cite: 26]
    \item \textbf{Теплоемкость:} В отличие от классической модели ($C_v = \frac{3}{2}nk_B$), квантовая теория предсказывает линейную зависимость от $T$, так как в возбуждении участвуют только электроны вблизи поверхности Ферми:
    $$C_{el} = \frac{\pi^2}{2} n k_B \left( \frac{T}{T_F} \right)$$ [cite: 26]
\end{itemize}

%%%%%%%%%%%%%%%%%%%%%%%%%%%%%%%%%%%%%%%%%%%%%%%%%%%%%%%%%%%%%%%%%%%%%%%%%%%%%

\section{Теория металлической проводимости Зоммерфельда}

В рамках теории Зоммерфельда классическое описание Друде дополняется квантовой статистикой и кинетическим уравнением Больцмана.

\subsection*{i. Квазиклассическое уравнение Больцмана}
Для описания неравновесного состояния электронного газа используется функция распределения $f(\vec{r}, \vec{k}, t)$[cite: 31]. Уравнение Больцмана в приближении времени релаксации имеет вид:
$$ \frac{\partial f}{\partial t} + \vec{v} \cdot \nabla_{\vec{r}} f + \vec{F} \cdot \frac{1}{\hbar} \nabla_{\vec{k}} f = -\frac{f - f_0}{\tau} $$
где $f_0$ — равновесное распределение Ферми-Дирака, а $\vec{F} = -e(\vec{E} + \frac{1}{c}[\vec{v} \times \vec{H}])$ — сила Лоренца[cite: 31, 32]. Условие применимости уравнения: внешние поля должны медленно меняться на расстояниях порядка длины свободного пробега и временах порядка времени релаксации[cite: 28, 31, 32].

\subsection*{ii-iii. Интраграл столкновений и транспортное время}
\begin{itemize}
    \item \textbf{Интеграл столкновений} учитывает принцип Паули: электрон может рассеяться из состояния $\vec{k}$ в $\vec{k}'$ только если последнее свободно[cite: 29].
    \item \textbf{Принцип детального баланса} утверждает, что в равновесии число переходов $\vec{k} \to \vec{k}'$ равно числу переходов $\vec{k}' \to \vec{k}$[cite: 29].
    \item \textbf{Транспортное время релаксации} $\tau_{tr}$ учитывает угол рассеяния $\theta$ через множитель $(1 - \cos\theta)$, что эффективно подавляет вклад малоуглового рассеяния в сопротивление[cite: 30].
\end{itemize}

\subsection*{iv. Электропроводность и теплопроводность}
При упругом рассеянии на дефектах и ионах в вырожденном электронном газе ток переносят только электроны вблизи поверхности Ферми[cite: 31, 32].
\begin{itemize}
    \item \textbf{Статическая проводимость:} $\sigma = \frac{ne^2\tau}{m}$ (формально совпадает с Друде, но $\tau$ берется при энергии $\varepsilon_F$)[cite: 31, 32].
    \item \textbf{Теплопроводность:} $\kappa = \frac{\pi^2}{3} \left( \frac{k_B}{e} \right)^2 \sigma T$. В отличие от модели Друде, здесь используется квантовая теплоемкость[cite: 31, 32].
    \item \textbf{Дифференциальная термо-э.д.с. (S):} пропорциональна $T$ и определяется логарифмической производной проводимости по энергии: $S = -\frac{\pi^2}{3e} k_B^2 T \frac{d \ln \sigma(\varepsilon)}{d \varepsilon} \big|_{\varepsilon_F}$[cite: 31, 32].
\end{itemize}

\subsection*{v-vi. Электрохимический потенциал и число Лоренца}
\begin{itemize}
    \item \textbf{Электрохимический потенциал} $\bar{\mu} = \mu - e\phi$ является истинным драйвером тока в неоднородных системах[cite: 33].
    \item \textbf{Число Лоренца} в квантовой теории имеет универсальное теоретическое значение:
    $$ L = \frac{\kappa}{\sigma T} = \frac{\pi^2}{3} \left( \frac{k_B}{e} \right)^2 \approx 2.44 \times 10^{-8} \, \text{Вт}\cdot\text{Ом}/\text{К}^2 $$
    Это значение гораздо лучше согласуется с экспериментом, чем классическое значение Друде[cite: 34].
\end{itemize}

\section{Магнетизм электронного газа}

Магнитные свойства электронного газа определяются наличием у электронов собственного (спинового) и орбитального магнитных моментов, а также их реакцией на внешнее магнитное поле $\vec{H}$.

\subsection*{i. Собственный и орбитальный магнитные моменты}
\begin{itemize}
    \item \textbf{Собственный (спиновый) магнитный момент:} Обусловлен наличием у электрона спина $\vec{s}$[cite: 35]. Он направлен противоположно вектору спина: 
    $$\vec{\mu}_s = -g \mu_B \vec{s} / \hbar$$ 
    где $g \approx 2$ — g-фактор свободного электрона, а $\mu_B = \frac{e\hbar}{2mc}$ — магнетон Бора[cite: 35].
    \item \textbf{Орбитальный магнитный момент:} Возникает при движении электрона по замкнутой или искривленной траектории под действием силы Лоренца[cite: 35]. В квантовомеханическом описании он связан с оператором орбитального углового момента $\vec{L}$[cite: 35].
\end{itemize}

\subsection*{ii. Магнитная восприимчивость}
Магнитная восприимчивость $\chi$ характеризует способность вещества намагничиваться во внешнем поле и определяется как отношение намагниченности $\vec{M}$ (суммарного магнитного момента единицы объема) к напряженности магнитного поля $\vec{H}$[cite: 35]:
$$ \chi = \frac{\partial M}{\partial H} $$

\begin{itemize}
    \item Для \textbf{парамагнетиков} $\chi > 0$: вектор намагниченности сонаправлен с внешним полем[cite: 36].
    \item Для \textbf{диамагнетиков} $\chi < 0$: намагниченность направлена против внешнего поля[cite: 36].
\end{itemize}

\subsection*{iii. Теорема Бора — ван Леевен}
Классическая статистическая механика предсказывает отсутствие магнетизма у системы заряженных частиц[cite: 36]. Согласно теореме Бора — ван Леевен, при любых конечных температурах и в любом магнитном поле статистическая сумма классического газа заряженных частиц не зависит от магнитного поля, следовательно[cite: 36]:
$$ M = -\frac{\partial F}{\partial H} = 0, \quad \chi = 0 $$
Этот результат подчеркивает, что магнетизм является чисто квантовым эффектом: без учета постоянной Планка $\hbar$ (и принципа Паули) существование пара- и диамагнетизма невозможно[cite: 36, 38].

\section{Парамагнетизм свободных электронов (Парамагнетизм Паули)}

Парамагнетизм Паули обусловлен наличием у электронов собственного магнитного момента (спина) и их подчинением статистике Ферми-Дирака.

\subsection*{i. Механизм намагничивания}
Во внешнем магнитном поле $\vec{H}$ энергия электрона со спином, направленным по полю ($\uparrow$) и против поля ($\downarrow$), изменяется на величину $\pm \mu_B H$:
$$ \varepsilon_{\pm} = \varepsilon(\vec{k}) \pm \mu_B H $$
Это приводит к относительному сдвигу энергетических подзон с разными проекциями спина. Чтобы химический потенциал $\mu$ оставался единым для обеих подзон, часть электронов «перетекает» из состояния с энергией, увеличившейся во внешнем поле, в состояние с меньшей энергией. В результате создается избыток электронов со спином, направленным вдоль поля.

\subsection*{ii. Расчет намагниченности и восприимчивости}
Намагниченность $M$ определяется разностью концентраций электронов с противоположными спинами:
$$ M = \mu_B (n_+ - n_-) $$
Для вырожденного электронного газа ($T \ll T_F$) расчет дает:
$$ M \approx \mu_B^2 g(\varepsilon_F) H $$
где $g(\varepsilon_F)$ — плотность состояний на уровне Ферми. Подставляя значение $g(\varepsilon_F) = \frac{3n}{2\varepsilon_F}$, получаем парамагнитную восприимчивость Паули:
$$ \chi_P = \frac{3n\mu_B^2}{2\varepsilon_F} = \frac{n\mu_B^2}{k_B T_F} $$

\subsection*{iii. Свойства парамагнетизма Паули}
\begin{itemize}
    \item \textbf{Слабая температурная зависимость:} В отличие от классического парамагнетизма (закон Кюри $\chi \propto 1/T$), восприимчивость Паули практически не зависит от температуры при $T \ll T_F$. Это объясняется тем, что только электроны в узком слое толщиной $\sim k_B T$ вблизи поверхности Ферми могут изменить ориентацию своего спина.
    \item \textbf{Малость эффекта:} $\chi_P$ в металлах примерно в $T/T_F$ раз меньше, чем предсказывала бы классическая теория при комнатной температуре.
\end{itemize}

\section{Диамагнетизм свободных электронов (Диамагнетизм Ландау)}

В то время как парамагнетизм Паули связан со спином электрона, диамагнетизм Ландау обусловлен квантованием его орбитального движения во внешнем магнитном поле.

\subsection*{i. Квантование орбитального движения (Уровни Ландау)}
В классической физике электрон в магнитном поле $\vec{H} \parallel z$ движется по спирали, а его проекция на плоскость $(xy)$ представляет собой окружность. В квантовой механике это движение квантуется, превращая непрерывный энергетический спектр в набор дискретных уровней — \textbf{уровней Ландау}:
$$ E_n = \frac{\hbar^2 k_z^2}{2m} + \hbar \omega_c \left( n + \frac{1}{2} \right) $$
где $n = 0, 1, 2, \dots$ — орбитальное квантовое число, а $\omega_c = \frac{eH}{mc}$ — циклотронная частота.

\subsection*{ii. Физическая природа диамагнетизма}
Согласно теореме Бора — ван Леевен, в классическом пределе орбитальный магнетизм отсутствует. Однако квантование уровней энергии приводит к тому, что при изменении магнитного поля общая энергия системы изменяется. 
\begin{itemize}
    \item В отличие от парамагнетизма, этот эффект стремится уменьшить внутреннее поле в образце.
    \item Для свободного электронного газа диамагнитный вклад всегда противоположен парамагнитному.
\end{itemize}

\subsection*{iii. Магнитная восприимчивость Ландау}
Для вырожденного электронного газа в слабых полях ($ \hbar \omega_c \ll k_B T $) теоретический расчет (впервые выполненный Л.Д. Ландау в 1930 г.) дает следующее значение диамагнитной восприимчивости:
$$ \chi_L = -\frac{1}{3} \chi_P $$
где $\chi_P$ — парамагнитная восприимчивость Паули.

\subsection*{iv. Полная восприимчивость электронного газа}
Суммарная восприимчивость газа свободных электронов складывается из двух вкладов:
$$ \chi = \chi_P + \chi_L = \chi_P \left( 1 - \frac{1}{3} \right) = \frac{2}{3} \chi_P $$
Таким образом, свободный электронный газ в целом остается \textbf{парамагнитным}, но его восприимчивость на треть меньше из-за эффекта Ландау.

\begin{itemize}
    \item \textit{Примечание:} В реальных металлах из-за отличия эффективной массы электрона $m^*$ от массы свободного электрона $m$, вклад Ландау может значительно усиливаться ($\chi_L \propto (m/m^*)^2$), что иногда делает металл диамагнитным (например, висмут).
\end{itemize}


%%%%%%%%%%%%%%%%%%%%%%%%%%%%%%%%%%%%%%%%%%%%%%%%%%%%%%%%%%%%%%%%%%%%%%%

\chapter{Свойства кристаллических решеток}

\section{Трансляционная симметрия кристаллов. }

\subsection{1. Трансляционная симметрия и решётка Бравэ}Основным свойством идеального кристалла является трансляционная симметрия: при смещении на определенный вектор $\mathbf{R}$ окружение любой точки кристалла остается неизменным.\textbf{Решётка Бравэ} — это бесконечная совокупность точек, положение которых задается вектором:$$\mathbf{R} = n_1 \mathbf{a}_1 + n_2 \mathbf{a}_2 + n_3 \mathbf{a}_3$$где:\begin{itemize}\item $\mathbf{a}_1, \mathbf{a}_2, \mathbf{a}_3$ — \textbf{основные векторы трансляций} (базисные векторы), которые должны быть линейно независимыми.\item $n_1, n_2, n_3$ — любые целые числа.\end{itemize}Решётка Бравэ выглядит одинаково из любой своей точки.
\subsection{2. Типы кубических решёток}В кубической системе выделяют три типа решёток Бравэ (параметр решётки обозначим через $a$, орты осей — $\hat{x}, \hat{y}, \hat{z}$):\begin{enumerate}\item \textbf{Простая кубическая (ПК / SC):} Точки только в вершинах куба.$$\mathbf{a}_1 = a\hat{x}, \quad \mathbf{a}_2 = a\hat{y}, \quad \mathbf{a}_3 = a\hat{z}$$\item \textbf{Объёмно-центрированная кубическая (ОЦК / BCC):} Точки в вершинах и в центре куба. Примитивные векторы:
$$\mathbf{a}_1 = a\hat{x}, \quad \mathbf{a}_2 = a\hat{y}, \quad \mathbf{a}_3 = \frac{a}{2}(\hat{x} + \hat{y} + \hat{z}) \quad \text{(один из вариантов)}$$
Симметричный набор: $\mathbf{a}_1 = \frac{a}{2}(\hat{y} + \hat{z} - \hat{x})$, $\mathbf{a}_2 = \frac{a}{2}(\hat{z} + \hat{x} - \hat{y})$, $\mathbf{a}_3 = \frac{a}{2}(\hat{x} + \hat{y} - \hat{z})$.

\item \textbf{Гранецентрированная кубическая (ГЦК / FCC):} Точки в вершинах и центрах всех граней. Примитивные векторы:
$$\mathbf{a}_1 = \frac{a}{2}(\hat{y} + \hat{z}), \quad \mathbf{a}_2 = \frac{a}{2}(\hat{z} + \hat{x}), \quad \mathbf{a}_3 = \frac{a}{2}(\hat{x} + \hat{y})$$
\end{enumerate}ShutterstockОткрыть
\subsection{3. Типы ячеек}\begin{itemize}\item \textbf{Примитивная (элементарная) ячейка:} Область пространства, которая при трансляциях на векторы решётки Бравэ заполняет весь объем, не перекрываясь, и содержит ровно \textbf{одну} точку решётки. Её объем равен $V_p = |\mathbf{a}_1 \cdot (\mathbf{a}_2 \times \mathbf{a}_3)|$.\item \textbf{Условная ячейка:} Выбирается больше примитивной для наглядного отображения симметрии (например, куб в ОЦК или ГЦК). Она содержит целое число точек решётки (в ОЦК — 2, в ГЦК — 4).
Трансляционная симметрия кристаллов. 
\item \textbf{Ячейка Вигнера — Зейтца:} Особый тип примитивной ячейки, обладающий полной симметрией решётки. 
\textit{Построение:} вокруг выбранной точки проводятся отрезки ко всем ближайшим соседям. Через середины этих отрезков проводятся перпендикулярные плоскости. Ячейка — это наименьший многогранник, ограниченный этими плоскостями.
\end{itemize}
\subsection{4. Кристаллические структуры и решётки с базисом}Не все кристаллы являются чистыми решётками Бравэ (например, алмаз или соль NaCl).\textbf{Кристаллическая структура} — это решётка Бравэ, в каждой точке которой помещена группа атомов, называемая \textbf{базисом}.Положение любого атома в кристалле:$$\mathbf{r} = \mathbf{R}_n + \mathbf{\tau}_j$$где $\mathbf{R}_n$ — вектор решётки Бравэ, а $\mathbf{\tau}_j$ — вектор, задающий положение $j$-го атома внутри базиса относительно узла решётки.\textbf{Примеры:}\begin{itemize}\item \textbf{Структура алмаза:} ГЦК решётка с базисом из двух атомов в позициях $(0,0,0)$ и $(\frac{a}{4}, \frac{a}{4}, \frac{a}{4})$.\item \textbf{Структура NaCl:} ГЦК решётка с базисом: Na в $(0,0,0)$ и Cl в $(\frac{a}{2}, 0, 0)$.\end{itemize}

\end{document}