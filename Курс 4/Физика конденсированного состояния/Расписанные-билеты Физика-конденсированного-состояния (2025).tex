\documentclass[a4paper,12pt]{report}

\usepackage[utf8]{inputenc}
\usepackage[russian]{babel}
\usepackage{amsmath}
\usepackage{amssymb}
\usepackage{amsfonts}
\usepackage{geometry}
\usepackage{physics}
\usepackage{slashed}
\usepackage{cancel}
\usepackage{tikz}
\usepackage[hidelinks]{hyperref}
\usepackage{tocloft}

\geometry{top=2cm, bottom=2cm, left=2cm, right=2cm}

\title{Билеты по курсу\\<<{\it Физика конденсированного состояния}>>}
\author{Google Gemini 3 Pro \and Андрей Можаров}

\renewcommand{\thechapter}{Билет~№\arabic{chapter}}
\renewcommand{\thesection}{\arabic{section}}
\renewcommand{\theequation}{\arabic{equation}}
\renewcommand{\cftchapnumwidth}{7em}

\makeatletter

\renewcommand{\@makechapterhead}{
{\parindent=0pt \centering\newpage \normalfont\Huge\bfseries
\center{\thechapter} \par
\nopagebreak \vspace{1cm} } }

\makeatother

\begin{document}

\maketitle
\tableofcontents

\chapter{}
\section{Принцип запрета Паули и свойства идеального газа свободных
электронов в основном состоянии.}

\subsection{Введение: Модель свободных электронов (Квантовая теория Зоммерфельда)}

В отличие от классической теории Друде,
теория Зоммерфельда рассматривает электроны как {\bf квантовый газ},
подчиняющийся статистике Ферми-Дирака.
\begin{itemize}
    \item {\bf «Свободные»} означает, что потенциальная энергия взаимодействия электрона с ионами решетки и другими электронами полагается равной нулю ($U = 0$).
    \item Электроны находятся в потенциальном ящике объемом $V$ (кристалл).
\end{itemize}

\subsection{Принцип запрета Паули}

Это фундаментальное положение, определяющее поведение электронной системы.
\begin{itemize}
    \item {\bf Суть:} Электроны являются фермионами (спин $s = 1/2$).
    Согласно принципу Паули, в одном квантовом состоянии не может находиться более одного фермиона.
    \item {\bf Квантовое состояние:} Характеризуется волновым вектором $\vec{k}$
    и проекцией спина $\sigma$ ($\uparrow$ или $\downarrow$).
    Следовательно, каждое пространственное состояние $\vec{k}$
    может быть занято {\bf не более чем двумя} электронами
    (с противоположными спинами).
\end{itemize}

\subsection{Основное состояние ($T = 0$)}

«Основное состояние» означает состояние системы при абсолютном нуле температуры.
В классической физике энергия всех частиц была бы равна нулю.
В квантовой физике из-за принципа Паули это невозможно:
электроны вынуждены заполнять энергетические уровни, начиная с нижнего,
"этаж за этажом".

\subsubsection{Сфера Ферми и волновой вектор Ферми}

Разрешенные значения волнового вектора  определяются граничными условиями Борна-Кармана (периодическими):
$$
    k_{x, y, z} = \dfrac{2\pi}{L}n_{x, y, z} \hspace{2em}\text{где }n_i\text{ --- целые числа}
$$
В $k$-пространстве одно разрешенное значение $\vec{k}$
занимает объем $(2\pi L)^3 = 8\pi^3/V$.

При $T = 0$ $N$ электронов заполняют сферу в пространстве импульсов
(сферу Ферми) радиуса $k_F$.
Объем этой сферы $\Omega = 4/3 \pi k_F^3$.

Число электронов $N$ в объеме $V$ связано с $k_F$ формулой
(учитываем фактор 2 за счет спина):
$$
    N = 2 \cdot \dfrac{V_\text{сферы}}{V_\text{на одно состояние}} = 2 \cdot \dfrac{4/3 \pi k_F^3}{(2\pi)^3/V} = \dfrac{Vk_F^3}{3\pi^2}
$$
Отсюда выводится важнейшее соотношение для концентрации электронов $n = N/V$:
$$
    n = \dfrac{k_F^3}{3\pi^2} \Rightarrow k_F = (3\pi^2n)^{1/3}
$$

\subsubsection{Энергия Ферми}

Энергия электрона с волновым вектором $\vec{k}$ равна:
$$
    \varepsilon(\vec{k}) = \dfrac{\hbar^2k^2}{2m}
$$
Максимальная энергия занятого состояния при $T = 0$ называется {\bf энергией Ферми}:
$$
    \varepsilon_F = \dfrac{\hbar^2k_F^2}{2m} = \dfrac{\hbar^2}{2m}(3\pi^2n)^{2/3}
$$
{\it Физический смысл:} Это химический потенциал системы при $T = 0$.
Все состояния с $\varepsilon < \varepsilon_F$ заняты,
а с $\varepsilon > \varepsilon_F$ свободны.

\noindent {\it Порядок величины:} Для металлов $\varepsilon_F$ составляет несколько
электрон-вольт (1.5–10 эВ), что соответствует гигантской «температуре Ферми»
$T_F = \varepsilon_F / k_B \approx 10^4 - 10^5$ К.

\subsubsection{Полная и средняя энергия}

Даже при абсолютном нуле электронный газ обладает огромной энергией.
Полная энергия $E$ получается суммированием энергий всех электронов
внутри сферы Ферми (интегрированием по слоям сферы):
$$
    E = 2 \sum_{k<k_F} \dfrac{\hbar^2 k^2}{2m}
$$
В Ашкрофте-Мермине (Том 1, стр. 47) приводится знаменитый результат для средней энергии на один электрон:
$$
    \dfrac{E}{N} = \dfrac{3}{5}\varepsilon_F
$$
Это означает, что средняя энергия электрона составляет 60$\%$ от максимальной.

\subsubsection{Давление вырожденного газа}

Из-за высокой кинетической энергии электронный газ оказывает давление
на стенки кристалла, даже если температура равна нулю.
Это давление (давление вырождения) можно получить
из термодинамического соотношения $P = -(\partial E / \partial V)_N$.
$$
    P = \dfrac{2}{3} \dfrac{E}{V} = \dfrac{2}{3} n \cdot \dfrac{3}{5}\varepsilon_F = \dfrac{2}{5} n \varepsilon_F
$$
Именно это давление удерживает кристалл от сжатия (противодействует притяжению ионов).

\section{Теорема о связи семейств атомных плоскостей с векторами обратной
решётки.}

\subsection{Основные определения}
\begin{itemize}
    \item \textbf{Обратная решетка:} Множество векторов $\mathbf{K}$, удовлетворяющих условию $e^{i\mathbf{K}\cdot\mathbf{R}} = 1$ для любого вектора прямой решетки Бравэ $\mathbf{R}$.
    \item \textbf{Семейство атомных плоскостей:} Набор параллельных, равноотстоящих плоскостей, которые содержат все узлы решетки Бравэ.
    \item \textbf{Индексы Миллера $(h, k, l)$:} Три целых числа (взаимно простых), определяющих ориентацию плоскости. Плоскость пересекает оси примитивных векторов $\mathbf{a}_1, \mathbf{a}_2, \mathbf{a}_3$ в точках $x_1 \mathbf{a}_1, x_2 \mathbf{a}_2, x_3 \mathbf{a}_3$, где отрезки отсечения обратно пропорциональны индексам:
    $$ x_1 : x_2 : x_3 = \frac{1}{h} : \frac{1}{k} : \frac{1}{l} $$
\end{itemize}

\subsection{Формулировка теоремы}
Для любого семейства плоскостей решетки, отстоящих друг от друга на расстояние $d$ и определяемых индексами Миллера $(h, k, l)$, существует вектор обратной решетки $\mathbf{K}_{hkl} = h\mathbf{b}_1 + k\mathbf{b}_2 + l\mathbf{b}_3$, такой что:
\begin{enumerate}
    \item Вектор $\mathbf{K}_{hkl}$ перпендикулярен плоскостям этого семейства.
    \item Длина вектора $|\mathbf{K}_{hkl}|$ связана с межплоскостным расстоянием формулой:
    $$ d_{hkl} = \frac{2\pi}{|\mathbf{K}_{hkl}|} $$
\end{enumerate}

\subsection{Доказательство}

\subsubsection{Доказательство перпендикулярности ($\mathbf{K} \perp \text{плоскости}$)}

Рассмотрим плоскость, проходящую через точки пересечения с осями: $\mathbf{a}_1/h$, $\mathbf{a}_2/k$, $\mathbf{a}_3/l$. Построим два вектора, лежащих в этой плоскости:
$$ \mathbf{u} = \frac{\mathbf{a}_1}{h} - \frac{\mathbf{a}_2}{k}, \quad \mathbf{v} = \frac{\mathbf{a}_1}{h} - \frac{\mathbf{a}_3}{l} $$
Вычислим скалярное произведение вектора обратной решетки $\mathbf{K} = h\mathbf{b}_1 + k\mathbf{b}_2 + l\mathbf{b}_3$ с вектором $\mathbf{u}$, используя соотношение $\mathbf{b}_i \cdot \mathbf{a}_j = 2\pi \delta_{ij}$:
$$ \mathbf{K} \cdot \mathbf{u} = (h\mathbf{b}_1 + k\mathbf{b}_2 + l\mathbf{b}_3) \cdot \left( \frac{\mathbf{a}_1}{h} - \frac{\mathbf{a}_2}{k} \right) $$
$$ = h\frac{\mathbf{b}_1 \cdot \mathbf{a}_1}{h} - k\frac{\mathbf{b}_2 \cdot \mathbf{a}_2}{k} = 2\pi - 2\pi = 0 $$
Аналогично $\mathbf{K} \cdot \mathbf{v} = 0$. Так как $\mathbf{K}$ перпендикулярен двум не коллинеарным векторам в плоскости, он перпендикулярен самой плоскости.

\subsubsection{Доказательство формулы для расстояния $d$}

Расстояние от начала координат до ближайшей плоскости (являющееся также межплоскостным расстоянием $d$) равно проекции любого вектора, соединяющего начало координат с точкой на плоскости, на направление нормали $\mathbf{n}$.
$$ \mathbf{n} = \frac{\mathbf{K}}{|\mathbf{K}|} $$
В качестве точки на плоскости выберем пересечение с осью $\mathbf{a}_1$: $\mathbf{r} = \mathbf{a}_1/h$.
$$ d = \mathbf{n} \cdot \mathbf{r} = \frac{\mathbf{K}}{|\mathbf{K}|} \cdot \frac{\mathbf{a}_1}{h} $$
Раскрывая $\mathbf{K}$:
$$ d = \frac{(h\mathbf{b}_1 + k\mathbf{b}_2 + l\mathbf{b}_3) \cdot \mathbf{a}_1}{h |\mathbf{K}|} = \frac{h(\mathbf{b}_1 \cdot \mathbf{a}_1)}{h |\mathbf{K}|} = \frac{2\pi}{|\mathbf{K}|} $$
\textit{Что и требовалось доказать.}

\chapter{}

\section{Температурное разложение Зоммерфельда. Расчёт удельной
теплоёмкости вырожденного электронного газа.}

\section{Кристаллические структуры и решётки с базисом. Примитивная
(элементарная) ячейка, ячейка Вигнера-Зейтца и условная ячейка.}


\chapter{}
\section{Магнетизм электронного газа. Теорема Бора - Ван Леевен.}

\section{Приближение почти свободных электронов: теория возмущений по
слабому периодическому псевдопотенциалу, поведение уровней
энергии вблизи брэгговских плоскостей.}


\chapter{}
\section{Магнитная восприимчивость больцмановского газа электронов с
учётом их собственного магнитного момента. Закон Кюри.}

\section{Решётка Бравэ и её свойства.}


\chapter{}
\section{Парамагнетизм вырожденного электронного газа, связанный с
существованием собственного магнитного момента у электрона
(парамагнетизм Паули).}

\section{Гексагональная плотноупакованная структура.}


\chapter{}
\section{Электро- и теплопроводность металлов: закон Ома; температурная
зависимость удельного сопротивления; закон Фурье; коэффициент
теплопроводности; закон Видемана-Франца.}

\section{Граничные условия Борна-Кармана для блоховских электронов и
число разрешённых значений квазиимпульса. Критерии металла и
изолятора.}


\chapter{}
\section{Простая, объёмно-центрированная и гранецентрированная
кубические решётки.}

\section{Статическая электропроводность металлов в рамках квантовой
модели свободных электронов Зоммерфельда.}


\chapter{}
\section{Cтруктуры типа хлорида натрия, алмаза и цинковой обманки.}

\section{Коэффициент теплопроводности металлов в рамках квантовой
модели свободных электронов Зоммерфельда. Закон Видемана-
Франца.}


\chapter{}
\section{Координационное число и коэффициент компактности
(упаковочный множитель). Алгоритм построения различных
плотноупакованных структур.}

\section{Решение уравнения Больцмана в пределе малых градиентов
электрического потенциала и температуры. Транспортное время
свободного пробега электронов.}


\chapter{}
\section{Поворотные оси симметрии. Теорема о симметрии кристаллических
решёток по отношению к поворотам.}

\section{Общая структура и свойства интеграла столкновений. Принцип
детального баланса.}


\chapter{}
\section{Обратная решётка и её свойства. Обратные решётки для г.ц.к. и
о.ц.к. решёток.}

\section{Эффект Зеебека. Оценка дифференциальной термо-э.д.с. металлов в
рамках элементарной кинетической теории Друде.}


\chapter{}
\section{Электропроводность металла под действием нестационарного, но
однородного электрического поля в модели свободных электронов
Друде.}

\section{Разрешённые и запрещённые энергетические зоны в кристаллах.
Отсутствие вклада в электрический ток от полностью заполненных
зон (инертность заполненных зон). Критерии металла и диэлектрика.}


\chapter{}
\section{Теорема Блоха о виде волновой функции электрона в периодическом
потенциале. Квазиимпульс и эго свойства.}

\section{Теория электропроводности металлов Друде. Среднее время и средняя
длина свободного пробега.}


\chapter{}
\section{Определение собственного и орбитального магнитного моментов
электрона. Намагниченность и магнитная восприимчивость
электронного газа. Диамагнетизм и парамагнетизм.}

\section{Теорема о средней скорости блоховского электрона.}


\chapter{}
\section{Плотность одноэлектронных уровней энергии в модели свободных
электронов Зоммерфельда. Расчёт температурной зависимости
химического потенциала для сильно вырожденного электронного газа.}

\section{Геометрические формулировки условий конструктивной
интерференции рентгеновских лучей в кристалле: построения
Бриллюэна и Эвальда.}


\chapter{}
\section{Принцип запрета Паули и волновая функция невзаимодействующих
электронов. Модель свободных электронов Зоммерфельда.}

\section{Экспериментальные методы определения кристаллических структур:
метод Лауэ, метод вращающегося кристалла, порошковый метод
(метод Дебая-Шеррера).}


\chapter{}
\section{Условия конструктивной интерференции рентгеновских лучей в
кристалле в формулировках Брэгга и Лауэ. Доказательство
эквивалентности этих формулировок.}

\section{Распределение Ферми-Дирака. Температура Ферми, химический
потенциал и условие вырождения электронного газа.}


\chapter{}
\section{Теория теплопроводности металлов Друде. Закон Видемана-Франца в
рамках модели Друде.}

\section{Геометрический структурный фактор кристаллических структур.}


\chapter{}
\section{Агрегатные состояния и термодинамические фазы вещества. Фазовые
переходы, критические точки, полиморфизм.}

\section{Расчёт дифференциальной термо-э.д.с. металлов в рамках
кинетического уравнения Больцмана.}


\chapter{}
\section{Атомные плоскости и семейства атомных плоскостей. Индексы
Миллера и их геометрическая интерпретация.}

\section{Свойства энергетического спектра электронов вблизи экстремумов
энергии в зоне Бриллюэна. Тензор обратных эффективных масс
блоховского электрона и его свойства.}


\chapter{}
\section{Количественный критерий принадлежности твёрдого тела к металлу
или диэлектрику. Диэлектрики, полупроводники, металлы и
полуметаллы.}

\section{Эффективные массы блоховских электронов в двузонной модели.}


\end{document}