\documentclass[a4paper, usenames, dvipsnames]{article}

\title{Измерение ёмкости конденсатора}
\author{Бояринцева Н.А. \and Можаров А.Р.}
\date{26 декабря 2023}

\usepackage{cmap}
\usepackage[T2A]{fontenc}
\usepackage[utf8]{inputenc}
\usepackage[russian]{babel}
\usepackage{indentfirst}
\usepackage[hidelinks]{hyperref}
\usepackage{circuitikz}
\usepackage{pgfplots}
\usepackage{amsmath}
\usepackage{subcaption}
\usepackage{comment}
\usepackage[left=3.5cm, right=3.5cm, top=2.5cm, bottom=3cm]{geometry}

\renewcommand{\thesubfigure}{.\arabic{subfigure}}
\DeclareCaptionLabelFormat{My}{Рис. \thefigure #2: }
\captionsetup[subfigure]{labelformat=My}
\renewcommand{\theenumii}{\arabic{enumii}}

\begin{document}

\maketitle

\section*{\centering Теоретическая часть}

В данной лабораторной работе для измерения ёмкости конденсатора используется
измерительный мост (рис. \ref{Измерительный мост}).

\begin{figure}[h]
    \centering
    \begin{circuitikz}[european resistors, american inductors]
        \draw (5.5, 0) node[spdt](Key){};
        \draw (0, 0) to [short, *-] (0.5, 0) to [R, l=$R_1$, -*] (2.5, 2)
        to [C, l=$C_1$, -*] (4.5, 0) to [C, l=$C_x$, -*] (2.5, -2)
        to [R, l=$R_2$, -*] (0.5, 0);
        \draw (4.5, 0) -- (Key.in);
        \draw (0, 0) -- (0, 2.5) -- (6, 2.5) -| (Key.out 1);
        \draw (0, 0) -- (0, -3) to [battery1, l={$\mathcal{E}, r$}] (6, -3) -| (Key.out 2);
        \draw (2.5, 2) to [esource] (2.5, -2);
        \draw (2.5, 0) node [open]{$G$};
        \draw (Key.out 1) node [open, anchor=south east]{S};
        \draw (Key.out 2) node [open, anchor=north east]{T};
        \draw (Key.in) node [open, anchor=south]{K};
        \draw (0, 0) node [open, anchor=south west]{D};
        \draw (2.5, 2) node [open, anchor=south]{A};
        \draw (2.5, -2) node [open, anchor=north]{B};
    \end{circuitikz}
    \caption{Измерительный мост}
    \label{Измерительный мост}
\end{figure}

Здесь $R_1$ и $R_2$ представляют собой магазины сопротивлений.
Конденсатор $C_1$ является эталонным
(его ёмкость известна с достаточно высокой точностью).
Конденсатор $C_x$, соответственно, является неизвестным,
ёмкость которого предстоит измерить.
В качестве измерительного прибора $G$ ({\it нуль-индикатора}) используются
{\it нуль-гальванометр} (обычный гальванометр,
применяемый для регистрации отсутствия тока) или осциллограф.

В положении ключа $T$ происходит зарядка конденсаторов.
В положении $S$, соответственно, разрядка.

\subsection*{Условие баланса моста}

Условием баланса моста называется условие,
при котором напряжение на нуль-индикаторе $U_G$ равно нулю.
Попробуем найти это условие.

Выведем это условие из упрощающих нам жизнь предположений,
что сопротивление измерительного прибора достаточно велико,
чтобы через него тёк ток, и внутреннее сопротивление источника достаточно мало,
чтобы создавать заметное падение напряжения.

Запишем второй закон Кирхгофа для контура DAKTD:
\begin{gather*}
    I_1R_1 + \dfrac{q_1}{C_1} = \mathcal{E}
\end{gather*}
Ввиду наших предположений:
\begin{gather*}
    I_1 = \dfrac{dq_1}{dt}
\end{gather*}
Тогда поделив всё это безобразие на $R_1$ получаем
несложное дифференциальное уравнение:
\begin{gather*}
    \dfrac{dq_1}{dt} + \dfrac{q_1}{R_1C_1} = \dfrac{\mathcal{E}}{R_1}
\end{gather*}
Разделяем переменные:
\begin{gather*}
    \dfrac{dq_1}{dt} = \dfrac{\mathcal{E}}{R_1} - \dfrac{q_1}{R_1C_1} = \dfrac{\mathcal{E}C_1 - q_1}{R_1C_1} \\
    \dfrac{dq_1}{\mathcal{E}C_1 - q_1} = \dfrac{dt}{R_1C_1} \\
    \dfrac{d(q_1 - \mathcal{E}C_1)}{q_1 - \mathcal{E}C_1} = - \dfrac{dt}{R_1C_1} \\
    ln(q_1 - \mathcal{E}C_1) = - \dfrac{1}{R_1C_1}t + const \\
    q_1(t) = \mathcal{E}C_1 + const \cdot exp\left(-\dfrac{t}{R_{1}C_{1}}\right)
\end{gather*}
И, ввиду начальных условий $q_1(0) = 0$, получаем зависимость заряда $q_1$
на конденсаторе $C_1$ от времени:
\begin{gather*}
    q_1(t) = \mathcal{E}C_1\left(1 - exp\left(-\dfrac{t}{R_{1}C_{1}}\right)\right)
\end{gather*}
Тогда напряжение на том же конденсаторе:
\begin{gather*}
    U_1(t) = \mathcal{E}\left(1 - exp\left(-\dfrac{t}{R_{1}C_{1}}\right)\right)
\end{gather*}
Аналогичным образом найдём напряжение на измеряемом конденсаторе:
\begin{gather*}
    U_x(t) = \mathcal{E}\left(1 - exp\left(-\dfrac{t}{R_{2}C_x}\right)\right)
\end{gather*}
Пусть в точке $A$ потенциал $\varphi_A$, точке $B$ потенциал $\varphi_B$,
в точке $K$ потенциал $\varphi_K$, тогда:
\begin{eqnarray*}
    \varphi_K - \varphi_A = U_1(t) & \varphi_K - \varphi_B = U_x(t)
\end{eqnarray*}
Тогда напряжение на нуль-индикаторе:
\begin{gather*}
    U_G(t) = \varphi_A - \varphi_B = (\varphi_K - \varphi_B) - (\varphi_K - \varphi_A) = U_x(t) - U_1(t) \\
    U_G(t) = \mathcal{E}\left(exp\left(-\dfrac{t}{R_2C_x}\right) - exp\left(-\dfrac{t}{R_1C_1}\right)\right)
\end{gather*}
Чтобы выполнялось условие баланса моста:
\begin{gather*}
    U_G(t) = 0 \\
    \mathcal{E}\left(exp\left(-\dfrac{t}{R_2C_x}\right) - exp\left(-\dfrac{t}{R_1C_1}\right)\right) = 0 \\
    exp\left(-\dfrac{t}{R_2C_x}\right) - exp\left(-\dfrac{t}{R_1C_1}\right) = 0 \\
    exp\left(-\dfrac{t}{R_2C_x}\right) = exp\left(-\dfrac{t}{R_1C_1}\right)
\end{gather*}
Прологарифмируем по основанию $e$:
\begin{gather*}
    -\dfrac{t}{R_2C_x} = -\dfrac{t}{R_1C_1}
\end{gather*}
Таким образом условием баланса моста будет:
\begin{gather}
    R_1C_1 = R_2C_x \label{Условие баланса}
\end{gather}
Неизвестную ёмкость можно же будет найти из формулы:
\begin{gather}
    C_x = C_1\dfrac{R_1}{R_2} \label{Ёмкость конденсатора}
\end{gather}

\subsection*{Режимы измерения}

Заметим, что время зарядки конденсатора достаточно не велико
и при использовании мостовой схемы имеет место не моментальный отклик прибора.
В зависимости от включённого в цепь измерительного прибора,
данная лабораторная работа предполагает два метода измерений
с использованием мостовой схемы:
\begin{enumerate}
    \item {\it Безынерционный}.

          Этот метод предполагает использование прибора с достаточным
          быстродействием для регистрации импульсов $U_G(t)$.
          В данной лабораторной работе этим прибором является осциллограф.
    \item {\it Инерционный} (он же {\it баллистический}).

          Этот метод, соответственно, предполагает использование прибора с недостаточным
          быстродействием. В данной лабораторной работе этим прибором является
          высокочувствительный нуль-гальванометр.
\end{enumerate}

\subsection*{Погрешности}

{\it Погрешность измерения} --- отклонение измеряемого значения величины от истинного.
Погрешности можно классифицировать по источнику возникновения:
\begin{enumerate}
    \item {\it Инструментальная}.
    
    Обусловлена несовершенством измерительных приборов.
    \item {\it Методическая}.
    
    Обусловлена неточностью модели физического процесса.
    \item {\it Субъективная}.
    
    Обусловлена личными особенностями экспериментатора (к примеру скорость реакции).
\end{enumerate}
И прочие. По характеру проявления погрешности могут быть:
\begin{enumerate}
    \item {\it Систематическая}.
    
    Остаётся постоянной или закономерно изменяющаяся при
    повторных измерениях одной и той же величины.
    \item {\it Случайная}.
    
    Возникает при некотором случайном событии во время эксперимента.

    (К примеру хлопок дверью соседней лаборатории во время проведения акустического эксперимента)
\end{enumerate}
А по способу выражения:
\begin{enumerate}
    \item {\it Абсолютная}.
    
    Разность истинного значения величины и измеренного.
    \begin{gather*}
        \Delta X_i = X_i - X_0
    \end{gather*}
    Поскольку при проведении эксперимента не всегда бывает
    точно известно истинное значение измеряемой величины,
    то вместо него принимается некоторое среднее значение,
    полученное в ходе эксперимента.
    \begin{gather*}
        \Delta X_i = X_i - \overline{X}
    \end{gather*}
    
    \item {\it Относительная}.
    
    Отношение погрешности величины к самой величине.
    \begin{gather*}
        \delta X = \dfrac{\Delta X_i}{X_i}
    \end{gather*}
\end{enumerate}

Все же измерения можно классифицировать на два класса:
\begin{enumerate}
    \item {\it Прямые измерения}.
    
    Непосредственное измерение значения величины соответствующим прибором.

    (Измерение длины стола рулеткой)
    \item {\it Косвенные измерения}.
    
    Вычисление значения величины по известной зависимости искомой
    величины от известных.

    (Измерение средней скорости: отдельно измеряется пройденное расстояние,
    отдельно время движения)
\end{enumerate}

Погрешности при косвенных измерениях.
Если требуется измерить некоторую величину $f(x_1, x_2, \ldots)$, то:
\begin{enumerate}
    \item При сумме складываются абсолютные погрешности.
    
    Т.е. если $f(x_1, x_2, \ldots) = x_1 + x_2 + \ldots$, то
    \begin{gather*}
        \Delta f = \Delta x_1 + \Delta x_2 + \ldots
    \end{gather*}
    \item При произведении (и частном) складываются относительные погрешности.
    
    Т.е. если $f(x_1, x_2, \ldots) = x_1 \cdot x_2 \cdot \ldots$, то
    \begin{gather*}
        \delta f = \delta x_1 + \delta x_2 + \ldots
    \end{gather*}
    \item При произвольной зависимости имеет место более общая формула:
    \begin{gather*}
        \Delta f = \sqrt{\left(\dfrac{\partial f}{\partial x_1}\Delta x_1\right)^2 +
        \left(\dfrac{\partial f}{\partial x_1}\Delta x_1\right)^2 + \ldots}
    \end{gather*}
    Однако данная формула справедлива только при условии независимости параметров $x_1$, $x_2$, \ldots
\end{enumerate}

\newpage\section*{\centering Практическая часть}

Класс точности магазинов сопротивлений $R_1$ и $R_2$: $0,2$,
т.е. при максимально возможном сопротивлении $R_{max} = 100$ кОм
абсолютная погрешность составляет $\Delta R = 200$ Ом.
Ёмкость эталонного конденсатора $C_1 = 1 \pm 0,002$ мкФ.

Ввиду формулы (\ref{Ёмкость конденсатора}),
относительная формула для погрешности измерения
ёмкости конденсатора будет иметь вид:
\begin{gather*}
  \delta C_x = \delta C_1 + \delta R_2 + \delta R_1
\end{gather*}

\begin{enumerate}
    \item Измерить неизвестную ёмкость $C_x$ прямым измерением (рис. \ref{Прямое измерение}).
          \begin{figure}[h]
              \centering
              \begin{tikzpicture}
                  \begin{axis}[legend pos = north west, y post scale=1, grid=both, x post scale = 1.75, xlabel = {$R_1$, Ом}, ylabel = {$\tau$, с}]
                      \addplot+[error bars/.cd, x explicit, y explicit, x dir=both, y dir=both] coordinates{(1000, 0.001)+-(200, 0.000204)(2000, 0.0019)+-(200, 0.000208)(5000, 0.0048)+-(200, 0.00022)(7500.0, 0.0076)+-(200, 0.00023)(10000, 0.0098)+-(200, 0.00024)(15000, 0.0112)+-(200, 0.00026)(20000, 0.0192)+-(200, 0.00028)};
                      \addlegendentry{Практика}
                      \addplot coordinates{(1000, 0.001)(2000, 0.002)(5000, 0.005)(7500.0, 0.0075)(10000, 0.01)(15000, 0.015)(20000, 0.02)};
                      \addlegendentry{Теория}
                  \end{axis}
              \end{tikzpicture}
              \caption{Прямое измерение}
              \label{Прямое измерение}
          \end{figure}

          Как видно из графика, линейная зависимость $\tau$ от $R_1C_1$
          соблюдается достаточно точно,
          за исключением точки $R_1 = 15$ кОм, в которой, видимо,
          имела место ошибка проведения эксперимента.
    \item Измерить неизвестную ёмкость $C_x$ безынерционным методом (рис. \ref{Безынерционный метод}).

          \begin{figure}[h]
              \centering
              \begin{tikzpicture}
                  \begin{loglogaxis}[xlabel = {$R_1$, Ом}, y post scale=1, ylabel = {$R_2$, Ом}, x post scale=1.75, grid=both]
                      \addplot coordinates{(10000, 2432.0)(1000, 244.66666666666666)(100, 24.566666666666663)(10, 2.4)};
                  \end{loglogaxis}
              \end{tikzpicture}
              \caption{Безынерционный метод}
              \label{Безынерционный метод}
          \end{figure}

          Результат измерения ёмкости данным способом: $C_x = 4,11$ мкФ.

    \item Измерить неизвестную ёмкость $C_x$ инерционным методом (рис. \ref{Инерционный метод}).

          \begin{figure}[h]
              \centering
              \begin{subfigure}{\textwidth}
                  \centering
                  \begin{tikzpicture}
                      \begin{loglogaxis}[xlabel = {$R_1$, Ом}, ylabel = {$R_2$, Ом}, x post scale=1.75, grid=both, y post scale=1]
                          \addplot coordinates{(10000, 2411.6666666666665)(1000, 240.66666666666666)(100, 24.666666666666668)(10, 4.066666666666666)};
                      \end{loglogaxis}
                  \end{tikzpicture}
                  \caption{12 В}
                  \label{12 В}
              \end{subfigure}
              \begin{subfigure}{\textwidth}
                  \centering
                  \begin{tikzpicture}
                      \begin{loglogaxis}[xlabel = {$R_1$, Ом}, ylabel = {$R_2$, Ом}, x post scale=1.75, grid=both, y post scale=1]
                          \addplot coordinates{(1000, 241.0)(100, 24.0)(10, 3.466666666666667)};
                      \end{loglogaxis}
                  \end{tikzpicture}
                  \caption{24 В}
                  \label{24 В}
              \end{subfigure}
              \caption{Инерционный метод}
              \label{Инерционный метод}
          \end{figure}

          Результат измерения ёмкости данным способом:
          при $U = 12$ В $C_x = 3,70$ мкФ, а при $U = 24$ В $C_x = 3,73$ мкФ.
\end{enumerate}

\end{document}