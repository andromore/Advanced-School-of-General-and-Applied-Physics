\documentclass{article}

\title{Диод и триод}
\author{Бояринцева Н.А. \and Можаров А.Р.}
\date{24 декабря 2023}

\usepackage{cmap}
\usepackage[T2A]{fontenc}
\usepackage[utf8]{inputenc}
\usepackage[russian]{babel}
\usepackage{indentfirst}
\usepackage[hidelinks]{hyperref}
\usepackage{circuitikz}
\usepackage{pgfplots}
\usepackage{amsmath}
\usepackage{subcaption}
\usepackage{comment}
\usepackage[left=3.5cm, right=3.5cm, top=2.5cm, bottom=3cm]{geometry}

\begin{document}

\maketitle

\section*{\centering Теоретическая часть}

\subsection*{Диод}

При постоянной температуре анодный ток зависит только от анодного напряжения.
Данная зависимость ($I_\text{а.}(U_\text{а.})$) называется {\it анодной характеристикой диода}.

Зависимость анодного тока от напряжения накала $I_\text{а.}(U_\text{н.})$
лампы называется {\it температурной характеристикой диода}.

{\it Вольт-амперной характеристикой (ВАХ)} элемента называется
зависимость $U(I)$ напряжения на элементе от тока на нём.

{\it Внутреннем сопротивлением} $R_i$ элемента называется отношение приращения напряжения
на элементе приращению тока:
\begin{equation*}
    R_i = \dfrac{dU_\text{а.}}{dI_\text{а.}}
\end{equation*}
(т.е. производная от ВАХ). Т.к. ВАХ диода, вообще говоря,
не линейна, то внутреннее сопротивление зависит от анодного тока
(или анодного напряжения --- в зависимости от задаваемого параметра).

Обратной величиной к внутреннему сопротивлению является {\it крутизна ВАХ} $S$,
которая, в свою очередь, также зависит от анодного тока или анодного напряжения:
\begin{equation*}
    S = \dfrac{dI_\text{а.}}{dU_\text{а.}}
\end{equation*}

Заметим, что построение зависимостей внутреннего сопротивления
и крутизны ВАХ при знании самой ВАХ не составляют особой сложности.

\subsection*{Триод}

У триода, ввиду особенностей его внутреннего строения --- наличия сетки, анодный ток
является функцией как анодного напряжения, так и сеточного.

Крутизна ВАХ триода:
\begin{gather*}
    S = \dfrac{\partial I_a}{\partial U_c} = \left.\dfrac{dI_a}{dU_c}\right|_{U_a = const}
\end{gather*}
Крутизна имеет смысл тангенса угла наклона касательной к сеточной характеристике.

Внутреннее сопротивление триода:
\begin{gather*}
    R_i = \dfrac{\partial U_a}{\partial I_a} = \left.\dfrac{dU_a}{dI_a}\right|_{U_c = const}
\end{gather*}
Внутреннее сопротивление имеет смысл котангенса угла наклона касательной к анодной характеристике.

Статический коэффициент усиления триода:
\begin{gather*}
    \mu = \left.\dfrac{dU_a}{dU_c}\right|_{I_a = const}
\end{gather*}

Найдём связь между крутизной и внутренним сопротивлением.
Для этого вычислим полный дифференциал анодного тока $I_a(U_c, U_a)$.
\begin{gather*}
    dI_a = \dfrac{\partial I_a}{\partial U_c} dU_c + \dfrac{\partial I_a}{\partial U_a} dU_a
\end{gather*}
Если изменять $U_a$ и $U_c$ так, чтобы $I_a = const$ (а следовательно $dI_a = 0$), то,
учитывая также, что изменение сеточного и анодного напряжений имеют различный знак, получим:
\begin{gather*}
    \dfrac{\partial I_a}{\partial U_c} dU_c - \dfrac{\partial I_a}{\partial U_a} dU_a = 0
\end{gather*}
Поделим на $dU_c$ и получим:
\begin{gather*}
    \dfrac{\partial I_a}{\partial U_c} - \dfrac{\partial I_a}{\partial U_a}\dfrac{dU_a}{dU_c} = 0
\end{gather*}
И т.к. $I_a = const$:
\begin{gather*}
    \mu = S R_i
\end{gather*}

Нетрудно показать, что связь крутизны триода в статическом $S_s$
и динамическом $S_d$ режимах выглядит следующим образом:
\begin{gather*}
    S_d = \dfrac{S_s}{1 + \dfrac{R_i}{R_a}}
\end{gather*}
Выражение для коэффициента усиления триода имеет вид:
\begin{gather*}
    K = \dfrac{dU_a}{dU_c}
\end{gather*}
Тогда, заметив, что $dU_a = R_a \cdot dI_a$, получим:
\begin{gather*}
    K = R_a\dfrac{dI_a}{dU_a} = R_a S_d = \dfrac{R_aS_s}{1 + \dfrac{R_i}{R_a}}
    = \dfrac{\mu}{1 + \dfrac{R_i}{R_a}}
\end{gather*}
где $R_a$ --- сопротивление нагрузки, $\mu$ --- коэффициент усиления,
$R_i$ --- внутреннее сопротивление триода.

\section*{\centering Практическая часть}

\begin{enumerate}
    \item Снята воль-амперная характеристика диода (рис. \ref{ВАХ диода}).
          \begin{figure}[h]
              \centering
              \begin{tikzpicture}
                  \begin{axis}[grid=both, scale=1.5, xlabel={$U$, В}, ylabel={$I$, А}]
                      \addplot coordinates{(2, 0.003562)(4, 0.009099000000000001)(6, 0.015094)(8, 0.022261)(10, 0.030033999999999998)(12, 0.037764000000000006)(14, 0.045844)(16, 0.05718)(18, 0.06770999999999999)(20, 0.07867)(22, 0.09014)};
                  \end{axis}
              \end{tikzpicture}
              \caption{ВАХ диода}
              \label{ВАХ диода}
          \end{figure}
          
          Построим зависимость крутизны ВАХ диода от напряжения (рис. \ref{Крутизна ВАХ диода}).
          \begin{figure}[h]
              \centering
              \begin{tikzpicture}
                  \begin{axis}[grid=both, scale=1.5, xlabel={$U$, В}, ylabel={$S$, $\text{Ом}^{-1}$}]
                      \addplot coordinates{(2, 0.0027685000000000006)(4, 0.0029974999999999993)(6, 0.0035835)(8, 0.0038864999999999993)(10, 0.003865000000000004)(12, 0.0040399999999999985)(14, 0.005667999999999999)(16, 0.005264999999999995)(18, 0.005480000000000006)(20, 0.005734999999999997)};
                  \end{axis}
              \end{tikzpicture}
              \caption{Крутизна ВАХ диода}
              \label{Крутизна ВАХ диода}
          \end{figure}
    \item Снято семейство сеточных (рис. \ref{Семейство сеточных характеристик триода})
          и анодных (рис. \ref{Семейтство анодных характеристик триода}) характеристик триода.
          \begin{figure}[h]
              \centering
              \begin{tikzpicture}
                  \begin{axis}[grid=both, scale=1.5, xlabel={$U_c$, В}, ylabel={$I$, А}, legend pos=north west]
                      \addplot coordinates{(-2.5, 0.00044)(-2.0, 0.00103)(-1.5, 0.00185)(-1.0, 0.00278)(-0.5, 0.00387)(0.0, 0.00491)(0.5, 0.006110000000000001)(1.0, 0.0072900000000000005)(1.5, 0.00838)(2.0, 0.00967)(2.5, 0.010750000000000001)(3.0, 0.012240000000000001)(3.5, 0.013300000000000001)(4.0, 0.0147)(4.5, 0.01599)};
                      \addlegendentry{$U_a = 50$ В}
                      \addplot coordinates{(-2.5, 0.0025299999999999997)(-2.0, 0.0035470000000000002)(-1.5, 0.00467)(-1.0, 0.00588)(-0.5, 0.0072900000000000005)(0.0, 0.008539999999999999)(0.5, 0.00991)(1.0, 0.01134)(1.5, 0.012539999999999999)(2.0, 0.013859999999999999)(2.5, 0.01528)(3.0, 0.01684)(3.5, 0.018170000000000002)(4.0, 0.019760000000000003)(4.5, 0.02122)};
                      \addlegendentry{$U_a = 80$ В}
                      \addplot coordinates{(-2.5, 0.00563)(-2.0, 0.0068)(-1.5, 0.00826)(-1.0, 0.009640000000000001)(-0.5, 0.01123)(0.0, 0.01272)(0.5, 0.01428)(1.0, 0.01575)(1.5, 0.01718)(2.0, 0.01884)(2.5, 0.020300000000000002)(3.0, 0.02194)(3.5, 0.02355)(4.0, 0.02539)(4.5, 0.026940000000000002)};
                      \addlegendentry{$U_a = 110$ В}
                  \end{axis}
              \end{tikzpicture}
              \caption{Семейство сеточных характеристик триода}
              \label{Семейство сеточных характеристик триода}
          \end{figure}
          \begin{figure}[h]
              \centering
              \begin{tikzpicture}
                  \begin{axis}[grid=both, scale=1.5, xlabel={$U_a$, В}, ylabel={$I$, А}]
                      \addplot coordinates{(80, 3e-06)(90, 2.5e-05)(100, 9.900000000000001e-05)(110, 0.000263)(120, 0.00059)(130, 0.00103)(140, 0.0016200000000000001)(150, 0.00233)(160, 0.00315)(170, 0.004070000000000001)(180, 0.00509)(190, 0.006200000000000001)(200, 0.00738)(210, 0.00862)(220, 0.009890000000000001)};
                      \addlegendentry{$U_c = -6$ В}
                      \addplot coordinates{(40, 3.3e-05)(50, 0.00016800000000000002)(60, 0.000479)(70, 0.0010400000000000001)(80, 0.00173)(90, 0.00254)(100, 0.00346)(110, 0.00445)(120, 0.00558)(130, 0.00674)(140, 0.00797)(150, 0.00926)(160, 0.01061)(170, 0.012050000000000002)};
                      \addlegendentry{$U_c = -3$ В}
                      \addplot coordinates{(2, 0.00058)(10, 0.00122)(20, 0.00206)(30, 0.0029900000000000005)(40, 0.0039900000000000005)(50, 0.005070000000000001)(60, 0.00628)(70, 0.00751)(80, 0.00878)(90, 0.010130000000000002)(100, 0.01154)(110, 0.01298)(120, 0.014480000000000002)(130, 0.01611)};
                      \addlegendentry{$U_c = 0$ В}
                      \addplot coordinates{(10, 0.00899)(20, 0.01111)(30, 0.013000000000000001)(40, 0.01476)(50, 0.0165)(60, 0.01825)(70, 0.02)(80, 0.021750000000000002)(90, 0.023600000000000003)};
                      \addlegendentry{$U_c = 3$ В}
                  \end{axis}
              \end{tikzpicture}
              \caption{Семейство анодных характеристик триода}
              \label{Семейтство анодных характеристик триода}
          \end{figure}
    \item Снята зависимость коэффициента усиления от частоты (рис. \ref{Зависимость коэффициента усиления триода от частоты входного сигнала})
          \begin{figure}[h]
              \centering
              \begin{tikzpicture}
                  \begin{axis}[grid=both, scale=1.5, xlabel={$\nu$, Гц}, ylabel={$U_\text{вых.}$, В}]
                      \addplot coordinates{(10000, 18.799999999999997)(20000, 16.7)(1000, 19.2)(30000, 14.499999999999998)(50000, 11.0)(80000, 8.0)(60000, 9.499999999999998)(100000, 6.5)(140000, 5.0)(160000, 4.35)(200000, 3.4999999999999996)(300000, 2.4)(400000, 1.9)(600000, 1.3800000000000001)(700000, 1.1199999999999999)(800000, 1.0)};
                  \end{axis}
              \end{tikzpicture}
              \caption{Зависимость коэффициента усиления триода от частоты входного сигнала}
              \label{Зависимость коэффициента усиления триода от частоты входного сигнала}
          \end{figure}
    \item Снята зависимость коэффициента усиления от нагрузки (рис. \ref{Зависимость коэффициента усиления триода от сопротивления нагрузки})
          \begin{figure}[h]
              \centering
              \begin{tikzpicture}
                  \begin{semilogxaxis}[grid=both, scale=1.5, xlabel={$R$, Ом}, ylabel={$U_\text{вых.}$, В}, legend pos=north west]
                      \addplot coordinates{(1800.0, 2.74)(5000, 6.1499999999999995)(18000, 10.299999999999999)(32000, 13.3)(46000, 16.2)(83000, 18.4)(192000, 19.099999999999998)(495000, 18.799999999999997)(1000000, 17.599999999999998)};
                      \addlegendentry{Практический}
                      \addplot coordinates{(1800.0, 3.525866922796675)(5000, 7.500634115409004)(18000, 13.838099571038606)(32000, 16.13160586430276)(46000, 17.250364593240757)(83000, 18.562052144977404)(192000, 19.61473087818697)(495000, 20.147299692032412)(1000000, 20.32388316151203)};
                      \addlegendentry{Теоретический}
                  \end{semilogxaxis}
              \end{tikzpicture}
              \caption{Зависимость коэффициента усиления триода от сопротивления нагрузки}
              \label{Зависимость коэффициента усиления триода от сопротивления нагрузки}
          \end{figure}
\end{enumerate}

\end{document}
