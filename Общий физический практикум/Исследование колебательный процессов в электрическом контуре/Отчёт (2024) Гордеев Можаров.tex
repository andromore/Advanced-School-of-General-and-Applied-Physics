\documentclass[a4paper, usenames, dvipsnames]{article}

\title{Исследование колебательных процессов в колебательном контуре}
\author{Гордеев К.М. \and Можаров А.Р.}
\date{26 февраля 2024}

\usepackage{cmap}
\usepackage[T2A]{fontenc}
\usepackage[utf8]{inputenc}
\usepackage[russian]{babel}
\usepackage{indentfirst}
\usepackage[hidelinks]{hyperref}
\usepackage{circuitikz}
\usepackage{pgfplots}
\usepackage{amsmath}
\usepackage{amssymb}
\usepackage{subcaption}
\usepackage{comment}
\usepackage[left=3.5cm, right=3.5cm, top=2.5cm, bottom=3cm]{geometry}
\usepackage{xcolor}

\renewcommand{\thesubfigure}{.\arabic{subfigure}}
\DeclareCaptionLabelFormat{My}{Рис. \thefigure #2: }
\captionsetup[subfigure]{labelformat=My}
\newcommand{\frc}[2]{\raisebox{2pt}{$#1$}\big/\raisebox{-3pt}{$#2$}}

\begin{document}

\maketitle

\section*{\centering Теоретическая часть}

\subsection*{Базовые понятия}

{\it Колебательным контуром} (последовательным) называется электрическая цепь, состоящая из резистора, конденсатора, индуктивности и
источника, в общем случае, некоторого переменного, не обязательно гармонического, напряжения (рис. \ref{Колебательный контур}).
\begin{figure}[h]
    \centering
    \begin{circuitikz}[european resistors, american inductors]
        \draw (0, 0) to [C, l=$C$] (0, 3) to [R, l=$R$] (3, 3) to [sinusoidal voltage source, l=$\mathcal{E}$] (3, 0) to [L, l=$L$] (0, 0);
    \end{circuitikz}
    \caption{Колебательный контур}
    \label{Колебательный контур}
\end{figure}

Запишем {\it второй закон Кирхгофа} для данной цепи
\begin{gather*}
    U_L + U_R + U_C = \mathcal{E}(t)
\end{gather*}
Выразим каждое из напряжений через заряд
\begin{gather*}
    U_L = L\dfrac{dI}{dt} = L\ddot{q} \hspace{2em} U_R = RI = R \dot{q} \hspace{2em} U_C = \dfrac{q}{C}
\end{gather*}
При подстановке получим {\it уравнение колебательного контура}
\begin{gather*}
    L\ddot{q} + R\dot{q} + \dfrac{q}{C} = \mathcal{E}(t)
\end{gather*}
Поделив на $L$ и сделав классические замены получим {\it уравнение гармонического осциллятора}
\begin{gather*}
    \ddot{q} + 2\delta\dot{q} + \omega_0^2 q = f(t)
\end{gather*}
Где $\delta = \dfrac{R}{2L}$ называется {\it декрементом затухания},
$\omega_0 = \dfrac{1}{\sqrt{LC}}$ {\it собственной частотой} колебательного контура,
$f(t)$ {\it вынуждающей силой}.

Если $\delta \not= 0$, то колебания называются {\it затухающими}
(при $\delta = 0$, соответственно, {\it незатухающими}),
а при $f(t) \not\equiv 0$ колебания называют {\it вынужденными}
(в противном случае, соответственно, называют {\it свободными}).

Уравнение гармонического осциллятора (в общем случае) --- это
{\it неоднородное дифференциальное уравнение второго порядка},
решением которого является сумма общего
решения однородного дифференциального уравнения (при $f(t) \equiv 0$,
т.е. когда в осцилляторе происходят {\it свободные колебания})
и любого частного решения исходного неоднородного дифференциального уравнения.
\begin{gather*}
    x_\text{о.н.} = x_\text{о.о.} + x_\text{ч.н.}
\end{gather*}
(сокращения имеют смысл: о.н. --- общее неоднородное, о.о. --- общее однородное,
ч.н. --- частное неоднородное)

{\it Циклическая частота} колебаний $\omega$ --- это величина,
равная числу полных колебаний в системе,
совершаемых за $2\pi$ секунд.
Более глубокий физический смысл эта величина имеет при рассмотрении вращательного движения,
при котором она будет в точности равна угловой скорости вращения $\dot{\varphi}$.

    {\it Частота} колебаний $\nu$ --- это количество колебаний за секунду.
Из определения можно установить нехитрую связь с циклической частотой
\begin{gather*}
    \omega = 2\pi\nu
\end{gather*}

{\it Периодом} $T$ колебаний называется время, за которое система совершает одно полное колебание.
Таким образом величина, обратная периоду, есть количество колебаний за секунду,
т.е. частота.
\begin{gather*}
    T = \dfrac{1}{\nu} = \dfrac{2\pi}{\omega}
\end{gather*}

\subsection*{Свободные затухающие колебания}

Найдём решение в случае свободных затухающих колебаний ($\delta \not= 0$, $f(t) \equiv 0$).
Будем искать его в виде $e^{\lambda t}$. Тогда
\begin{gather*}
    \lambda^2e^{\lambda t} + 2\delta\lambda e^{\lambda t} + \omega_0^2 e^{\lambda t} = 0 \\
    \lambda^2 + 2\delta \cdot \lambda + \omega_0^2 = 0 \\
    \lambda = -\delta \pm \sqrt{\delta^2 - \omega_0^2} \\
    q_\text{о.о.}(t) = c_1 \cdot e^{(-\delta + \sqrt{\delta^2 - \omega_0^2})t} + c_2 \cdot e^{(-\delta - \sqrt{\delta^2 - \omega_0^2})t}
\end{gather*}
Далее в зависимости от $\delta^2$ и $\omega_0^2$ возможны три случая:
\begin{enumerate}
    \item Случай $\delta^2 - \omega_0^2 \geqslant 0$. В этом случае в показателях экспонент
          будет находиться действительное число, что означает, что процесс не будет периодическим.
    \item Случай $\delta^2 - \omega_0^2 < 0$. В этом случае под корнем
          будет стоять отрицательное число, т.е. показатель экспоненты станет комплексным.
\end{enumerate}

Первый случай не представляет особого интереса,
поэтому рассмотрим подробнее второй случай.
Заменим $\sqrt{\omega_0^2 - \delta^2}$ на $\omega$. В этом случае решение примет вид
\begin{gather*}
    q(t) = c_1 \cdot e^{(-\delta + i\omega)t} + c_2 \cdot e^{(-\delta - i\omega)t}
\end{gather*}
Используя формулу Эйлера
($e^{i\varphi} = cos(\varphi) + i \cdot sin(\varphi)$)
и сделав более приятный коэффициент перед синусом (так можно делать т.к.
требуется лишь линейная независимость решений) получим более привычный вид
\begin{gather*}
    q(t) = A e^{-\delta t}cos(\omega t) + B e^{-\delta t} sin(\omega t)
\end{gather*}
Приведём к другой форме записи. Вынесем за скобку $e^{-\delta t} \sqrt{A^2 + B^2}$.
\begin{gather*}
    q(t) = \sqrt{A^2 + B^2} \cdot e^{-\delta t} \left( \dfrac{A}{\sqrt{A^2 + B^2}} cos(\omega t) + \dfrac{B}{\sqrt{A^2 + B^2}} sin(\omega t) \right)
\end{gather*}
Заметим, что (т.к. $A, B \in \mathbb{R}$) можно представить
\begin{gather*}
    \dfrac{A}{\sqrt{A^2 + B^2}} = cos(\varphi) \hspace{2em} -\dfrac{B}{\sqrt{A^2 + B^2}} = sin(\varphi) \hspace{2em} \varphi = - arc\ tg\left(\dfrac{B}{A}\right)
\end{gather*}
(будет выполнятся {\it основное тригонометрическое тождество}), тогда мы получим формулу косинуса суммы
\begin{gather*}
    q(t) = \sqrt{A^2 + B^2} \cdot e^{-\delta t} (cos(\omega t) cos(\varphi) - sin(\omega t) sin(\varphi)) = \sqrt{A^2 + B^2} \cdot e^{-\delta t} cos(\omega t + \varphi)
\end{gather*}
И т.к. $A, B \in \mathbb{R}$, то можно обозначить $\sqrt{A^2 + B^2} = C \in \mathbb{R}$ --- амплитуда колебаний,
тогда получим
\begin{gather*}
    q(t) = C e^{-\delta t} cos(\omega t + \varphi)
\end{gather*}
Т.е. мы получили уравнение гармонических колебаний
с экспоненциально убывающей со временем амплитудой.

{\it Логарифмическим декрементом} колебаний называется отношение амплитуд
(предшествующей к данной).
\begin{gather*}
    d = ln\left(\dfrac{q_n}{q_{n + 1}}\right)
\end{gather*}
Таким образом логарифмический декремент есть величина,
обратная количеству $N$ колебаний, за которое амплитуда уменьшится в $e$ раз.

В случае свободных колебаний без затухания, амплитуда колебаний сохраняется,
поэтому для колебаний в идеальном колебательном контуре
(без потерь, т.е. $R = 0$ или,по другому, $\delta = 0$) $d = 0$.
Таким образом, чем ближе $d$ к нулю, тем лучше колебательный контур.

В случае затухающих колебаний отношение амплитуд колебаний будет выглядеть
следующим образом
\begin{gather*}
    d = ln\left(\dfrac{C e^{-\delta t}}{C e^{-\delta (t + T)}}\right) = ln\left(\dfrac{1}{e^{-\delta T}}\right) = ln\left(e^{\delta T}\right) = \delta T
\end{gather*}
Заметим, что формула остаётся верной и для незатухающих колебаний (просто $\delta = 0$, откуда $d = ln(1) = 0$, что согласуется с формулой).

    {\it Добротностью} колебаний называется величина, в $\pi$ раз большая количества колебаний,
за которое амплитуда последних уменьшается в $e$ раз.
\begin{gather*}
    Q = \pi N = \dfrac{\pi}{d}
\end{gather*}
Добротность имеет физическим смыслом отношение энергии,
запасенной в колебательной системе, к энергии,
теряемой системой за один период колебания.

Запишем выражения для введённых величин в приложении к используемому контору
\begin{gather*}
    \omega = \sqrt{\omega_0^2 - \delta^2} = \sqrt{\dfrac{1}{LC} - \dfrac{R^2}{4L^2}} = \dfrac{1}{2LC}\sqrt{4LC - R^2C^2} \\
    T = \dfrac{2\pi}{\omega} = \dfrac{4LC}{\sqrt{4LC - R^2C^2}} \\
    d = \delta T = \dfrac{2RC}{\sqrt{4LC - R^2C^2}} \hspace{2em} Q = \dfrac{\pi}{d} = \dfrac{\pi}{4RC} \sqrt{4LC - R^2C^2}
\end{gather*}
При $\delta \ll \omega_0$
эти формулы перейдут в
\begin{gather*}
    \omega \approx \omega_0 = \dfrac{1}{\sqrt{LC}} \hspace{2em} T \approx 2\pi\sqrt{LC} \\
    d \approx \pi R \sqrt{\dfrac{C}{L}} \hspace{2em} Q \approx \dfrac{1}{R} \sqrt{\dfrac{L}{C}}
\end{gather*}

\subsection*{Вынужденные колебания}

Рассмотрим случай, когда колебания в контуре происходят под действием гармонической силы.
В более общем случае если сила изменяется не по гармоническому закону,
то разложением её в тригонометрический ряд Фурье сводим задачу к случаю гармонической силы.
Тогда сумма частных решений дифференциального уравнения для каждой гармоники даст частное решение для исходной негармонической силы.

Пусть $f(t) = f_0 \cdot cos(\omega t)$, где $f_0 = \dfrac{\mathcal{E}_0}{L}$, тогда решение дифференциального уравнения
\begin{gather*}
    \ddot{q} + 2\delta \dot{q} + \omega_0^2 q = f_0 \cdot cos(\omega t)
\end{gather*}
соответствующее установившемуся режиму колебаний, будет иметь вид
\begin{gather*}
    q_\text{ч.н.}(t) = \dfrac{f_0}{\sqrt{(\omega^2 - \omega_0^2)^2 + 4\delta^2\omega^2}} \cdot cos\left(\omega t + arc\ tg\left(\dfrac{2\delta\omega}{\omega^2 - \omega_0^2}\right)\right)
\end{gather*}
Т.е. при установлении колебаний система будет колебаться с частотой вынуждающей силы,
но со сдвигом по фазе --- с некоторым откликом.

До установления колебаний в системе будет действовать суперпозиция колебаний,
но собственные колебания системы, как уже было сказано, экспоненциально затухают,
поэтому со временем они станут практически не заметны.

Продифференцировав зависимость заряда по времени получим ток
\begin{gather*}
    I(t) = \dfrac{\omega f_0}{\sqrt{(\omega^2 - \omega_0^2)^2 + 4\delta^2\omega^2}} = \dfrac{\mathcal{E}_0}{\sqrt{\left(\omega L - \dfrac{1}{\omega C}\right)^2 + R^2}}
\end{gather*}
Заметим, что в знаменателе стоит модуль импеданса колебательного контура,
т.е. это выражение можно было получить и из {\it законов Кирфгофа} для переменных токов.

Зная ток, можно получить амплитуды напряжений на элементах контура: на индуктивности,
на конденсаторе и на резисторе они будут равны, соответственно
\begin{gather*}
    U_L = \dfrac{\omega L \mathcal{E}_0}{\sqrt{R^2 + \left(\omega L - \dfrac{1}{\omega C}\right)^2}} \cdot e^{i\dfrac{\pi}{2}} \\
    U_C = \dfrac{\mathcal{E}_0}{\omega C \sqrt{R^2 + \left(\omega L - \dfrac{1}{\omega C}\right)^2}} \cdot e^{-i\dfrac{\pi}{2}} \\
    U_R = \dfrac{R \mathcal{E}_0}{\sqrt{R^2 + \left(\omega L - \dfrac{1}{\omega C}\right)^2}}
\end{gather*}

\section*{\centering Практическая часть}

\begin{enumerate}
    \item Исследование собственных колебаний

          Построены графики практических зависимостей $d$ (рис. \ref{1 Логарифмический декремент от сопротивления}),
          $Q$ (рис. \ref{1 Добротность от сопротивления}), $\delta$ (рис. \ref{1 Декремент затухания от сопротивления})
          и $T$ (рис. \ref{1 Период от сопротивления}) от $R$
          и посчитаны ($T = 480$ мкс, $C = 32.3$ нФ) индуктивность $L = 0.181 \pm 0.047$ Гн
          и критическое сопротивление контура $R_\text{кр.} = 2 \sqrt{\dfrac{L}{C}} = 4730$ Ом.

          Теоретические зависимости $d$, $Q$, $\delta$ и $T$ от $R$ имеют вид
          \begin{gather*}
              d(R) = k_d \cdot R \hspace{2em} Q(R) = \dfrac{k_Q}{R} \hspace{2em} \delta(R) = k_\delta \cdot R \hspace{2em} T(R) = k_T
          \end{gather*}
          где $k_d = \pi\sqrt{\dfrac{C}{L}}$, $k_Q = \sqrt{\dfrac{L}{C}}$, $k_\delta = \dfrac{1}{2L}$, $k_T = 2\pi\sqrt{LC}$.

    \item Исследование вынужденных колебаний

          На рис. \ref{Эксперимент 50 Ом} и рис. \ref{Эксперимент 500 Ом} показаны
          практические зависимости напряжений на различных элементах контура
          при сопротивлениях, соответственно, $R = 50$ Ом и $R = 500$ Ом.

          На самом деле, ввиду неидеальности контура у последнего имеется внутреннее сопротивление $r \approx 80$ Ом,
          поэтому в формулах для $U_C$, $U_L$ и $U_R$ в модуле импеданса системы должно браться полное сопротивление системы $R + r$,
          но эта поправка не вносит изменений в импеданс резистора и он берётся просто $R$.

            \begin{gather*}
                I(t) = \dfrac{\omega f_0}{\sqrt{(\omega^2 - \omega_0^2)^2 + 4\delta^2\omega^2}} = \dfrac{\mathcal{E}_0}{\sqrt{\left(\omega L - \dfrac{1}{\omega C}\right)^2 + (R + r)^2}} \\
                U_L = \dfrac{\omega L \mathcal{E}_0}{\sqrt{(R + r)^2 + \left(\omega L - \dfrac{1}{\omega C}\right)^2}} \cdot e^{i\dfrac{\pi}{2}} \\
                U_C = \dfrac{\mathcal{E}_0}{\omega C \sqrt{(R + r)^2 + \left(\omega L - \dfrac{1}{\omega C}\right)^2}} \cdot e^{-i\dfrac{\pi}{2}} \\
                U_R = \dfrac{R \mathcal{E}_0}{\sqrt{(R + r)^2 + \left(\omega L - \dfrac{1}{\omega C}\right)^2}}
            \end{gather*}

    \item Исследование процессов установления вынужденных колебаний

          На рис. \ref{Установление колебаний} показана практическая зависимость
          времени установления колебаний в контуре от сопротивления резистора.
\end{enumerate}

\begin{figure}[p]
    \centering
    \begin{tikzpicture}
        \begin{axis}[xlabel={$R$, Ом}, ylabel={$d(R)$}, x post scale=1.75]
            \addplot coordinates{(0, 0.1008046991219655)(20, 0.13353139262452257)(50, 0.1808837420243271)(100, 0.24807293435673516)(230, 0.42744401482693956)(500, 0.78845736036427)(1000, 1.5040773967762742)};
        \end{axis}
    \end{tikzpicture}
    \caption{$d(R)$}
    \label{1 Логарифмический декремент от сопротивления}
\end{figure}

\begin{figure}[p]
    \centering
    \begin{tikzpicture}
        \begin{axis}[xlabel={$R$, Ом}, ylabel={$Q(R)$}, x post scale=1.75]
            \addplot coordinates{(0, 31.165140920550947)(20, 23.526996849524735)(50, 17.368021130209033)(100, 12.663987958767413)(230, 7.349717260310075)(500, 3.984480089244606)(1000, 2.0887174159542887)};
        \end{axis}
    \end{tikzpicture}
    \caption{$Q(R)$}
    \label{1 Добротность от сопротивления}
\end{figure}

\begin{figure}[p]
    \centering
    \begin{tikzpicture}
        \begin{axis}[xlabel={$R$, Ом}, ylabel={$\delta(R)$, $\text{с}^{-1}$}, x post scale=1.75]
            \addplot coordinates{(0, 210.00978983742812)(20, 278.1904013010887)(50, 376.8411292173481)(100, 516.8186132431982)(230, 890.5083642227908)(500, 1642.6195007588958)(1000, 3133.4945766172377)};
        \end{axis}
    \end{tikzpicture}
    \caption{$\delta(R)$}
    \label{1 Декремент затухания от сопротивления}
\end{figure}

\begin{figure}[p]
    \centering
    \begin{tikzpicture}
        \begin{axis}[xlabel={$R$, Ом}, ylabel={$T(R)$, с}, x post scale=1.75]
            \addplot coordinates{(0, 0.00048)(20, 0.00048)(50, 0.00048)(100, 0.00048)(230, 0.00048)(500, 0.00048)(1000, 0.00048)};
        \end{axis}
    \end{tikzpicture}
    \caption{$T(R)$}
    \label{1 Период от сопротивления}
\end{figure}

\begin{figure}[p]
    \centering
    \begin{tikzpicture}
        \begin{axis}[x post scale=1.75, ylabel={$U$, В}, xlabel={$\nu$, Гц}]
            \addplot coordinates{(1000, 0.3092)(1100, 0.3292)(1200, 0.3543)(1300, 0.3864)(1400, 0.4287)(1500, 0.4858)(1600, 0.5661)(1700, 0.6873)(1800, 0.8865)(1900, 1.2836)(1950, 1.6654)(2000, 2.3594)(2025, 2.9406)(2050, 3.7567)(2060, 4.1315)(2070, 4.4879)(2080, 4.7632)(2085, 4.8488)(2088, 4.8784)(2090, 4.8897)(2091, 4.8905)(2092, 4.8915)(2093, 4.8902)(2094, 4.8868)(2095, 4.8824)(2096, 4.8753)(2097, 4.864)(2099, 4.8429)(2102, 4.7946)(2105, 4.7363)(2110, 4.6101)(2120, 4.2913)(2130, 3.9364)(2150, 3.2518)(2200, 2.0934)(2300, 1.1378)(2400, 0.764)(2500, 0.5665)(2600, 0.4466)(2700, 0.3659)(2800, 0.3079)(2900, 0.2647)(3000, 0.2309)};
            \addlegendentry{$U_C$}
            \addplot coordinates{(1000, 0.0701)(1100, 0.0903)(1200, 0.1155)(1300, 0.1479)(1400, 0.1903)(1500, 0.2477)(1600, 0.3284)(1700, 0.4505)(1800, 0.6534)(1900, 1.0511)(1950, 1.4384)(2000, 2.1469)(2025, 2.745)(2050, 3.5999)(2060, 3.9985)(2070, 4.3885)(2080, 4.7034)(2085, 4.8139)(2088, 4.8559)(2090, 4.8752)(2091, 4.8827)(2092, 4.8875)(2093, 4.8913)(2094, 4.8924)(2095, 4.8923)(2096, 4.8903)(2097, 4.8846)(2099, 4.8705)(2102, 4.8379)(2105, 4.7906)(2110, 4.6852)(2120, 4.4017)(2130, 4.0727)(2150, 3.4222)(2200, 2.305)(2300, 1.3676)(2400, 0.9939)(2500, 0.7986)(2600, 0.6825)(2700, 0.6022)(2800, 0.5447)(2900, 0.5015)(3000, 0.468)};
            \addlegendentry{$U_L$};
            \addplot coordinates{(1000, 0.0023)(1100, 0.0031)(1200, 0.0039)(1300, 0.0049)(1400, 0.006)(1500, 0.0075)(1600, 0.0095)(1700, 0.125)(1800, 0.0171)(1900, 0.0264)(1950, 0.035)(2000, 0.0508)(2025, 0.064)(2050, 0.0828)(2060, 0.0918)(2070, 0.1004)(2080, 0.1076)(2085, 0.1102)(2088, 0.1112)(2090, 0.1117)(2091, 0.1118)(2092, 0.1119)(2093, 0.112)(2094, 0.1121)(2095, 0.1122)(2096, 0.1121)(2097, 0.112)(2099, 0.1118)(2102, 0.1109)(2105, 0.11)(2110, 0.1074)(2120, 0.1009)(2130, 0.0929)(2150, 0.076)(2200, 0.0509)(2300, 0.0287)(2400, 0.0198)(2500, 0.0152)(2600, 0.0123)(2700, 0.0104)(2800, 0.009)(2900, 0.0078)(3000, 0.007)};
            \addlegendentry{$U_R$}
        \end{axis}
    \end{tikzpicture}
    \caption{Экспериментальные значения при $R = 50$ Ом}
    \label{Эксперимент 50 Ом}
\end{figure}

\begin{figure}[p]
    \centering
    \begin{tikzpicture}
        \begin{axis}[x post scale=1.75, ylabel={$U$, В}, xlabel={$\nu$, Гц}]
            \addplot coordinates{(1000, 0.3054)(1100, 0.3238)(1200, 0.3465)(1300, 0.3748)(1400, 0.4106)(1500, 0.4566)(1600, 0.5166)(1700, 0.5955)(1800, 0.6983)(1900, 0.8197)(1950, 0.8814)(2000, 0.9319)(2030, 0.9517)(2040, 0.956)(2045, 0.9576)(2050, 0.959)(2055, 0.96)(2060, 0.9606)(2063, 0.9609)(2065, 0.961)(2066, 0.961)(2067, 0.961)(2068, 0.9609)(2069, 0.9609)(2070, 0.9609)(2072, 0.9607)(2075, 0.9606)(2080, 0.9597)(2090, 0.9573)(2100, 0.9534)(2110, 0.9482)(2120, 0.9416)(2125, 0.9377)(2128, 0.9355)(2131, 0.9329)(2133, 0.9312)(2134, 0.9303)(2135, 0.9294)(2137, 0.9276)(2140, 0.925)(2145, 0.92)(2150, 0.9148)(2160, 0.9038)(2170, 0.892)(2190, 0.866)(2200, 0.8522)(2300, 0.706)(2400, 0.5721)(2500, 0.4682)(2600, 0.3898)(2700, 0.3305)(2800, 0.2847)(2900, 0.2486)(3000, 0.2192)};
            \addlegendentry{$U_C$}
            \addplot coordinates{(1000, 0.0691)(1100, 0.0888)(1200, 0.1129)(1300, 0.1434)(1400, 0.1823)(1500, 0.2327)(1600, 0.2994)(1700, 0.3897)(1800, 0.5128)(1900, 0.6734)(1950, 0.7631)(2000, 0.846)(2030, 0.8905)(2040, 0.9033)(2045, 0.9094)(2050, 0.9152)(2055, 0.9207)(2060, 0.9258)(2063, 0.9288)(2065, 0.9308)(2066, 0.9316)(2067, 0.9326)(2068, 0.9334)(2069, 0.9343)(2070, 0.9353)(2072, 0.937)(2075, 0.9393)(2080, 0.9432)(2090, 0.9499)(2100, 0.9552)(2110, 0.9592)(2120, 0.9616)(2125, 0.9623)(2128, 0.9626)(2131, 0.9627)(2133, 0.9627)(2134, 0.9627)(2135, 0.9627)(2137, 0.9627)(2140, 0.9625)(2145, 0.9719)(2150, 0.9609)(2160, 0.9584)(2170, 0.9544)(2190, 0.9439)(2200, 0.9373)(2300, 0.8444)(2400, 0.747)(2500, 0.6722)(2600, 0.5959)(2700, 0.5445)(2800, 0.5038)(2900, 0.4709)(3000, 0.4446)};
            \addlegendentry{$U_L$}
            \addplot coordinates{(1000, 0.0325)(1100, 0.0378)(1200, 0.0441)(1300, 0.0518)(1400, 0.0611)(1500, 0.0728)(1600, 0.0878)(1700, 0.1075)(1800, 0.1334)(1900, 0.1659)(1950, 0.183)(2000, 0.1986)(2030, 0.206)(2040, 0.208)(2045, 0.2088)(2050, 0.2097)(2055, 0.2104)(2060, 0.2112)(2063, 0.2115)(2065, 0.2117)(2066, 0.2118)(2067, 0.2119)(2068, 0.212)(2069, 0.2121)(2070, 0.2122)(2072, 0.2124)(2075, 0.2127)(2080, 0.2131)(2090, 0.2136)(2100, 0.2138)(2110, 0.2137)(2120, 0.2131)(2125, 0.2128)(2128, 0.2127)(2131, 0.2124)(2133, 0.2122)(2134, 0.2121)(2135, 0.2119)(2137, 0.2118)(2140, 0.2115)(2145, 0.2108)(2150, 0.2102)(2160, 0.2086)(2170, 0.207)(2190, 0.2028)(2200, 0.2004)(2300, 0.1728)(2400, 0.1458)(2500, 0.124)(2600, 0.1072)(2700, 0.0941)(2800, 0.0839)(2900, 0.0756)(3000, 0.0689)};
            \addlegendentry{$U_R$}
        \end{axis}
    \end{tikzpicture}
    \caption{Экспериментальные значения при $R = 500$ Ом}
    \label{Эксперимент 500 Ом}
\end{figure}

\begin{figure}[p]
    \centering
    \begin{tikzpicture}
        \begin{axis}[x post scale=1.75, ylabel={$U$, В}, xlabel={$\nu$, Гц}]
            \addplot coordinates{(1000, 0.30832210159782275)(1100, 0.32898347399541755)(1200, 0.35502991401260403)(1300, 0.3884388065239095)(1400, 0.4323420345594973)(1500, 0.4919967202221874)(1600, 0.5769272527895799)(1700, 0.7062765899084328)(1800, 0.9246718681230192)(1900, 1.3624475939704634)(1950, 1.787603788258848)(2000, 2.5469249229160926)(2025, 3.136656202210166)(2050, 3.8352428573851585)(2060, 4.083627213210104)(2070, 4.259066249437758)(2080, 4.322708954284527)(2085, 4.3058761393385225)(2088, 4.2802834638080185)(2090, 4.257073474601883)(2091, 4.243704900989725)(2092, 4.229202828654919)(2093, 4.2136018681999525)(2094, 4.196938716890355)(2095, 4.179251925514688)(2096, 4.160581660973751)(2097, 4.140969467225173)(2099, 4.099090927230295)(2102, 4.030300324078034)(2105, 3.95541806598567)(2110, 3.8203060718765878)(2120, 3.5291740474232673)(2130, 3.2355116940983035)(2150, 2.7058269385447757)(2200, 1.8158548903694403)(2300, 1.0356680779058003)(2400, 0.7074003495615352)(2500, 0.5303916305783455)(2600, 0.4204390311122055)(2700, 0.34578801738340015)(2800, 0.29193337513110273)(2900, 0.2513356717303729)(3000, 0.21969620497678066)};
            \addlegendentry{$U_C$}
            \addplot coordinates{(1000, 0.07116156514453291)(1100, 0.09187562079332177)(1200, 0.11799626845304212)(1300, 0.15151308797927443)(1400, 0.1955799664288353)(1500, 0.2554968556259305)(1600, 0.34088026106439195)(1700, 0.4711004164276143)(1800, 0.6914702332683692)(1900, 1.1351881321187893)(1950, 1.5688496421075808)(2000, 2.351348318930995)(2025, 2.968641872427701)(2050, 3.719987031156963)(2060, 3.9996442444087)(2070, 4.212073300449801)(2080, 4.316418045813936)(2085, 4.320305773744737)(2088, 4.306994864249932)(2090, 4.291850168658254)(2091, 4.282467500805111)(2092, 4.271916038132106)(2093, 4.260227471141778)(2094, 4.247435732753485)(2095, 4.233576772379924)(2096, 4.21868832433455)(2097, 4.202809673204135)(2099, 4.168245242475781)(2102, 4.110017459464702)(2105, 4.045176047997578)(2110, 3.9255806408253817)(2120, 3.6608811740913056)(2130, 3.3879969806829595)(2150, 2.8868071344513564)(2200, 2.0284641449874825)(2300, 1.2644949413417241)(2400, 0.940434574734741)(2500, 0.7650987874352263)(2600, 0.6559799365144372)(2700, 0.5818061421457852)(2800, 0.5282513856490655)(2900, 0.487855225898322)(3000, 0.45635824189371094)};
            \addlegendentry{$U_L$}
            \addplot coordinates{(1000, 0.0031286505113006465)(1100, 0.0036721394271455755)(1200, 0.004323132916827791)(1300, 0.005124108987466799)(1400, 0.006141972854557239)(1500, 0.007488690150516237)(1600, 0.0093668471254927)(1700, 0.01218361390430698)(1800, 0.016889334810553543)(1900, 0.02626792709595244)(1950, 0.03537189056656359)(2000, 0.05168904805092353)(2025, 0.06445317894638135)(2050, 0.07978093575614811)(2060, 0.08536222068694595)(2070, 0.08946169781834412)(2080, 0.09123715316117019)(2085, 0.09110033779296084)(2088, 0.09068916864016208)(2090, 0.09028379938033596)(2091, 0.09004334168054208)(2092, 0.08977855048200861)(2093, 0.08949012635225796)(2094, 0.08917881560217612)(2095, 0.08884540543563295)(2096, 0.08849071899407297)(2097, 0.0881156103516562)(2099, 0.08730766747113156)(2102, 0.08596516798889259)(2105, 0.08448836153275582)(2110, 0.08179617756712441)(2120, 0.07592089738985364)(2130, 0.06993184274325082)(2150, 0.059032458589346415)(2200, 0.04053743038164652)(2300, 0.024171345235314876)(2400, 0.01722776372278528)(2500, 0.013455166184964815)(2600, 0.011092482943623076)(2700, 0.009473841154463638)(2800, 0.008294575692698489)(2900, 0.00739612980709956)(3000, 0.006687999080857364)};
            \addlegendentry{$U_R$}
        \end{axis}
    \end{tikzpicture}
    \caption{Теоретические значения при $R = 50$ Ом}
    \label{Теория R = 50 Ом}
\end{figure}

\begin{figure}[p]
    \centering
    \begin{tikzpicture}
        \begin{axis}[x post scale=1.75, ylabel={$U$, В}, xlabel={$\nu$, Гц}]
            \addplot coordinates{(1000, 0.3049534130989124)(1100, 0.3240620963885382)(1200, 0.34773175708863385)(1300, 0.3773581394078164)(1400, 0.4149440036282049)(1500, 0.4633904128016677)(1600, 0.526852591152936)(1700, 0.610839390885614)(1800, 0.7204278889838499)(1900, 0.8505364645061314)(1950, 0.912284327893395)(2000, 0.9583017436869279)(2030, 0.9729438135390265)(2040, 0.9751286274071402)(2045, 0.9756816776686272)(2050, 0.9758687089221743)(2055, 0.9756863907977882)(2060, 0.9751326431535833)(2063, 0.9746217199711087)(2065, 0.9742066609058873)(2066, 0.9739768160747321)(2067, 0.9737321071015898)(2068, 0.9734725461077239)(2069, 0.9731981472737264)(2070, 0.9729089268357141)(2072, 0.9722860963413852)(2075, 0.9712412120915525)(2080, 0.9692065643376706)(2090, 0.9640548871177202)(2100, 0.9575026048548464)(2110, 0.9496161527283568)(2120, 0.9404774482307261)(2125, 0.9354676797803215)(2128, 0.9323281565088224)(2131, 0.9290918749660049)(2133, 0.9268820416272872)(2134, 0.9257617603120456)(2135, 0.9246313835666338)(2137, 0.9223407901189207)(2140, 0.918831986957436)(2145, 0.912797123108155)(2150, 0.9065411334662635)(2160, 0.8934237849534826)(2170, 0.8795960752636046)(2190, 0.8502644700408701)(2200, 0.8349772255440666)(2300, 0.6790747602867203)(2400, 0.5468160142905024)(2500, 0.4465312327732244)(2600, 0.37172805789973623)(2700, 0.3151745210693086)(2800, 0.2715050361657287)(2900, 0.23705118257881008)(3000, 0.20932806518068872)};
            \addlegendentry{$U_C$}
            \addplot coordinates{(1000, 0.07038406283501783)(1100, 0.09050122159539523)(1200, 0.11557068331324277)(1300, 0.14719100155682444)(1400, 0.18770956282827644)(1500, 0.24064142814724565)(1600, 0.31129340475126865)(1700, 0.4074419222275388)(1800, 0.5387364508665106)(1900, 0.7086649825758846)(1950, 0.8006455069721871)(2000, 0.8847144153220621)(2030, 0.9253811969051955)(2040, 0.9366192409155621)(2045, 0.9417499548442042)(2050, 0.9465421294800181)(2055, 0.950987336163595)(2060, 0.9550782821476919)(2063, 0.9573602148884103)(2065, 0.9588088644078945)(2066, 0.9595112865278833)(2067, 0.9601990622871115)(2068, 0.9608721607740235)(2069, 0.9615305531168304)(2070, 0.9621742124896401)(2072, 0.9634172352824333)(2075, 0.9651707197167201)(2080, 0.967796062300678)(2090, 0.971930401144637)(2100, 0.9745842475937639)(2110, 0.9757843259753194)(2120, 0.9755756272205087)(2125, 0.9749615532090515)(2128, 0.9744350150340363)(2131, 0.9737924330031381)(2133, 0.9733006481484647)(2134, 0.9730359853050959)(2135, 0.9727589217941732)(2137, 0.9721679375620098)(2140, 0.9711906358915431)(2145, 0.9693256154781723)(2150, 0.9671754591856808)(2160, 0.9620681703125313)(2170, 0.9559685052106092)(2190, 0.9412025501537831)(2200, 0.9327404810153489)(2300, 0.829113706885445)(2400, 0.7269500024649079)(2500, 0.6441287626922426)(2600, 0.5799798062911697)(2700, 0.530297358461279)(2800, 0.49128645020747225)(2900, 0.4601283113943046)(3000, 0.4348212924978267)};
            \addlegendentry{$U_L$}
            \addplot coordinates{(1000, 0.03094467269360125)(1100, 0.03617206622994007)(1200, 0.042342646238053894)(1300, 0.049779378402922805)(1400, 0.05894811521258661)(1500, 0.07053273075935047)(1600, 0.08553847396075709)(1700, 0.10537275909225932)(1800, 0.1315877366163124)(1900, 0.1639830400888106)(1950, 0.18051663139104704)(2000, 0.19448435417563853)(2030, 0.20041775539797566)(2040, 0.20185730417437234)(2045, 0.20246681769309907)(2050, 0.20300075293285774)(2055, 0.2034578583003725)(2060, 0.20383713678528273)(2063, 0.2040270304724515)(2065, 0.2041378543347334)(2066, 0.20418852481825794)(2067, 0.20423603096701987)(2068, 0.20428037079968397)(2069, 0.204321542772188)(2070, 0.20435954577798845)(2072, 0.2044260426511962)(2075, 0.20450201819939862)(2080, 0.20456535170530282)(2090, 0.20445627386853696)(2100, 0.2040382800105803)(2110, 0.203321330772388)(2120, 0.2023189870636404)(2125, 0.20171589277578486)(2128, 0.20132273367010273)(2131, 0.2009067400954766)(2133, 0.20061699376461675)(2134, 0.20046845711319347)(2135, 0.20031750597161896)(2137, 0.20000844471188628)(2140, 0.19952727627756003)(2145, 0.19867990981074443)(2150, 0.197778177009604)(2160, 0.19582298047960764)(2170, 0.19368474303592273)(2190, 0.18895158594776876)(2200, 0.18640163005462698)(2300, 0.15848852370413571)(2400, 0.13316952839888996)(2500, 0.1132776536686595)(2600, 0.09807336704708841)(2700, 0.08635097801072976)(2800, 0.07714154205267318)(2900, 0.06975775882542433)(3000, 0.06372371828972007)};
            \addlegendentry{$U_R$}
        \end{axis}
    \end{tikzpicture}
    \caption{Теоретические значения при $R = 500$ Ом}
    \label{Теория R = 500 Ом}
\end{figure}

\begin{figure}[p]
    \centering
    \begin{tikzpicture}
        \begin{axis}[xlabel={$R$, кОм}, ylabel={$\tau$, мс}, x post scale=1.75]
            \addplot coordinates{(0, 5.2)(0.02, 3.8)(0.05, 2.8)(0.1, 1.6)(0.23, 1.4)(0.5, 0.7)};
        \end{axis}
    \end{tikzpicture}
    \caption{Установление колебаний}
    \label{Установление колебаний}
\end{figure}

\begin{table}[p]
    \centering
    \begin{tabular}{|c|c|c|}
        \hline
        $R$, Ом & $q_n$ & $q_{n+1}$ \\
        \hline
        0       & 1.46  & 1.32      \\
        \hline
        20      & 1.44  & 1.26      \\
        \hline
        50      & 1.39  & 1.16      \\
        \hline
        100     & 1.32  & 1.03      \\
        \hline
        230     & 1.15  & 0.75      \\
        \hline
        500     & 0.88  & 0.4       \\
        \hline
        1000    & 0.54  & 0.12      \\
        \hline
    \end{tabular}
    \caption{Собственные колебания}
\end{table}

\begin{table}[p]
    \centering
    \begin{tabular}{|c|c|c|c|}
        \hline
        $\nu$, кГц & $U_C$, В & $U_L$, В & $U_R$, В \\
        \hline
        1000       & 0.31     & 0.07     & 0.0      \\
        \hline
        1100       & 0.33     & 0.09     & 0.0      \\
        \hline
        1200       & 0.35     & 0.12     & 0.0      \\
        \hline
        1300       & 0.39     & 0.15     & 0.0      \\
        \hline
        1400       & 0.43     & 0.19     & 0.01     \\
        \hline
        1500       & 0.49     & 0.25     & 0.01     \\
        \hline
        1600       & 0.57     & 0.33     & 0.01     \\
        \hline
        1700       & 0.69     & 0.45     & 0.12     \\
        \hline
        1800       & 0.89     & 0.65     & 0.02     \\
        \hline
        1900       & 1.28     & 1.05     & 0.03     \\
        \hline
        1950       & 1.67     & 1.44     & 0.04     \\
        \hline
        2000       & 2.36     & 2.15     & 0.05     \\
        \hline
        2025       & 2.94     & 2.75     & 0.06     \\
        \hline
        2050       & 3.76     & 3.6      & 0.08     \\
        \hline
        2060       & 4.13     & 4.0      & 0.09     \\
        \hline
        2070       & 4.49     & 4.39     & 0.1      \\
        \hline
        2080       & 4.76     & 4.7      & 0.11     \\
        \hline
        2085       & 4.85     & 4.81     & 0.11     \\
        \hline
        2088       & 4.88     & 4.86     & 0.11     \\
        \hline
        2090       & 4.89     & 4.88     & 0.11     \\
        \hline
        2091       & 4.89     & 4.88     & 0.11     \\
        \hline
        2092       & 4.89     & 4.89     & 0.11     \\
        \hline
        2093       & 4.89     & 4.89     & 0.11     \\
        \hline
        2094       & 4.89     & 4.89     & 0.11     \\
        \hline
        2095       & 4.88     & 4.89     & 0.11     \\
        \hline
        2096       & 4.88     & 4.89     & 0.11     \\
        \hline
        2097       & 4.86     & 4.88     & 0.11     \\
        \hline
        2099       & 4.84     & 4.87     & 0.11     \\
        \hline
        2102       & 4.79     & 4.84     & 0.11     \\
        \hline
        2105       & 4.74     & 4.79     & 0.11     \\
        \hline
        2110       & 4.61     & 4.69     & 0.11     \\
        \hline
        2120       & 4.29     & 4.4      & 0.1      \\
        \hline
        2130       & 3.94     & 4.07     & 0.09     \\
        \hline
        2150       & 3.25     & 3.42     & 0.08     \\
        \hline
        2200       & 2.09     & 2.31     & 0.05     \\
        \hline
        2300       & 1.14     & 1.37     & 0.03     \\
        \hline
        2400       & 0.76     & 0.99     & 0.02     \\
        \hline
        2500       & 0.57     & 0.8      & 0.02     \\
        \hline
        2600       & 0.45     & 0.68     & 0.01     \\
        \hline
        2700       & 0.37     & 0.6      & 0.01     \\
        \hline
        2800       & 0.31     & 0.54     & 0.01     \\
        \hline
        2900       & 0.26     & 0.5      & 0.01     \\
        \hline
        3000       & 0.23     & 0.47     & 0.01     \\
        \hline
    \end{tabular}
    \caption{Вынужденные колебания $R = 50$ Ом}
\end{table}

\begin{table}[p]
    \centering
    \begin{tabular}{|c|c|c|c|}
        \hline
        $\nu$, кГц & $U_C$, В & $U_L$, В & $U_R$, В \\
        \hline
        1000       & 0.31     & 0.07     & 0.03     \\
        \hline
        1100       & 0.32     & 0.09     & 0.04     \\
        \hline
        1200       & 0.35     & 0.11     & 0.04     \\
        \hline
        1300       & 0.37     & 0.14     & 0.05     \\
        \hline
        1400       & 0.41     & 0.18     & 0.06     \\
        \hline
        1500       & 0.46     & 0.23     & 0.07     \\
        \hline
        1600       & 0.52     & 0.3      & 0.09     \\
        \hline
        1700       & 0.6      & 0.39     & 0.11     \\
        \hline
        1800       & 0.7      & 0.51     & 0.13     \\
        \hline
        1900       & 0.82     & 0.67     & 0.17     \\
        \hline
        1950       & 0.88     & 0.76     & 0.18     \\
        \hline
        2000       & 0.93     & 0.85     & 0.2      \\
        \hline
        2030       & 0.95     & 0.89     & 0.21     \\
        \hline
        2040       & 0.96     & 0.9      & 0.21     \\
        \hline
        2045       & 0.96     & 0.91     & 0.21     \\
        \hline
        2050       & 0.96     & 0.92     & 0.21     \\
        \hline
        2055       & 0.96     & 0.92     & 0.21     \\
        \hline
        2060       & 0.96     & 0.93     & 0.21     \\
        \hline
        2063       & 0.96     & 0.93     & 0.21     \\
        \hline
        2065       & 0.96     & 0.93     & 0.21     \\
        \hline
        2066       & 0.96     & 0.93     & 0.21     \\
        \hline
        2067       & 0.96     & 0.93     & 0.21     \\
        \hline
        2068       & 0.96     & 0.93     & 0.21     \\
        \hline
        2069       & 0.96     & 0.93     & 0.21     \\
        \hline
        2070       & 0.96     & 0.94     & 0.21     \\
        \hline
        2072       & 0.96     & 0.94     & 0.21     \\
        \hline
        2075       & 0.96     & 0.94     & 0.21     \\
        \hline
        2080       & 0.96     & 0.94     & 0.21     \\
        \hline
        2090       & 0.96     & 0.95     & 0.21     \\
        \hline
        2100       & 0.95     & 0.96     & 0.21     \\
        \hline
        2110       & 0.95     & 0.96     & 0.21     \\
        \hline
        2120       & 0.94     & 0.96     & 0.21     \\
        \hline
        2125       & 0.94     & 0.96     & 0.21     \\
        \hline
        2128       & 0.94     & 0.96     & 0.21     \\
        \hline
        2131       & 0.93     & 0.96     & 0.21     \\
        \hline
        2133       & 0.93     & 0.96     & 0.21     \\
        \hline
        2134       & 0.93     & 0.96     & 0.21     \\
        \hline
        2135       & 0.93     & 0.96     & 0.21     \\
        \hline
        2137       & 0.93     & 0.96     & 0.21     \\
        \hline
        2140       & 0.93     & 0.96     & 0.21     \\
        \hline
        2145       & 0.92     & 0.97     & 0.21     \\
        \hline
        2150       & 0.91     & 0.96     & 0.21     \\
        \hline
        2160       & 0.9      & 0.96     & 0.21     \\
        \hline
        2170       & 0.89     & 0.95     & 0.21     \\
        \hline
        2190       & 0.87     & 0.94     & 0.2      \\
        \hline
        2200       & 0.85     & 0.94     & 0.2      \\
        \hline
        2300       & 0.71     & 0.84     & 0.17     \\
        \hline
        2400       & 0.57     & 0.75     & 0.15     \\
        \hline
        2500       & 0.47     & 0.67     & 0.12     \\
        \hline
        2600       & 0.39     & 0.6      & 0.11     \\
        \hline
        2700       & 0.33     & 0.54     & 0.09     \\
        \hline
        2800       & 0.28     & 0.5      & 0.08     \\
        \hline
        2900       & 0.25     & 0.47     & 0.08     \\
        \hline
        3000       & 0.22     & 0.44     & 0.07     \\
        \hline
    \end{tabular}
    \caption{Вынужденные колебания $R = 500$ Ом}
\end{table}

\begin{table}[p]
    \centering
    \begin{tabular}{|c|c|}
        \hline
        $R$, кОм & $\tau$, мс \\
        \hline
        0        & 5.2        \\
        \hline
        0.02     & 3.8        \\
        \hline
        0.05     & 2.8        \\
        \hline
        0.1      & 1.6        \\
        \hline
        0.23     & 1.4        \\
        \hline
        0.5      & 0.7        \\
        \hline
    \end{tabular}
    \caption{Установление колебаний}
\end{table}

\end{document}