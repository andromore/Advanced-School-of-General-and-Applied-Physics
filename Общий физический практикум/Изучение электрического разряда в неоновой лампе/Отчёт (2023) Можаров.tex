\documentclass[a4paper, usenames, dvipsnames]{article}

\title{Газовый разряд в неоновой лампе}
\author{Бояринцева Н.А. \and Можаров А.Р.}
\date{19 декабря 2023}

\usepackage{cmap}
\usepackage[T2A]{fontenc}
\usepackage[utf8]{inputenc}
\usepackage[russian]{babel}
\usepackage{indentfirst}
\usepackage[hidelinks]{hyperref}
\usepackage{circuitikz}
\usepackage{pgfplots}
\usepackage{amsmath}
\usepackage{subcaption}
\usepackage{comment}
\usepackage[left=3.5cm, right=3.5cm, top=2.5cm, bottom=3cm]{geometry}
\usepackage{xcolor}
\usepackage{verbatim}
\usepackage{circuitikz}

\renewcommand{\thesubfigure}{.\arabic{subfigure}}
\DeclareCaptionLabelFormat{My}{Рис. \thefigure #2: }
\captionsetup[subfigure]{labelformat=My}
\renewcommand{\theenumii}{\arabic{enumii}}

\begin{document}

\maketitle

\section*{\centering Теоретическая часть}

\subsection*{Газовый разряд}

{\it Плазма} --- четвёртое агрегатное состояние вещества,
представляющее собой ионизированный газ.
В классическом понимании, плазма представляет собой газ из
свободных электронов ({\it электронный газ}) и ионов,
однако, в широком смысле, плазма может состоять из любых заряженных частиц.
Ввиду того, что заряженные частицы в плазме обладают подвижностью,
она способна проводить электрический ток.

{\it Газовый разряд} --- явление протекания электрического тока через газ.
Протекание тока через газ становится возможным при ионизации газа.
Газовые разряды можно классифицировать:
\begin{enumerate}
    \item По изменения характеристик газа с течением времени
          \begin{enumerate}
              \item {\it Стационарный} --- газовый разряд, характеристики которого меняются слабо
                    или вовсе не меняются с течением времени.
              \item {\it Нестационарный}  --- характеристики которого, соответственно,
                    заметно меняются с течением времени.
          \end{enumerate}
    \item По состоянию ионизированного газа
          \begin{enumerate}
              \item {\it Пробой газа} --- нестационарный процесс интенсивной ионизации газа
                    под действием внешнего поля при достижении некоторой пороговой величины.
              \item {\it Неравновесная плазма} --- плазма,
                    не находящаяся в термодинамическом равновесии,
                    т.е. температура электронного газа намного выше температуры тяжёлых частиц.
              \item {\it Равновесная плазма} (также {\it максвелловская плазма}) --- плазма,
                    находящаяся в термодинамическом равновесии.
          \end{enumerate}
\end{enumerate}
Причём наиболее распространёнными являются:
\begin{enumerate}
    \item {\it Коронный} --- стационарный газовый разряд,
          наблюдающийся в резко неоднородных полях у тел
          с большой кривизной поверхности. Пример: {\it огни Святого Эльма}.
    \item {\it Искровой} --- нестационарный газовый разряд,
          наблюдающийся при атмосферном давлении и мощности источника,
          недостаточной для поддержания стационарного дугового разряда.
          Пример: {\it молния}.
    \item {\it Дуговой} --- стационарный газовый разряд,
          наблюдающийся при атмосферном давлении. Для стационарности разряда
          требуется мощности источника, достаточная для поддержания стационарного
          дугового разряда. В противном случае будет искровой разряд.
    \item {\it Тлеющий} --- стационарный газовый разряд,
          наблюдающийся при давлении, ниже атмосферного.
\end{enumerate}

Рассмотрим также процесс неупругого соударения электроном с молекулами газа.
Все неупругие соударения делятся на два типа:
\begin{eqnarray*}
    \text{\it Первого рода} & \text{\it Второго рода} \\
    e' + X = X^* + e & e' + X = X^+ + e + e \\
    X^* = X + \gamma &
\end{eqnarray*}
Тип неупругого соударения зависит от энергии ударяющегося электрона $e'$.
Если его энергия достаточна или даже превышает {\it энергию ионизации} молекулы,
то произойдёт соударение второго рода. В противном случае произойдёт
соударение первого рода.

В результате соударения первого рода,
поступательная энергия движения электрона переходит во внутреннюю энергию молекулы,
в результате чего молекула переходит в {\it возбуждённое состояние} $X^*$,
которое заключается в переходе электронов молекулы на более высокий
{\it энергетический уровень}.

Но в результате {\it спонтанного перехода},
который рано или поздно происходит ввиду того,
что {\it любая система стремится в состояние с наименьшей потенциальной энергией},
электрон с более высокоэнергетического уровня перейдёт на уровень
с меньшей энергией, испустив при этом {\it квант света} $\gamma$ --- {\it фотон}.

При неупругом соударении второго рода,
ударяющийся электрон просто вышибет один из электронов молекулы,
что приведёт к её ионизации.
Неупругое соударение второго рода называется {\it ударной ионизацией}.

Заметим, что при неупругом соударении обоих родов выполняется
{\it закон сохранения энергии}.

\subsection*{Тлеющий разряд}

{\it Тлеющий разряд} --- стационарный газовый разряд, зажигающийся в разреженном газе
и имеющий {\it холодный катод} (т.е. не требуется нагревание катода
для запуска процесса термоэлектронной эмиссии).

Рассмотрим внутреннюю структуру тлеющего разряда,
следуя от катода к аноду, одновременно составляя качественную картину происходящего.
Сначала рассмотрим картину уже зажжённого разряда.

Зона, непосредственно прилегающая к катоду,
называется {\it астоновым пространством}.
В этой зоне электроны, вылетевшие с катода,
под действием внешнего поля не успели набрать достаточную
энергию для возбуждения или ионизации молекул,
поэтому в этой зоне нет свечения.

Далее следует {\it катодное свечение}. Здесь энергии электроном уже
достаточно для возбуждения молекул газа, но не достаточно для ионизации.
Катодное свечение может иметь несколько слоёв разных цветов.

Следом за ним следует {\it тёмное катодное пространство}.
Здесь энергии электронов уже достаточно ионизации,
соответственно, здесь и зарождается {\it электронная лавина}.
При чём граница между катодным свечением и тёмным
катодным пространством нечёткая.

Астоново пространство, катодное свечение и тёмное катодное пространство
вместе образуют зону, именуемую {\it катодным слоем}.
В тёмном катодном пространстве в результате лавинной ионизации образуются
положительные ионы, которые под действием внешнего поля дружно ускоряются
и выбивают из катода новые электроны.
Таким образом на протяжении катодного слоя
от его катодной границы и вплоть до катодной границы тёмного катодного пространства
образуется положительный пространственный заряд, причём к тёмному катодному пространству
заряд возрастает. В тёмном катодном пространстве при движении к его анодной границе
будет преобладать отрицательный заряд.

Далее следует зона {\it отрицательного свечения}.
Здесь лавина электронов, которые находятся в возбуждённом состоянии,
переходят в основное состояние, испуская при этом лишнюю энергию в
качестве квантов света. В следствии того, что сначала электроны будут испускать
фотоны более высоких частот (более высоко-энергетических уровней),
в начале слоя будут наблюдаться спектральные линии более высоких энергий,
в отличии от катодного свечения, где сначала наблюдаются спектральные линии
более низких частот.

Далее следует {\it фарадеево пространство},
в котором происходит плавное возрастание поля после спада в тёмном катодном пространстве
до значений {\it положительного столба}.

В {\it положительном столбе} находится газ, перешедший в состояние неравновесной плазмы,
с хаотическим распределением электронов по скоростям, хотя распределение электронов
по скоростям имеет смещение в виде скорости дрейфа электронов к аноду.
Кроме того, до положительного столба долетает некоторое количество
возбуждённых электронов из катодного слоя,
что обуславливает его свечение.

Далее, вблизи анода, располагается {\it анодный слой}.
Этому слою соответствует пространство, в котором поле анода
начинает ускорять электроны, вытягивая их из положительного столба.
В анодном слое выделяются две зоны --- {\it тёмное анодное пространство},
в котором электроны начинают ускоряться полем анода,
но их энергии ещё недостаточно для возбуждения молекул газа, и
{\it анодное свечение}, где энергии ускоренных полем анода электронов уже стало
достаточно для возбуждения молекул газа, в результате чего наблюдается свечение.

При зажигании заряда первичными электронами будут выступать те электроны,
молекулы которых ионизировались в результате естественной ионизации.
Для зажигания требуется некоторое напряжение, называемое {\it напряжением зажигания}.
При этом гаснуть разряд будет при напряжении называемом {\it напряжении гашения}.
Напряжение зажигания будет строго выше напряжение гашения.

В неоновой лампе катодное свечение будет жёлтого цвета,
отрицательное свечение оранжевое,
а положительному столбу будет соответствовать красный цвет,
причём наибольшую интенсивность будет иметь отрицательно, т.е. оранжевое, свечение.

\subsection*{Релаксационный генератор}

Рассчитаем период колебаний генератора.
Уравнение цепи для токов имеет вид:
\begin{gather*}
    C \dfrac{dU}{dt} + I = \dfrac{\mathcal{E} - U}{R}
\end{gather*}

При постоянном напряжении на конденсаторе,
сила тока в цепи определяется соотношением:
\begin{gather*}
    I_\text{ст.} = \dfrac{\mathcal{E} - U}{R}
\end{gather*}
Для установления стационарного режима,
сопротивление
\begin{gather*}
    R_\text{кр.} = \dfrac{\mathcal{E} - U_\text{г.}}{I_\text{г.}}
\end{gather*}
является максимально возможным.
При дальнейшем увеличении сопротивления,
в системе установятся колебания.

Рассмотрим колебательный режим работы схемы.
Заметим, что т.к. лампа и конденсатор соединены параллельно,
то напряжения на них совпадают. (На самом деле,
чтобы в момент зажигания лампы ток через лампу не превышал значений,
при которых электроды лампы разрушатся,
последовательно с лампой включён резистор сопротивлением 10 кОм).
Если напряжение $U$ на конденсаторе меньше напряжения на зажигания $U_\text{з.}$,
то он начинает заряжаться. При напряжении $U$ на конденсаторе больше или равном
напряжению зажигания $U_\text{з.}$, конденсатор начнём разряжаться вплоть
до установления напряжения напряжения гашения $U_\text{г.}$.

Вычислим период колебаний $T$ генератора.
За одно колебания генератор успевает зарядиться за время $\tau_\text{з.}$
и разрядиться за время $\tau_\text{р.}$, соответственно,
тогда $T = \tau_\text{з.} + \tau_\text{р.}$.

При зарядке конденсатора, т.к. лампа не проводит ток,
то уравнение уравнение цепи примет вид:
\begin{gather*}
    RC\dfrac{dU}{dt} = \mathcal{E} - U
\end{gather*}
Если отсчитывать время от момента,
когда на лампе напряжение равно напряжению гашения,
то получаем начальное условие $U(0) = U_\text{г.}$.
Тогда решим дифференциальное уравнение с разделяемыми переменными:
\begin{gather*}
    -RC\dfrac{d(U - \mathcal{E})}{dt} = U - \mathcal{E} \\
    \dfrac{d(U - \mathcal{E})}{U - \mathcal{E}} = -\dfrac{dt}{RC} \\
    ln(U(t) - \mathcal{E}) = -\dfrac{t}{RC} + C_1 \\
    U(t) - \mathcal{E} = C_2 \cdot e^{-\dfrac{t}{RC}} \text{, где } C_2 = e^{C_1}
\end{gather*}
Подставим начальные условия:
\begin{gather*}
    U_\text{г.} - \mathcal{E} = C_2 \cdot e^0 = C_2
\end{gather*}
Тогда перенеся $\mathcal{E}$ в правую часть и подставив значение
константы интегрирования $C_2$ получим:
\begin{gather*}
    U(t) = \mathcal{E} - (\mathcal{E} - U_\text{г.})\cdot e^{-\dfrac{t}{RC}}
\end{gather*}
По прошествии времени зажигания получим $U(\tau_\text{з.}) = U_\text{з.}$.
Подставив это в уравнение получим:
\begin{gather*}
    U_\text{з.} = \mathcal{E} - (\mathcal{E} - U_\text{г.}) \cdot
    e^{-\dfrac{\tau_\text{з.}}{RC}}
\end{gather*}
Преобразуем выражение:
\begin{gather*}
    U_\text{з.} - \mathcal{E} = - (\mathcal{E} - U_\text{г.}) \cdot e^{-\dfrac{\tau_\text{з.}}{RC}} \\
    \dfrac{\mathcal{E} - U_\text{з.}}{\mathcal{E} - U_\text{г.}} = e^{-\dfrac{\tau_\text{з.}}{RC}} \\
    -\dfrac{\tau_\text{з.}}{RC} = ln\left(\dfrac{\mathcal{E} - U_\text{з.}}{\mathcal{E} - U_\text{г.}}\right) \\
    \dfrac{\tau_\text{з.}}{RC} = ln\left(\left(\dfrac{\mathcal{E} - U_\text{з.}}{\mathcal{E} - U_\text{г.}}\right)^{\hspace{-0.25em}-1}\right)
\end{gather*}
Тогда время зарядки $\tau_\text{з.}$:
\begin{gather*}
    \tau_\text{з.} = RC \cdot ln\left(\dfrac{\mathcal{E} - U_\text{г.}}{\mathcal{E} - U_\text{з.}}\right)
\end{gather*}

Найдём теперь время разрядки $\tau_\text{р.}$.
Рассмотрим приближение, при котором вольт-амперная характеристика лампы
представляет собой линейную зависимость тока от напряжения:
\begin{gather*}
    I(U) = k \cdot U + b = \dfrac{U - U_0}{R_0}
\end{gather*}
Тогда уравнение цепи примет вид:
\begin{gather*}
    C\dfrac{dU}{dt} + \dfrac{U - U_0}{R_0} = \dfrac{\mathcal{E} - U}{R}
\end{gather*}
Пусть:
\begin{gather*}
    \dfrac{1}{\rho} = \dfrac{1}{R} + \dfrac{1}{R_0} \hspace{2em} \rho = \dfrac{RR_0}{R + R_0}
\end{gather*}
Тогда уравнение цепи примет вид:
\begin{gather*}
    C \rho \dfrac{dU}{dt} + U = \rho \left(\dfrac{\mathcal{E}}{R} + \dfrac{U_0}{R_0}\right)
\end{gather*}
Будем теперь отсчитывать время от момента максимальной зарядки конденсатора.
Тогда получим начальное условие $U(0) = U\text{з.}$.
Решив ещё одно уравнение с разделяющимися переменными, получим:
\begin{gather*}
    U(t) = \rho \left(\dfrac{\mathcal{E}}{R} + \dfrac{U_0}{R_0}\right)
    + \left(U_\text{з.} - \rho \left(\dfrac{\mathcal{E}}{R} + \dfrac{U_0}{R_0}\right)\right)
    \cdot e^{-\dfrac{t}{C \rho}}
\end{gather*}
После прошествии времени разрядки конденсатора получим $U(\tau_\text{р.}) = U_\text{г.}$,
тогда:
\begin{gather*}
    U_\text{г.} = \rho \left(\dfrac{\mathcal{E}}{R} + \dfrac{U_0}{R_0}\right)
    + \left(U_\text{з.} - \rho \left(\dfrac{\mathcal{E}}{R} + \dfrac{U_0}{R_0}\right)\right)
    \cdot e^{-\dfrac{\tau_\text{р.}}{C \rho}}
\end{gather*}
Тогда получим время разрядки $\tau_\text{р.}$:
\begin{gather*}
    \tau_\text{р.} = C \rho \cdot ln\left(\dfrac{(U_\text{з.} - U_0)R + (U_\text{з.}
            - \mathcal{E})R_0}{(U_\text{г.} - U_0)R + (U_\text{г.} - \mathcal{E})R_0}\right)
\end{gather*}

Однако в данной работе последовательно к лампе включён резистор сопротивлением
$R_\text{з.} = 3,3$ кОм.
Он нужен для того, чтобы в момент зажигания разряда ток не достигал значений,
при которых электроды лампы разрушаются. Однако заметим,
что в таком случае вольт-амперная характеристика ламы будет выглядеть несколько иначе,
чем указано выше, в результате чего требуется подкорректировать все те формулы,
которые были получены с использованием вольт-амперной характеристики лампы.

Вольт-амперная характеристика ламы будет преобразуется в:
\begin{gather*}
    I(U) = \dfrac{U - U_0}{R_0 + R_\text{з.}}
\end{gather*}
Замена $\rho$ примет вид:
\begin{gather*}
    \dfrac{1}{\rho} = \dfrac{1}{R} + \dfrac{1}{R_0 + R_\text{з.}} \hspace{2em} \rho = \dfrac{R(R_0 + R_\text{з.})}{R + R_0 + R_\text{з.}}
\end{gather*}
Решением второго ДУ станет функция:
\begin{gather*}
    U(t) = \rho \left(\dfrac{\mathcal{E}}{R} + \dfrac{U_0}{R_0 + R_\text{з.}}\right)
    + \left(U_\text{з.} - \rho \left(\dfrac{\mathcal{E}}{R} + \dfrac{U_0}{R_0 + R_\text{з.}}\right)\right)
    \cdot e^{-\dfrac{t}{C \rho}}
\end{gather*}
А время разрядки $\tau_\text{р.}$ будет вычисляться как:
\begin{gather*}
    \tau_\text{р.} = C \rho \cdot ln\left(\dfrac{(U_\text{з.} - U_0)R + (U_\text{з.}
            - \mathcal{E})(R_0 + R_\text{з.})}{(U_\text{г.} - U_0)R + (U_\text{г.} - \mathcal{E})(R_0 + R_\text{з.})}\right)
\end{gather*}

\section*{\centering Практическая часть}

\begin{enumerate}
    \item Снята ВАХ неоновой лампы.
          \begin{figure}[h]
              \centering
              \begin{tikzpicture}
                  \begin{axis}[scale=1.5, xlabel={$I$, мА}, ylabel={$U$, В}]
                      \addplot coordinates{(130.18, 3.24)(135.06, 4.16)(140.74, 5.26)(150.08, 7.09)(160.35, 9.17)(170.38, 11.17)(180.04, 13.44)(195.11, 17.06)};
                      \addplot coordinates{(195.11, 17.06)(190.0, 15.99)(185.0, 14.91)(179.53, 13.69)(170.01, 11.54)(165.02, 10.42)(160.17, 9.36)(150.07, 7.2)(145.03, 6.19)(140.0, 5.19)(135.0, 4.26)(130.0, 3.33)(125.02, 2.51)(120.03, 1.77)(115.02, 1.08)};
                  \end{axis}
              \end{tikzpicture}
              \caption{ВАХ неоновой лампы}
              \label{ВАХ неоновой лампы}
          \end{figure}
    \item Дополнительно проведено определение напряжения зажигания $U_\text{з.}$
          и напряжения гашения $U_\text{г.}$ тлеющего разряда в неоновой лампе.
          \begin{table}[h]
              \centering
              \begin{tabular}{|c|c|c|c|}
                  \hline
                  $U_\text{з.}$, В & $U_\text{з.ср.}$, В & $U_\text{г.}$, В & $U_\text{г.ср.}$, В \\
                  \hline
                  128              &                     & 111,83           &                     \\
                  \cline{1-1}\cline{3-3}
                  129,87           & 128,357             & 112,91           & 112,61              \\
                  \cline{1-1}\cline{3-3}
                  127,2            &                     & 113,09           &                     \\
                  \hline
              \end{tabular}
              \caption{Напряжение зажигания и напряжение гашения}
              \label{Напряжение зажигания и напряжение гашения}
          \end{table}
    \item ВАХ неоновой лампы аппроксимирована линейной зависимостью с помощью метода наименьших квадратов.
          \begin{gather*}
              I(U) = k \cdot U + b = \dfrac{U - U_0}{R_0} \hspace{2em} k = \dfrac{1}{R_0} \hspace{2em} b = \dfrac{U_0}{R_0} \\
              k = 0,00020442\ \text{Ом}^{-1} \hspace{2em} b = -0,023238\ \text{А} \\
              R_0 = \dfrac{1}{k} \hspace{2em} U_0 = -\dfrac{b}{k} \\
              R_0 = 4892\ \text{Ом} \hspace{2em} U_0 = 113\ \text{В}
          \end{gather*}
    \item Построена теоретическая и практическая зависимости $T(C)$ на рис. \ref{Время от ёмкости}.
          \begin{figure}[h]
              \centering
              \begin{tikzpicture}
                  \begin{axis}[scale=1.25, xlabel={$C$, мкФ}, ylabel={$T$, мс}]
                      \addplot coordinates{(2.5e-07, 0.048)(3.3e-07, 0.061)(5e-07, 0.088)(7.5e-07, 0.126)(8.799999999999999e-07, 0.138)(1e-06, 0.162)(1.2499999999999999e-06, 0.2)(1.33e-06, 0.20800000000000002)(1.5e-06, 0.23600000000000002)(2e-06, 0.304)};
                      \addplot coordinates{(2.5e-07, 0.04243240840796665)(3.3e-07, 0.056010779098515984)(5e-07, 0.0848648168159333)(7.5e-07, 0.12729722522389997)(8.799999999999999e-07, 0.1493620775960426)(1e-06, 0.1697296336318666)(1.2499999999999999e-06, 0.21216204203983324)(1.33e-06, 0.22574041273038262)(1.5e-06, 0.25459445044779994)(2e-06, 0.3394592672637332)};
                  \end{axis}
              \end{tikzpicture}
              \caption{Зависимость $T(C)$}
              \label{Время от ёмкости}
          \end{figure}
    \item Построена теоретическая и практическая зависимости $T(R)$ на рис. \ref{Время от сопротивления}.
          \begin{figure}[h]
              \centering
              \begin{tikzpicture}
                  \begin{axis}[scale=1.25, xlabel={$R$, кОм}, ylabel={$T$, мс}]
                      \addplot coordinates{(300000, 0.03)(350000, 0.036000000000000004)(450000, 0.04)(500000, 0.048)(660000, 0.064)(880000, 0.078)(2200000, 0.196)(4300000, 0.376)};
                      \addplot coordinates{(300000, 0.026079192914510274)(350000, 0.03028807866981379)(450000, 0.03843209102198523)(500000, 0.042410680511128666)(660000, 0.05488201975118646)(880000, 0.07164127259859825)(2200000, 0.16927656086995263)(4300000, 0.32236291854802634)};
                  \end{axis}
              \end{tikzpicture}
              \caption{Зависимость $T(R)$}
              \label{Время от сопротивления}
          \end{figure}
    \item Построена теоретическая и практическая зависимости $T(\mathcal{E})$ на рис. \ref{Время от электродвижущей силы}.
          \begin{figure}[h]
              \centering
              \begin{tikzpicture}
                  \begin{axis}[scale=1.25, xlabel={$\mathcal{E}$, В}, ylabel={$T$, мс}]
                      \addplot coordinates{(135, 0.2)(140, 0.134)(145, 0.10200000000000001)(150, 0.084)(160, 0.064)(170, 0.052000000000000005)(180, 0.044)(190, 0.04)(195, 0.037200000000000004)};
                      \addplot coordinates{(135, 0.1607546077267214)(140, 0.11537589353520221)(145, 0.09127920809214284)(150, 0.0760224853477969)(160, 0.057514036347925714)(170, 0.046540293061731054)(180, 0.039194386329852234)(190, 0.03388713712423303)(195, 0.031739660201311755)};
                  \end{axis}
              \end{tikzpicture}
              \caption{Зависимость $T(\mathcal{E})$}
              \label{Время от электродвижущей силы}
          \end{figure}
\end{enumerate}

\end{document}